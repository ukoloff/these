%!TEX root = ../these.tex

\subsection{Точное решение}
\label{sec:pgstcp.solution}

Наконец,
при обработке последнего
<<этажа>> дерева поиска,
префиксов длины
$|\sigma|=m$
(если алгоритм не был остановлен ранее,
например, по исчерпанию времени или
достижению требуемой точности),
процедура расчета нижней оценки уже не применяется,
так как вспомогательная задача
$\mathcal P(\sigma)$
очевидно вырождается.

Вместо этого каждый лист дерева
(префикс длины $m$)
представляет собой допустимое решение
исходной задачи PCGTSP.
Его вес легко считается на основе той же матрицы
$D(\sigma)_{ij}$.
Для этого нужно построить
вспомогательный префикс
$\hat \sigma = \sigma + V_1$,
рассчитать для него матрицу
$D(\hat \sigma)_{ij}$
(она получается квадратной размера $|V_1|$)
и взять ее наименьший диагональный элемент.

При этом можно обновить текущую верхнюю оценку
по формуле
$$
UB \gets \min(UB, \min_i D(\hat \sigma)_{ii})
$$

Решение задачи PCGTSP дается префиксом $\sigma$
длины $m$ минимального веса.
Сам по себе префикс содержит только кластера $V_i$,
через которые проходит оптимальный маршрут,
вершины $v_i \in V_i$
могут быть легко восстановлены стандартным
для динамического программирования образом.
Если в ходе вычислений матрицы
$D(\sigma)_{ij}$
запоминались также индексы,
дающие в нее вклад,
это делается обратным ходом алгоритма.
Если же для экономии памяти
эти индексы не запоминались,
можно повторить расчет матрицы
$D(\sigma)_{ij}$
для (одного) найденного полного префикса $\hat \sigma$,
запоминая индексы,
которые и дадут номера вершин
$v_i$
в оптимальном маршруте $T$.
