%!TEX root = ../these.tex

\section{%
  Алгоритмы решения задачи непрерывной резки
}
\label{sec:ccp.algo}

Предлагаемый алгоритм решения задачи непрерывной резки
(см.~\cite{berlin2019,bi:ccp:ru})
состоит из нескольких шагов,
что хорошо соответствует самой природе
решаемой задачи.

\subsection{%
  О приеме удаления <<внешних>> контуров
  для сокращения трудоемкости решения
}

Для автоматического соблюдения
ограничения предшествования,
мы начинаем с удаления всех контуров,
внутри которых есть вложенные контура,
так, чтобы остались только:
$$
\{C_i | \forall j \ne i: C_j \cap \widetilde C_i = \varnothing \}
$$

В общем случае это приводит к уменьшению
(в некоторых случаях -- существенному)
сложности задачи
(с $N$ до некоторого $N'$),
что в свою очередь
сокращает время счёта
на втором и в особенности третьем
шагах алгоритма.

\subsection{%
  Поиск траектории перемещения инструмента
  для случая непрерывной модели
  описания геометрии контуров
}
\label{sec:ccp.relax}

На этом этапе мы полагаем,
что последовательность обхода контуров
$I = (i_1, i_2, ... i_N)$
задана (фиксирована)
и ищем координаты точек врезки
$M_i \in C_i$
во все контура,
минимизируя полную длину холостого хода
(\ref{eq:air-move-length}).
Для этого,
начальные позиции точек врезки выбираются
произвольным образом
(например, случайно)
и затем положение одной (каждой) из точек
$M_i$
изменяется, а все остальные остаются неподвижны:
$\mathcal{L}(M_i) \to \min$.
Большинство слагаемых в целевой функции
(\ref{eq:air-move-length})
при этом постоянны,
так что сама функция упрощается до
$$
|M_{i-1}M_i|+|M_iM_{i+1}| \to \min_{M_i \in C_i}
$$

Несложный геометрический анализ показывает,
что если точки
$M_{i-1}$
и
$M_{i + 1}$
расположены по разные стороны сегмента контура
$C_i$,
то оптимальное положение точки врезки
$M_i$
оказывается на пересечении с этим сегментом:
$M_i = M_{i-1} M_{i + 1} \cap C_i$
(если, конечно,
такое пересечение существует;
в противном случае
решением будет один из концов сегмента),
см. рис. \ref{fig:pierce-thru}.

Если же точки располагаются
с одной стороны сегмента,
решение легко находится при помощи
\textit{принципа Ферма},
или другими словами правила
<<угол падения равен углу отражения>>
(или опять на одном из концов сегмента),
см. рис. \ref{fig:pierce-fermat}.

\begin{figure}
  \centering
  \subfloat[На пересечении звена ломаной]{
    \label{fig:pierce-thru}
    \tikz[rotate=27]{
        \draw[thick,name path=C1]
            (0,5) node[above] {$C_{i-1}$}
            to[bend right] (1,0);
        \draw[thick,name path=C2]
            (3,0) node[below] {$C_i$}
            to[bend right] (3,2);
        \draw[thick,name path=C3]
            (4,3) node[above] {$C_{i+1}$}
            to[bend right] (5,0);
        \path[name path=L0]
            (0,1.5) -- (6,2);
        \path[name path=Lx]
            (0,1) -- (6,1);
        \fill[name intersections={of=C1 and Lx, name=X}]
            (X-1) coordinate(M1) circle(3pt) node[below] {$M_{i-1}$};
        \fill[name intersections={of=C2 and L0, name=X}]
            (X-1) coordinate(M2) circle(3pt) node[below left]{$M_i$};
        \draw[name intersections={of=C2 and Lx, name=X}]
            (X-1) coordinate(M2x) node[below right]{$M'_i$};
        \fill[name intersections={of=C3 and Lx, name=X}]
            (X-1) coordinate(M3) circle(3pt) node[right] {$M_{i+1}$};
        \draw[dashed]
            (M1) -- (M3);
        \draw[thin,-latex]
            (M2)
            to[bend right] (M2x);
    }
  }
  \subfloat[С использованием \textit{принципа Ферма}]{
    \label{fig:pierce-fermat}
    \tikz[rotate=-12]{
      \draw[thick]
          (0, 0) coordinate(zero) -- (5, 0) coordinate(future) node[right] {$C_i$};
      \fill[black]
          (1.5, 0) circle(3pt) coordinate(middle) node[below left]  {$M_i$}
          (1, 1) circle(3pt) coordinate(from) node[above right] {$M_{i-1}$} ++(-1.5,0) node[above] {$C_{i-1}$}
          (4.5, 2) circle(3pt) coordinate(to) node[above right] {$M_{i+1}$} ++(1.5,0) node[below] {$C_{i+1}$};
      \begin{scope}
          \clip (from) circle(1);
          \draw[thick] (from) ++(0, 3) circle(3);
      \end{scope};
      \begin{scope}
          \clip (to) circle(1.5);
          \draw[thick] (to) ++(3, 4) circle(5);
      \end{scope};
      % \draw[dashed] (from) -- (middle) -- (to);
      \draw[thin] (4.5, -2) circle(0.062) coordinate(mirror) node[right] {$\hat M_{i+1}$};
      \coordinate (opt) at (intersection of zero--future and mirror--from);
      % \draw[thin] (opt) circle(3pt);
      \draw[dotted]
          (mirror) -- (opt)
          (mirror) -- (to);
      \draw[dashed] (from) -- (opt) -- (to);

      \draw[thin,-latex] (middle) to[bend right] (opt) node[below] {$M'_i$};
    }
  }
  \caption{Оптимальное положение точки врезки}
  \label{fig:shift-pierce-point}
\end{figure}

Поиск позиции точки врезки выполняется для всех $N$
контуров поочерёдно,
и весь этот процесс повторяется
пока длина холостого пути
(\ref{eq:air-move-length})
не стабилизируется
с некоторой наперёд заданной точностью $\delta$
(на шаге \ref{alg:ccp-continuous-delta}),
см. Алгоритм~\ref{alg:ccp-continuous}.

\begin{algorithm}
  \caption{CCP :: Релаксация позиций точек врезки}
  \label{alg:ccp-continuous}
  \begin{algorithmic}[1]
    \STATE инициализация $M_i \in C_i, \forall i$ \COMMENT{Случайным образом}
    \STATE $d \gets 0$
    \STATE $d_0 \gets \infty$
    \WHILE {$|d-d_0|>\delta$}
      \label{alg:ccp-continuous-delta}
      \FORALL{$i \in \overline{1,N}$}
        \STATE $M_i \gets \argmin\limits_{M_i \in C_i} |M_{i-1}M_i|+|M_iM_{i+1}|$
        \STATE $d_0 \gets d$
        \STATE $d \gets \mathcal L(M_1, M_2, \dots, M_N)$ \COMMENT{см.~\eqref{eq:air-move-length}}
      \ENDFOR
    \ENDWHILE
    \RETURN $\{M_1, M_2, \dots, M_N\}$
  \end{algorithmic}
\end{algorithm}

На практике весь процесс хорошо сходится
за время
$O(N)$
и поэтому многократно применяется
в качестве подпрограммы на следующем шаге.

\subsection{%
  Алгоритмы комбинаторной оптимизации
  построения маршрута
}

Наиболее вычислительно сложная задача
заключается в поиске перестановки
$I = (i_1, i_2, ... i_N)$,
минимизирующей полную длину холостого хода
$\mathcal{L} \to \min$.
Фактически,
это решение
задачи коммивояжёра
(Traveling Salesman Problem, TSP),
только длина пути вычисляется
не аддитивно,
а при помощи процесса
непрерывной оптимизации,
описанного в разделе~\ref{sec:ccp.relax}.

В данной работе для поиска
такой перестановки применяется
метод переменных окрестностей
(Variable Neighborhood Search,
VNS~\cite{bi:VNS}),
см. Алгоритм~\ref{alg:ccp-discrete}.

\begin{algorithm}
  \caption{CCP :: Дискретная оптимизация}
  \label{alg:ccp-discrete}
  \begin{algorithmic}[1]
    \STATE Инициализация $I = (i_1, i_2, ... i_N)$ \COMMENT{выбирается случайным образом}
    \STATE $k \gets 1$
    \WHILE{$k < k_{max}$}
      \STATE $I' = \argmin\limits_{I' \in \mathcal N^k(I)} \mathcal L(I')$
        \label{alg:ccp-discrete-argmin}
      \IF {$\mathcal L(I')< \mathcal L(I)$}
        \STATE $I \gets I'$
        \STATE $k \gets 1$
      \ELSE
        \STATE $k \gets k+1$
      \ENDIF
    \ENDWHILE
    \RETURN $I$
  \end{algorithmic}
\end{algorithm}

На шаге \ref{alg:ccp-discrete-argmin}
многократно применяется
непрерывная оптимизация из предыдущего этапа:
$$
\mathcal L (I') = \min_{M_1, M_2 \dots M_N}
  \mathcal L (M_1, M_2 \dots M_N | I')
$$

Окрестности
$\mathcal N^k(I)$
различного размера
конструируются разнообразными способами,
например:

\begin{itemize}
  \item
  Все возможные парные перестановки
  (то есть, окрестности размера 1 в смысле транспозиционной метрики)
  \item
  Циклические перестановки 3 контуров.
  Поскольку всего таких перестановок получается
  $O (N ^ 3)$,
  выбираются только те из них,
  в которых задействованные контуры расположены
  в исходной перестановке
  $I = (i_1, i_2, ... i_N)$
  не далее, чем на предопределённом расстоянии
  друг от друга;
  это предопределённое расстояние является
  параметром алгоритма
  \item
  Подобным же образом,
  выбираются циклические перестановки 4 контуров,
  лежащих не далее заданного расстояния
  друг от друга в исходной перестановке
  $I = (i_1, i_2, ... i_N)$
  \item
  Выбирается последовательный блок контуров
  произвольной длины и к нему применяется
  циклический сдвиг
  \item
  Все контуры в последовательном блоке
  контуров
  (произвольной длины)
  переставляются в обратном порядке
  \item
  Перестановка двух последовательных
  (но не смежных) блоков контуров
  \item
  Циклическая перестановка нескольких
  последовательно расположенных
  последовательных блоков контуров
  (произвольной но одинаковой длины)
  \item
  И ещё порядка десяти других способов генерации
  <<близких>> к исходной перестановок
\end{itemize}

Если размер некоторой окрестности
$\mathcal N^k(I)$,
получаемой одним из способов,
оказывается слишком большим
(что приводит к увеличению времени счёта),
он легко может быть ограничен
при помощи введения дополнительного параметра
алгоритма,
подобно тому,
как это сделано для тройных
и четверных циклических перестановок.

Кроме того,
сам метод переменных окрестностей
допускает несколько вариантов применения,
например замена полного перебора
(на шаге~\ref{alg:ccp-discrete-argmin})
на <<первый подходящий>>
или метод Монте-Карло,
но их влияние на скорость
и качество получаемого решения
задачи непрерывной резки
требует дальнейшего исследования.

\subsection*{Восстановление удалённых контуров}

\begin{figure}
  \centering
  \tikz{
      \path
          (-2,0.5) coordinate(M1)
          (-1,1) coordinate(M2)
          (8,2) coordinate (M4)
          (9,4) coordinate (M5);
      ;
      \filldraw[thick,pattern=north east lines]
          (M2) -- +(80:2) -- +(140:2) -- cycle
          (M4) -- +(260:2) -- +(320:2) -- cycle;
      ;
      \filldraw[thick,rounded corners,pattern=north east lines,name path=Ai]
          (0,0) rectangle (7,4)
          (2,2) circle(1)
      ;
      \path[name path=Lx]
          (M2) -- (M4);
      \path[name intersections={of=Ai and Lx, name=Mi}]
          (Mi-3) coordinate (M3)
          (Mi-2) coordinate (M3_5)
      ;
      \fill
          \foreach \i in {1,...,5} {
              (M\i) circle (2pt)
          }
      ;
      \draw [ultra thick]
          (M3_5) circle(3pt) node[above right] {$M_+$}
      ;
      \draw[thin,-latex]
          (M1) node[below] {$M_1$} --
          (M2) node[below] {$M_2$} --
          (M3) node[above right] {$M_3$} --
          (M4) node[right] {$M_4$} --
          (M5) node[left] {$M_5$}
      ;

      % \fill [name intersections={of=Ai and Lx, name=i, total=\t}] [red]
      %     \foreach \s in {1,...,\t} {(i-\s) circle (2pt) node[black,above right] {\s}};
  }
  \caption{Добавление точек врезки во <<внешние>> контура $M_+$}
  \label{eq:extra-pierce-point}
\end{figure}

На предыдущем шаге мы получили
последовательность обхода контуров
и точки врезки для каждого из них,
но только для тех,
которые не содержат других контуров внутри себя.
Теперь необходимо дополнить эту последовательность
оставшимися контурами
(удалёнными на первом шаге)
и точками врезки для них,
причём таким образом,
чтобы ограничение предшествования
оказалось соблюдённым.

Заметим, что маршрут,
полученный к этому моменту
\begin{equation}
  \label{eq:route0}
  \mathfrak R = \left< M_0, M_1, M_2, \dots M_{N'}, M_{N'+1}\right>
\end{equation}
обязательно пересекает все исходные контуры
$C_i$,
потому что для контуров,
сохранённых на первом шаге,
он их явно посещает
по построению шагов 2 и 3,
а для остальных контуров --
ввиду того, что
каждый из них включает в себя
один из посещённых контуров,
а начальная и конечная точка
$M_0$
и
$M_{N + 1}$
всегда выбираются снаружи всех контуров
$C_i$.

Таким образом,
для каждого
<<внешнего>> контура
$C_i$,
который пока не включён в маршрут резки,
мы находим все точки пересечения
с полученным маршрутом
\eqref{eq:route0},
и если таких точек несколько
(что как правило и бывает),
выбираем из них самую последнюю,
то есть посещаемую маршрутом позже всех
других,
см. рис. \ref{eq:extra-pierce-point}.
Сам контур
$C_i$
вставляется в перестановку
в месте,
соответствующем выбранной точке врезки.

После добавления таким образом всех
<<внешних>> контуров
и соответствующих им точек врезки,
мы получаем уже полный маршрут,
который посещает все исходные контуры,
причём внутренние контуры посещаются
строго раньше содержащих их внешних.
Полная длина маршрута
при этой операции,
очевидно,
не меняется.

Легко понять,
что получаемый таким образом полный маршрут
является оптимальным решением исходной задачи
непрерывной резки.
Действительно,
если бы существовал более короткий маршрут,
посещающий все контура,
из него можно было бы просто удалить
точки врезки,
лежащие на
<<внешних>> контурах,
получив тем самым маршрут,
обходящий <<внутренние>> контура
и имеющий ту же,
то есть меньшую длину.
Таким образом, существует лучшее
решение для задачи обхода
части контуров без учёта
ограничений предшествования,
но это по предположению невозможно.

Таким образом,
мы строго выполняем
ограничение предшествования,
тратя на это линейное время
$O(N)$.

На этом выполнение
предложенного эвристического алгоритма
решения задачи непрерывной резки
завершается.
