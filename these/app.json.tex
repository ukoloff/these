%!TEX root = ../these.tex

\chapter{
  JSON-схемы
}
\label{app:json}

В ходе диссертационной работы был
разработан ряд форматов обмена информацией
\cite{bi:dbs-schema},
все они представляют собой современный
\textit{стандарт де-факто}
JSON
\cite{bi:JSON}
и их описание оформлено в виде
JSON-схем
\cite{bi:json-schema}.

\section{Сведения о геометрии деталей и раскроя}

\lstinputlisting[
  language=Java,
  basicstyle=\footnotesize,
  showstringspaces=false,
  numbers=left,
  label={lst:schema-dbs},
  captionpos=b,
  caption=Файл геометрии деталей и раскроя
  ]
{media/json/dbs.json}
\textattachfile[
  description=Файл геометрии деталей и раскроя
]{media/json/dbs.json}{}

\section{Задание на резку}

\lstinputlisting[
  language=Java,
  basicstyle=\footnotesize,
  showstringspaces=false,
  numbers=left,
  label={lst:schema-cut-task},
  captionpos=b,
  caption=Задание на резку
  ]
{media/json/rm-task.json}
\textattachfile[
  description=Задание на резку
]{media/json/rm-task.json}{}

\section{Результат резки}

\lstinputlisting[
  language=Java,
  basicstyle=\footnotesize,
  showstringspaces=false,
  numbers=left,
  label={lst:schema-cut-solution},
  captionpos=b,
  caption=Решение задачи резки
  ]
{media/json/rm-result.json}
\textattachfile[
  description=Решение задачи резки
]{media/json/rm-result.json}{}
