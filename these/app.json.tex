%!TEX root = ../these.tex

\chapter{
  JSON-схемы
}
\label{app:json}

В ходе диссертационной работы был
разработан ряд форматов обмена информацией
\cite{bi:dbs-schema},
все они представляют собой современный
\textit{стандарт де-факто}
JSON
\cite{bi:JSON}
и их описание оформлено в виде
JSON-схем
\cite{bi:json-schema}.

\newcommand{\AddJSON}[3]{
\lstinputlisting[
  language=Java,
  basicstyle=\footnotesize,
  showstringspaces=false,
  numbers=left,
  label={lst:schema-#2},
  captionpos=b,
  caption=#3
  ]
{media/json/#1.json}
\textattachfile[
  author=\theseAuthor,
  mimetype=application/json,
  print=false,
  description=#3
]{media/json/#1.json}{}
}

\section{Сведения о геометрии деталей и раскроя}

\AddJSON{dbs}{dbs}{Файл геометрии деталей и раскроя}

\section{Задание на резку}

\AddJSON{rm-task}{cut-task}{Задание на резку}

\section{Результат резки}

\AddJSON{rm-result}{cut-solution}{Решение задачи резки}
