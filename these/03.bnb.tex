%!TEX root = ../these.tex

\section{Алгоритм ветвей и границ}
\label{sec:pcgtsp.bnb}

Соображения раздела~\ref{sec:pcgtsp.algo}
естественным образом соединяются
в алгоритме ветвей и границ
(Branch-and-Bound, BnB)
классического дизайна.
Дерево поиска строится начиная с корня
$\sigma=V_1$
и обходится при помощи метода поиска в ширину,
см.~Алгоритм~\ref{alg:pcgtsp.bnb}.

\begin{algorithm}
  \caption{BnB :: Главная процедура}\label{alg:pcgtsp.bnb}
  \textbf{Вход}: орграф $G$, кластеры $\mathcal C$, частичный порядок $\Pi$ \\
  \textbf{Выход}: маршрут и его стоимость
  \begin{algorithmic}[1]
      \STATE инициализация $Q =$ пустая очередь
      \STATE начинаем с $Root = V_1$
      \STATE $Q$.push($Root$)
      \WHILE{\NOT $Q$.empty()}
          \STATE берём следующий префикс для обработки: $\sigma = Q$.pop()
          \STATE $process = Bound(\sigma)$ \COMMENT {Получение нижних границ}
          \IF{\NOT $process$}
              \STATE префикс отсекается; \textbf{continue} \COMMENT{Отсечение}
          \ENDIF
          \STATE $UpdateLowerBound(\sigma)$ \COMMENT{Обновление нижних границ}
          \FORALL{$child \in Branch(\sigma)$}
              \STATE помещаем префикс в очередь на обработку $Q$.push($child$)
          \ENDFOR
      \ENDWHILE
  \end{algorithmic}
\end{algorithm}

\begin{algorithm}
  \caption{BnB :: Bound}\label{alg:bnb:bound}
  \textbf{Вход}: префикс $\sigma$ \\
  \textbf{Выход}: подлежит ли префикс обработке?
  \begin{algorithmic}[1]
      \STATE \textbf{global} $D_{ij}^{\mathcal T}$
      \STATE \textbf{global} $Opt^{\mathcal T}$
      \STATE вычисляем кортеж $\mathcal T = \left(V_{i_1},
          \left\{V_{i_1}, V_{i_2}, \dots V_{i_r}\right\}, V_{i_r} \right)$
          \label{alg:bnb:bound:key}
      \STATE $D_{ij} = MinCosts(\sigma)$
          \label{alg:bnb:bound:pfx}
      \IF{$D_{ij}^{(\sigma)} \geqslant D_{ij}^{\mathcal T}[\mathcal T], \forall i, j$}
          \RETURN \FALSE
          \label{alg:bnb:cut:prefix}
      \ENDIF
      \STATE  обновляем веса маршрутов $D_{ij}^{\mathcal T}[\mathcal T]  = \min \left(
          D_{ij}^{\mathcal T}[\mathcal T], D_{ij} \right),
          \forall i, j$
      \STATE  $c_{min} = \min\limits_{i, j} D_{ij}$
      \IF{$\mathcal T \notin Opt^{\mathcal T}$}
          \STATE вычисляем нижнюю границу $Opt^{\mathcal T}[\mathcal T] = \max\left(L_1(\sigma), L_2(\sigma))\right)$
          \label{alg:bnb:bound:sfx}
      \ENDIF
      \STATE $LB = c_{min} + Opt^{\mathcal T}[\mathcal T]$
          \label{alg:bnb:bound:lb}
      \IF{$LB > UB$}
          \RETURN \FALSE
      \ENDIF
      \RETURN \TRUE
  \end{algorithmic}
\end{algorithm}

\begin{algorithm}
  \caption{BnB :: Branch}\label{alg:bnb:branch}
  \textbf{Вход}: префикс $\sigma$ \\
  \textbf{Выход}: список потомков префикса для обработки
  \begin{algorithmic}[1]
      \STATE инициализация $R =$ пустая очередь
      \FORALL{$V \in \mathcal C \setminus \sigma $}
          \STATE $valid =$ \TRUE
          \FORALL{$W \in \mathcal C$}
              \IF{$W \notin \sigma$ \AND $(W, V) \in \Pi$}
                  \STATE $valid =$ \FALSE
                  \STATE \textbf{break}
              \ENDIF
          \ENDFOR
          \IF{$valid$}
              \STATE добавляем новый префикс $R$.push($\sigma+V$)
          \ENDIF
      \ENDFOR
      \RETURN $R$
  \end{algorithmic}
\end{algorithm}

К каждому узлу дерева поиска
$\sigma$
применяется процедура построения нижней оценки $Bound$
(Алгоритм~\ref{alg:bnb:bound}),
выполняющая следующие действия:
\begin{itemize}
  \item
  на шаге~\ref{alg:bnb:bound:key}
  сопоставляем префиксу $\sigma$
  кортеж $\mathcal T(\sigma) $
  \eqref{eq:pcgtsp.cut.key}
  \item
  на шаге~\ref{alg:bnb:bound:pfx}
  находим матрицу~\eqref{eq:pcgtsp.cut.matrix}
  \item
  на шаге~\ref{alg:bnb:cut:prefix}
  отсекаем по условию~\eqref{eq:pcgtsp.cut.prefix}
  \item
  на шаге~\ref{alg:bnb:bound:sfx}
  вычисляем оценки $L_1$ и $L_2$
  (см.~табл.~\ref{tab:pcgtsp.LBs}),
  сохраняя их в глобальной переменной
  $Opt^{\mathcal T}$,
  используя формулу
  $$
  Opt^{\mathcal T(\sigma)} = \max(L_1, L_2)
  $$
  \item
  на шаге~\ref{alg:bnb:bound:lb}
  рассчитываем нижнюю границу
  \item
  наконец, узел отсекается по условию~\eqref{eq:pcgtsp.cut.lb}
\end{itemize}

Префиксы,
которые избежали отсечения,
обрабатываются процедурой ветвления $Branch$
(Алгоритм~\ref{alg:bnb:branch}),
которая пытается удлинить префикс
$\sigma$
на один кластер с соблюдением ограничений предшествования,
как описано в разделе~\ref{sec:pgstcp.branch}.
