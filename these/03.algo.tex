%!TEX root = ../these.tex

\section{Общие соображения}
\label{sec:pcgtsp.algo}

Описываемые в этой главе алгоритмы опираются на несколько общих идей.
Ветвление осуществляется по переменным
$V_i$,
то есть на каждом шаге мы выбираем какой кластер
будет посещен следующим.
Использование ограничений предшествования
$\Pi$
при этом позволяет уменьшить размер дерева поиска,
для некоторых постановок --- существенно.
Каждый узел дерева поиска соответствует
начальной части полного маршрута $T$,
который мы будем называть
\textit{префиксом}
$\sigma=(V_{i_1}, V_{i_2}, \dots V_{i_r})$,
где по постановке
$V_{i_1}=V_1$.
Кластеры, входящие в префикс,
посещаются в порядке $\sigma$,
а все остальные ---
в произвольном порядке,
соблюдая ограничения предшествования.

%!TEX root = ../these.tex

\subsection{Получение нижних оценок}
\label{sec:pgstcp.bound}

Используя подход динамического программирования
(Dynamic Programming, DP),
легко найти кратчайший путь
вдоль $\sigma$
между всеми парами вершин
$(v_1, v_r)$,
где
$v_1\in V_{i_1}, v_r\in V_{i_r}$
\begin{equation}
  \label{eq:pcgtsp.Cmin}
  c_{min}(\sigma)=\min_{\substack{v_1\in V_{i_1} \\ v_r\in V_{i_r}} } \left\{
    cost \left(P_{v_1, v_r}\right) \colon
    P_{v_1, v_r} \text{ --- частичный путь в порядке } \sigma \right\}
\end{equation}

Таким образом, исходная постановка задачи PCGTSP
может быть в каждом узле дерева поиска декомпозирована в две
(более простых) задачи:
\begin{enumerate}
  \item вычисление $c_{min}(\sigma)$
  \item задачу PCGTSP $\mathcal P(\sigma)$, полученную релаксацией исходной задачи PCGTSP следующим образом:
  \begin{itemize}
    \item использует те же ограничения предшествования
    \item содержит те же узлы и кластера,
    за исключением внутренних кластеров префикса
    $V_{i_2}, \dots V_{i_{r-1}}$
    \item ребра исходящие из $V_{i_1}$
    и входящие в $V_{i_r}$ удаляются
    \item веса ребер $c(u,v)$ сохраняются,
    \item для всех пар вершин
    $v_1\in V_{i_1}, v_r\in V_{i_r}$
    добавляются ребра нулевого веса
    $c(v_1, v_r)=0$
  \end{itemize}
\end{enumerate}

\begin{proposition}
\begin{multline}
  \label{eq:pcgtsp.decomposition}
  \min_T \{cost(T)| T \text{ начинается с префикса } \sigma\}
  \geqslant \\
  c_{min}(\sigma) + \min_{T}\{cost(T)| T \text{ -- допустимое решение задачи } \mathcal P(\sigma)\}
\end{multline}
\end{proposition}

\begin{proof}
  Любой маршрут $T$,
  начинающийся с префикса $\sigma$,
  делится на две части ---
  одну вдоль $\sigma$,
  вторую от $V_{i_r}$ до ${V_{i_1}}$.
  Первая имеет вес не менее $c_{min}(\sigma)$
  по определению.
  Вторая может быть дополнена ребром,
  идущим из $V_{i_r}$ в ${V_{i_r}}$
  (имеющим нулевой вес)
  и тем самым давая решение задачи
  $\mathcal P(\sigma)$.
\end{proof}

Нижняя оценка решения исходной задачи PCGTSP,
таким образом,
в каждом узле дерева поиска
$\sigma$
оценивается по формуле \eqref{eq:pcgtsp.decomposition}.
Важно заметить,
что многие задачи $\mathcal  P(\sigma)$
оказываются идентичными для разных $\sigma$
(отличающихся порядком посещения внутренних кластеров),
что позволяет значительно уменьшить общее время вычислений.

В то же время,
$\mathcal P(\sigma)$
все еще представляет собой экземпляр
задачи PCGTSP,
хотя и меньше размером,
чем исходная постановка.
Заменим ее решение в формуле~\eqref{eq:pcgtsp.decomposition}
на нижнюю границу,
полученную тем или иным способом
$$
\min_{T}\{cost(T)| T \text{ -- допустимое решение задачи } \mathcal P(\sigma)\}
\geqslant LB(\mathcal P(\sigma))
$$

В данной диссертационной работе для вычисления оценки
$LB(\mathcal P(\sigma))$
вслед за~\cite{SALMAN2020163}
используется двухступенчатая релаксация:
\begin{enumerate}
  \item $\mathcal P(\sigma) \to$ ATSP
  \item ATSP $\to \mathcal P_{rel}(\sigma)$
\end{enumerate}
и решение задачи $\mathcal P_{rel}(\sigma)$
принимается за нижнюю границу
$LB(\mathcal P(\sigma))$:
\begin{equation}
  \label{eq:pcgtsp.LB}
  \min_T \{cost(T)| T \text{ начинается с префикса } \sigma\}
  \geqslant
  c_{min}(\sigma) + \mathrm{OPT}(\mathcal P_{rel})
\end{equation}

На каждом из двух шагов релаксации используются
несколько методов,
так что в общем случае получается \textit{несколько}
оценок
$\mathrm{OPT}(\mathcal P_{rel})$,
в формулу~\eqref{eq:pcgtsp.LB}
подставляется \textit{наибольшая} из них.
Опишем применявшиеся в данной диссертационной работе
методы релаксации задачи
$\mathcal P(\sigma)$
подробнее.

\subsection*{Сведение PCGTSP к ATSP}



\subsection*{Вторая релаксация ATSP }

%!TEX root = ../these.tex

\subsection{Отсечение}
\label{sec:pgstcp.cut}

Для каждого узла дерева поиска,
задаваемого префиксом
$\sigma=(V_{i_1}, V_{i_2}, \dots V_{i_r})$,
алгоритм принимает решение
о возможном отсечении.
Для этого используются два соображения:

\begin{enumerate}
  \item На основе минимальной длины $c_{min}(\sigma)$
  \item На основе нижней оценки $LB(\sigma)$~\eqref{eq:pcgtsp.LB3}
\end{enumerate}

Рассмотрим их подробнее.

\subsubsection{%
Отсечение по длине путей вдоль префикса
}

Для быстрого инкрементального расчета
минимального пути вдоль префикса
$c_{min}(\sigma)$
методом динамического программирования
на самом деле требуется рассчитывать
и запоминать целую матрицу
\begin{equation}
  \label{eq:pcgtsp.cut.matrix}
D(\sigma)_{ij} = \min\left\{
  cost(P_{v_i,v_j})\colon
  v_i \in V_{i_1},
  v_j \in V_{i_r},
  P_{v_i,v_j} \text{ -- путь $v_i$-$v_j$ в порядке } \sigma
\right\}
,
\end{equation}
потому что она легко вычисляется
инкрементально на основе такой же матрицы
родительского префикса $\sigma': |\sigma| = |\sigma'|+1$,
и очевидно
\begin{equation}
  \label{eq.pcgtsp.c_min.dp}
  c_{min}(\sigma) = \min_{ij} D(\sigma)_{ij}
  .
\end{equation}

Далее, как сказано выше,
все префиксы $\sigma$,
отличающиеся только порядком
посещения внутренних кластеров,
по построению имеют одну и ту же задачу
$\mathcal P(\sigma)$,
а значит одну и ту же ее нижнюю оценку.
Построим кортеж
\begin{equation}
  \label{eq:pcgtsp.cut.key}
  \mathcal T(\sigma) = (V_{i_1},
  \left\{V_{i_1}, V_{i_2}, \dots V_{i_r} \right\}, V_{i_r})
\end{equation}
и множество
$
\mathbb T(\sigma) =\{\sigma' \colon \mathcal T(\sigma')=\mathcal T(\sigma)\}$
всех префиксов,
дающих ту же вспомогательную задачу
$\mathcal P(\sigma')\equiv \mathcal P(\sigma)$.
На этом множестве найдём минимальную
(поэлементно)
матрицу расстояний
$$
D(\mathbb T(\sigma))_{ij} =
  \min_{\sigma' \in \mathbb T(\sigma)} D(\sigma')_{ij}
$$

Несложно понять, что если
\begin{equation}
  \label{eq:pcgtsp.cut.prefix}
  D(\sigma)_{ij} > D(\mathbb T(\sigma))_{ij},
  \forall i, j
  ,
\end{equation}
то вес любого пути,
начинающего префиксом $\sigma$
может быть уменьшен простой
перестановкой кластеров внутри этого префикса,
и следовательно заведомо
не является оптимальным решением задачи PCGTSP.
Значит, этот префикс может быть
безопасно отброшен и не участвовать в
дальнейших вычислениях.

\subsubsection{Отсечение по нижней оценке}

Если же условие~\eqref{eq:pcgtsp.cut.prefix}
не соблюдается,
алгоритм вычисляет
$c_{min}$
по формуле~\eqref{eq.pcgtsp.c_min.dp},
строит вспомогательную задачу
$\mathcal P(\sigma)$,
как описано в разделе~\ref{sec:pgstcp.bound}
(если она не была построена ранее,
для некоторого префикса
$\sigma' \in \mathbb T(\sigma)$)
и рассчитывает полную нижнюю оценку
$LB(\sigma)$
по формуле~\eqref{eq:pcgtsp.LB3}.

Вторая
(и основная)
стратегия отсечения заключается в отбрасывании узлов дерева поиска,
для которых
\begin{equation}
  \label{eq:pcgtsp.cut.lb}
  LB(\sigma)>UB
  ,
\end{equation}
где
$UB$ --- некоторая верхняя оценка на вес оптимального решения.
Разумно в качестве такой оценки
взять вес
\textit{любого}
решения исходной задачи PCGTSP,
полученного произвольным образом.
В данной диссертационной работе
для получения такого решения
используется эвристика
PCGLNS~\cite{KKP-optima2020},
которая запускается однократно
перед началом работы алгоритма
и за короткое время
(буквально несколько секунд)
дает решение,
близкое к оптимальному.

Чем ниже верхняя граница $UB$,
тем очевидно больше узлов дерева поиска
будет отброшено и тем меньше будет
полное время работы алгоритма.

%!TEX root = ../these.tex

\subsection{Ветвление}
\label{sec:pgstcp.branch}


%!TEX root = ../these.tex

\subsection{Обновление нижней оценки}
\label{sec:pgstcp.upd}

После того, как
обработаны все префиксы
$\sigma$
некоторой длины $r$
(алгоритм обрабатывает
префиксы в порядке увеличения длины,
начиная с корня
$\sigma = V_1$
длины 1,
до листьев длины $m$),
может быть вычислена текущая
нижняя оценка для слоя $r$:
\begin{equation}
  \label{eq:pcgtsp.lb.layer}
  LB_{|r|}=\min_{|\sigma|=r} LB(\sigma)
\end{equation}
по всем префиксам длины $r$,
не подвергшимся отсечению,
как описано в разделе~\ref{sec:pgstcp.cut}.

Полная нижняя оценка при этом обновляется по формуле
\begin{equation}
  \label{eq:pcgtsp.lb.total}
  LB \gets \max(LB, LB_{|r|})
\end{equation}

