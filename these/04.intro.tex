%!TEX root = ../these.tex

Основная сложность интеграции
новых алгоритмов оптимальной маршрутизации
режущего инструмента для машин листовой
термической резки с ЧПУ,
включая описанные выше в данной
диссертационной работе,
является необходимость взаимодействия
большого количества программных подсистем,
выполняющих разные задачи ---
геометрическое проектирование деталей и листов,
раскрой,
собственно резка по полученной раскройной карте,
генерация УП,
визуализация всех этапов проектирования и т.п.
Все эти подсистемы писались и пишутся в разное время,
командами разной квалификации,
с разными целями,
по разным принципам,
иногда <<на скорую руку>>,
на разных языках программирования,
для разных платформ, включая разные операционные системы.

Такое положение дел приводит,
среди прочего к тому,
что потоки данных между подсистемами устроены хаотически,
используют всевозможные форматы данных,
от простейших, созданных \textit{ad hoc},
до сложнейших,
например DXF
(Drawing Interchange / Exchange  Format)
от Autodesk~\cite{bi:DXF}
или XML
(eXtensible Markup Language)
\cite{bi:XML}.
Точные спецификации форматов файлов
могут быть утеряны
(или даже никогда не созданы)
или наоборот ---
оказываются слишком сложными,
для того, чтобы быстро написать код,
осуществляющий чтение или корректную запись такого файла.

Кроме того,
это приводит к необходимости
дублирования усилий,
когда разным командам приходится
писать код для чтения или записи
одного и того же формата,
зачастую на разных языках программирования.
