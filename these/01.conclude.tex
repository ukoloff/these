%!TEX root = ../these.tex

\section{%
  Выводы по Главе \ref{ch:review}
}
\label{sec:review.conclude}

Анализ состояния вопросов автоматизации технологической подготовки УП
для машин листовой термической резки с~ЧПУ
позволяет сделать следующие выводы:

\begin{enumerate}
  \item
  Задача оптимальной маршрутизации режущего инструмента
  представляет собой чрезвычайно сложную задачу непрерывно-дискретной оптимизации,
  требующую решения множества теоретических и практических вопросов,
  включая:
  \begin{itemize}
    \item Использование различных техник резки, помимо стандартной;
    \item Учет разнообразных технологических ограничений на маршрут резки.
  \end{itemize}
  \item Несмотря на большой объем уже проведенных исследований,
  продолжение изучения задачи представляет большой интерес
  как с теоретической точки зрения,
  так и для практического применения,
  в частности:
  \begin{itemize}
    \item Разработка алгоритмов, решающих задачи разных классов,
    имея в виду решение задачи прерывистой резки ICP
    \item Развитие подходов к использованию непрерывной оптимизации
    в противовес чисто дискретным
    \item Оценка качества решений,
    получаемых эвристиками и метаэвристиками
  \end{itemize}
\end{enumerate}
