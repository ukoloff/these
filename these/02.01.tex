%!TEX root = ../these.tex

\section{Постановка задачи}

Рассмотрим евклидову плоскость
$\mathbb R ^ 2$
и на ней фигуру
$B$
(в большинстве практических случаев -- прямоугольник),
ограниченную замкнутым контуром.
Это -- модель листового материала,
подлежащего резке.
Пусть
$N$
попарно непересекающихся плоских контуров
$\{C_1, C_2, ... C_N\}$
расположены внутри
$B$,
ограничивая
$n$
деталей
$\{A_1, A_2 ... A_n\}$.
Деталь может быть ограничена
одним или несколькими контурами
(одним внешним и несколькими отверстиями),
так что в общем случае
$n \leqslant N$.

Контуры
$C_i$
могут быть произвольной формы,
но мы будем рассматривать только
состоящие из
(конечного числа)
отрезков прямых линий и дуг окружностей,
так как именно такие геометрические примитивы
поддерживаются программным обеспечением
современных машин термической резки с ЧПУ.
Частный случай,
когда контура состоят только
из отрезков прямых,
сводится к одному из вариантов
задачи обхода прямоугольников
(Touring Polygon Problem, TPP),
см.
\cite{bi:TPP}.
