%!TEX root = ../these.tex

\section{%
  Постановка задачи CCP
}
\label{sec:ccp.stmt}

Рассмотрим евклидову плоскость
$\mathbb R \times \mathbb R$
и на ней фигуру
$\mathcal B$
(в большинстве практических случаев -- прямоугольник),
ограниченную замкнутым контуром.
Это -- модель листового материала,
подлежащего резке.
Пусть
$N$
попарно непересекающихся плоских контуров
$\{C_1, C_2, ... C_N\}$
расположены внутри
$\mathcal B$,
ограничивая
$n$
деталей
$\{A_1, A_2 ... A_n\}$.
Деталь может быть ограничена
одним или несколькими контурами
(одним внешним и несколькими отверстиями),
так что в общем случае
$n \leqslant N$.

Контуры
$C_i$
могут быть произвольной формы,
но мы будем рассматривать только
состоящие из
(конечного числа)
отрезков прямых линий и дуг окружностей,
так как именно такие геометрические примитивы
поддерживаются программным обеспечением
современных машин термической резки с ЧПУ.
Частный случай,
когда контура состоят только
из отрезков прямых,
сводится к одному из вариантов
задачи обхода прямоугольников
(Touring Polygon Problem, TPP),
см.
\cite{bi:TPP}.

Далее,
внутри
$\mathcal B$
(как правило, на границе)
выберем две точки и обозначим их
$M_0$, $M_{N + 1}$
(почти всегда $M_0 = M_{N + 1}$),
которые будут использоваться
как начало и конец
маршрута резки.

Задача непрерывной резки
(Continuous Cutting Problem, CCP)
состоит в поиске:
\begin{enumerate}
\item
$N$ штук точек врезки $M_i \in C_i, i \in \overline{1, N}$
\item
Последовательности обхода контуров
$C_i$,
то есть перестановки
$N$
элементов
$I = (i_1, i_2, ... i_N)$
\end{enumerate}

Результатом решения задачи будет являться маршрут
\begin{equation}
  \mathfrak R =
  \left< M_0, M_{i_1}, M_{i_2}, \dots M_{i_N}, M_{N + 1} \right>
\end{equation}
Целевая функция в данном случае сильно упрощается
по сравнению с общей задачей маршрутизации резки
и сводится фактически к минимизации длины холостого хода:

\begin{equation}
  \label{eq:air-move-length}
  \mathcal{L} = \sum_{j=0}^N|M_{i_j}M_{i_{j+1}}|
\end{equation}
$$
\mathcal{L} \to \min
$$
где, для простоты записи мы полагаем
$M_{i_0} = M_0$,
$M_{i_{N + 1}} = M_{N + 1}$.

Кроме того,
мы потребуем,
чтобы искомое решение задачи
удовлетворяло
описанному выше
ограничению предшествования.

Хотя контуры
$C_i$
по условию не пересекаются,
они могут быть вложены друг в друга:
\( \widetilde C_a \subset \widetilde C_b \),
где
$\widetilde C_a$
обозначает 2-мерную фигуру,
ограниченную контуром
$C_a$
(в более традиционных обозначениях
$C_a = \partial \widetilde C_a$).
В общей задаче маршрутизации
режущего инструмента это
соответствует двум разным случаям
(наличие отверстий в деталях с одной стороны
и размещение меньших деталей в отверстиях больших),
но в нашем случае оба этих
варианта обрабатываются одинаково.

Если один контур расположен внутри другого,
то внутренний должен быть вырезан
(посещён)
ранее, чем внешний:
\( \widetilde C_a \subset \widetilde C_b \Rightarrow i_a < i_b \),
в перестановке
$I = (i_1, i_2, ... i_N)$.
Таким образом,
множество допустимых перестановок ограничено.
