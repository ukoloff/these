%!TEX root = ../these.tex

\section{%
Использование модели обобщенной задачи коммивояжера
с ограничениями предшествования
PCGTSP
для формализации задачи маршрутизации
}
\label{sec:pcgtsp.intro}

Обобщенная задача коммивояжера
(Generalized Traveling Salesman Problem, GTSP)
-- широко известная задача комбинаторной оптимизации,
впервые сформулированная в основополагающей статье Шриваставы и др.
\cite{SKGS1969}
и привлекшая внимание многих исследователей,
см., например, обзор в
\cite{GutinPunnen2007}.
В GTSP для заданного взвешенного орграфа
$ G = (V, E, c) $
и разбиения
$ V_1 \cup \ldots \cup V_m $
набора узлов $V$ графа $G$ на непустые взаимно непересекающиеся кластеры
требуется найти замкнутый тур с минимальной стоимостью,
который посещает каждый кластер
$V_i$
в точности один раз.

В этой главе рассматривается
обобщенная задача коммивояжера с ограничениями предшествования
(Precedence Constrained Generalized Traveling Salesman Problem, PCGTSP),
в которой необходимо посещать кластеры
в соответствии с заранее заданным частичным порядком.
Эта модификация задачи GTSP имеет множество практических применений, среди которых задачи
оптимизации траектории инструмента для станков с ЧПУ  \cite{CASTELINO2003173},
минимизации времени холостого хода в процессе раскроя листового металла \cite{bi:RoMa,Makarovskikh20181171},
настройки координатно-измерительного оборудования \cite{SALMAN2016138},
оптимизации траектории при множественном сверлении отверстий \cite{DEWIL2019}.

Задача GTSP является обобщением классической задачи коммивояжера
(Traveling Salesman Problem, TSP),
поэтому она также является NP-трудной
даже на евклидовой плоскости,
будучи параметризованной количеством кластеров
$m$
\cite{Papa77}.
В то же время хорошо известная схема динамического программирования Хелда и Карпа
\cite{HeldKarp1962},
адаптированная к GTSP,
обладает трудоемкостью
$ O (n ^ 3m ^ 2 \cdot 2 ^ m) $,
то есть
GTSP принадлежит классу FPT
относительно параметризации количеством кластеров.
Более того, при
$ m = O (\log n) $
её оптимальное решение может быть найдено за полиномиальное время.
Исследования в области алгоритмического анализа задачи GTSP
развивались по нескольким основным направлениям.

Первый подход основан на сведении исходной задачи
к подходящей постановке асимметричной задачи коммивояжера
(ATSP)
и последующем решении полученной вспомогательной задачи
\cite{LaporteSemet1999,NoonBean1993}
Несмотря на математическое изящество, этот подход не свободен от ряда известных недостатков:
\begin{enumerate}
  \item
  получаемые в результате такого сведения постановки задачи ATSP
  обладают специфической структурой,
  затрудняющей их численное решение даже на современных MIP-решателях,
  таких как Gurobi и CPLEX;
  \item
  даже близкие по функционалу к оптимальным
  приближенные решения вспомогательной задачи ATSP
  могут соответствовать недопустимым решениям исходной задачи
  \cite{KaraGut2012}.
\end{enumerate}

Другой известный подход связан
с разработкой точных алгоритмов для частных случаев задачи GTSP
и приближенных алгоритмов с теоретическими оценками,
включая алгоритмы ветвей и границ
(см., например, \cite{FishGonToth1997, Yuan2020})
и полиномиальные приближенные схемы (PTAS)
\cite{FerGriSit2006, KhN-PSIM2017}.

Наконец,
третий подход заключается в разработке
новых и адаптации известных эвристик и метаэвристик.
Среди известных результатов в этом направлении выделяются:
гибридный алгоритм Гутина и Карапетяна \cite{Gutin-2010},
адаптация известного солвера Лина-Кернигана-Хельсгауна \cite{Helsgaun-2015}
и метаэвристика адаптивного поиска в больших окрестностях
(Adaptive Large Neighborhood Search, ALNS)
\cite{SMITH20171},
обладающая рекордной на сегодняшний день практической производительностью.

К сожалению,
алгоритмические результаты для рассматриваемой
в данной главе задачи PCGTSP
до сих пор остаются немногочисленными и исчерпываются
приведенным ниже списком.
\begin{enumerate}
  \item
  Эффективные алгоритмы для специальных ограничений предшествования
  типа Баласа
  \cite{Balas-Sim2001, ChenKhKh2016, CKK-IFAC2016}
  и ограничений предшествования, приводящих к
  квази- и псевдопирамидальным оптимальным маршрутам
  \cite{KhN-OPTA2018,KhN-AMAI-2020},
  \item
  Общий подход к выводу нижних оценок в методе ветвей и границ
  \cite{SALMAN2020163},
  \item
  Новый метаэвристический солвер PCGLNS
  \cite{KKP-optima2020, bi:PCGLNS},
  развивающий результаты, полученные в
  \cite{SMITH20171}
  для GTSP.
\end{enumerate}
