%!TEX root = ../these.tex

\section{Современное состояние исследований}
\label{sec:cut.refs}

Если говорить в целом о проблеме оптимизации траектории инструмента для технологического оборудования листовой
резки с ЧПУ,
то в настоящее время не существует единой теоретической базы для решения этой проблемы.
Существуют отдельные научные группы,
которые занимаются исследованием частных случаев проблемы резки.
Кроме того,
в состав CAD/CAM систем,
предназначенных для проектирования раскроя и управляющих программ для машин листовой резки с ЧПУ,
входят отдельные модули,
позволяющие решать некоторые оптимизационные задачи,
например минимизацию холостого хода инструмента,
которые, однако не обеспечивают соблюдение
технологических требований резки материала на машинах с ЧПУ
и не позволяют получать маршруты резки,
близкие к оптимальным с точки зрения интегрированного критерия стоимости резки
с учетом рабочего хода инструмента, затрат на врезку и т.п.
Вместе с тем в сочетании с интерактивными методами проектирования
они обеспечивают рациональные и технологически допустимые варианты траекторий инструмента машины с~ЧПУ.
Следует подчеркнуть, что алгоритмы,
реализованные в коммерческом программном обеспечении,
не описываются в научной литературе.

В нашей стране первые работы по оптимизации проектирования маршрута листовой резки на машинах с~ЧПУ
были опубликованы проф. М.А.~Верхотуровом
(Уфа)
\cite{bi:верхотуров2008,bi:верхотуров2008цепь}
и проф. В.Д.~Фроловским
(Новосибирск)
\cite{bi:Ganelina,bi:пушкарева,bi:фроловский2005}.
Авторы использовали
простую модель, эквивалентную классической задаче коммивояжера.
В последние годы появилось несколько публикаций,
проводимых под руководством проф. А.В.~Панюкова
(Челябинск) по
этой тематике
\cite{bi:Makarovskikh2019Jan,Makarovskikh20181171,bi:Makarovskikh2019Jan}.
Эти работы представляют более теоретический интерес
в смысле общей маршрутизации в графах,
поскольку получаемые траектории не всегда могут быть реализованы на машинах листовой резки с ЧПУ.
