%!TEX root = ../these.tex

\section{%
Современное состояние проблемы исследования
и~применение алгоритмов маршрутизации для автоматизированного
проектирования управляющих программ
}
\label{sec:cut.refs}

Если говорить в целом о проблеме оптимизации траектории инструмента для технологического оборудования листовой
резки с ЧПУ,
то в настоящее время не существует единой теоретической базы для решения этой проблемы.
Существуют отдельные научные группы,
которые занимаются исследованием частных случаев проблемы резки.
Кроме того,
в состав CAD/CAM систем,
предназначенных для проектирования раскроя и управляющих программ для машин листовой резки с ЧПУ,
входят отдельные модули,
позволяющие решать некоторые оптимизационные задачи,
например минимизацию холостого хода инструмента,
которые, однако не обеспечивают соблюдение
технологических требований резки материала на машинах с ЧПУ
и не позволяют получать маршруты резки,
близкие к оптимальным с точки зрения интегрированного критерия стоимости резки
с учетом рабочего хода инструмента, затрат на врезку и т.п.
Вместе с тем в сочетании с интерактивными методами проектирования
они обеспечивают рациональные и технологически допустимые варианты траекторий инструмента машины с~ЧПУ.
Следует подчеркнуть, что алгоритмы,
реализованные в коммерческом программном обеспечении,
не принято описывать в научной литературе.

В нашей стране первые работы по оптимизации проектирования маршрута листовой резки на машинах с~ЧПУ
были опубликованы проф. М.А.~Верхотуровым
(Уфа)
\cite{bi:верхотуров2008,bi:верхотуров2008цепь,bi:верхотуров2020}
и проф. В.Д.~Фроловским
(Новосибирск)
\cite{bi:Ganelina,bi:пушкарева,bi:фроловский2005}.
Авторы использовали
простую модель, эквивалентную классической задаче коммивояжера.
В последние годы появилось несколько публикаций,
проводимых под руководством проф. А.В.~Панюкова
(Челябинск) по
этой тематике
\cite{bi:Makarovskikh2019Jan,Makarovskikh20181171,bi:Makarovskikh2019Other}.
Эти работы представляют более теоретический интерес
в смысле общей маршрутизации в графах,
поскольку получаемые траектории не всегда могут быть
реализованы на машинах листовой резки с ЧПУ.
Можно отметить также работы, проведенные в Перми
\cite{bi:мурзакаев2015построение,bi:мурзакаев2015применение,bi:файзрахманов2015,bi:бурылов2016}.

Из зарубежных конкурентов следует особо выделить группу ученых из Бельгии
\cite{bi:Dewil2014,bi:dewil-review,bi:Dewil2015Mar,bi:Dewil2015}.
Они занимаются
оптимизацией траектории инструмента преимущественно для лазерного оборудования и разработкой
алгоритмов для двух классов задач с дискретными моделями ---
GTSP и ECP.
Ряд исследователей из Китая, Гонконга,
Японии и др. стран также периодически публикуют свои результаты
(см., например,
\cite{bi:Kandasamy2020Mar,bi:Li2020Feb,bi:Vicencio,Helsgaun-2015,bi:Ezzat2014Mar,bi:Ye2013,bi:Yun2013May,bi:VijayAnand2015Feb}).
Однако они, как правило,
касаются разработки отдельных алгоритмов только для одного из вышеприведенных в
разделе~\ref{sec:cut.class}
классов задач, при этом часто не учитывают важные технологические
ограничения листовой резки на машинах с~ЧПУ.
В частности, учет термических деформаций заготовок и искажение их
геометрических размеров не являются предметом большинства исследований,
что делает эти работы в достаточной степени академическими
и не в полной мере приемлемыми для практики.
Вместе с тем, эти работы вносят свой вклад в теорию и практику
экстремальных задач маршрутизации инструмента машин листовой резки с~ЧПУ.

В целом  можно отметить,
что за рамками современных исследований отечественных и зарубежных коллег,
в основном, остаются следующие принципиальные моменты:
\begin{itemize}
  \item
  Разработка алгоритмов, обеспечивающих получение глобального оптимума оптимизационной задачи маршрутизации инструмента.
  \item
  Адекватный учет тепловых искажений заготовок при термической резке материала на машине с ЧПУ.
  \item
  Разработка комплексного подхода к решению оптимизационных задач
  всех вышеперечисленных в разделе~\ref{sec:cut.class}
  классов,
  в особенности задачи прерывистой резки ICP,
  для различного технологического оборудования листовой фигурной резки с ЧПУ.
  \item
  Разработка алгоритмов решения задач нескольких классов
  с применением специальных техник резки и ориентированных на минимизацию стоимости процесса резки,
  а не только на минимизацию холостого хода инструмента.
  \item
  Разработка алгоритмов,
  позволяющих эффективно решать задачи с непрерывными моделями
  (CCP, SCCP),
  для которых точки врезки в материал могут выбираться из континуальных множеств на плоскости.
  \item
  Разработка оценок вычислительной сложности и точности
  разрабатываемых алгоритмов для практических задач маршрутизации инструмента
  с дополнительными ограничениями.
\end{itemize}

Применение эффективных классических метаэвристических
алгоритмов дискретной оптимизации
(метод эмуляции отжига,
метод муравьиной колонии,
эволюционные алгоритмы,
метод переменных окрестностей и др.)
для дискретных моделей оптимизации траектории инструмента машин с~ЧПУ
возможно только при адаптации этих алгоритмов к~требованиям
технологических ограничений листовой резки.
Однако, эта адаптация наталкивается на серьёзные проблемы,
связанные с невозможностью учета некоторых технологических ограничений при
<<лобовом>> применении упомянутых методов
\cite{bi:panteleev2009,bi:Reihaneh2012,bi:Karapetyan2011,bi:gendreau2010handbook,bi:Ghilas2016Aug,Helsgaun-2015}.

Таким образом,
необходимость в создании новых оптимизационных постановок задач,
алгоритмов и программного обеспечения представляется
доминантой развития методов решения
задачи оптимальной маршрутизации
режущего инструмента для
машин термической резки с~ЧПУ.
