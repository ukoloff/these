%!TEX root = ../these.tex

\section{
  Использование открытых форматов файлов данных для взаимодействия подсистем
}
\label{sec:json.files}

Хотя задача полной унификации
используемых форматов файлов
вряд ли может быть
\textit{полностью}
решена для современных крупных программных систем,
тем не менее,
в последние десятилетия развития технологий
разработки программного обеспечения
накоплены подходы,
позволяющие значительно снизить остроту проблем,
прежде всего за счет продуманного использования
открытых форматов хранения и обмена данными.
При этом важны как функциональность формата,
то есть то, какие именно данные он содержит,
так и организация,
то есть представление
хранимых данных.

\subsection{Выбор открытого формата представления геометрической информации}

Например, для хранения и обмена геометрической информацией
в САПР~<<Сириус>>
используется унаследованный двоичный формат DBS
\cite{bi:DBS},
который обладает важными достоинствами:
\begin{itemize}
  \item
  эффективное хранение больших объемов геометрической информации
  за счет хранения массивов вещественных чисел в формате IEEE~754~\cite{bi:IEEE754} float32;
  \item
  возможность добавления новых типов записей для хранения ранее не предусмотренной информации;
  расширяемость формата
  \item
  механизм создания копий деталей и геометрических преобразований над ними.
\end{itemize}

В то же время,
его недостатки тоже оказываются
значительны,
практически блокируя его использование
для обмена информацией между
программным обеспечением разных команд разработчиков.
Работа с ним сопряжена с рядом сложностей,
прежде всего:
\begin{itemize}
  \item
  Сложность чтения двоичного формата,
  особенно в некоторых языках программирования.
  \item
  Структура DBS-файла,
  предназначенная для эффективного хранения,
  сильно отличается от удобного внутреннего представления
  геометрии;
  требуется нетривиальное преобразование при чтении файла.
  \item
  Формат DBS создавался в том числе для экономии памяти,
  как дисковой, так и оперативной,
  что более неактуально;
  отказ от этого позволяет резко упростить процедуры экспорта--импорта.
\end{itemize}

В рамках данной диссертационной работы
в целях упрощения взаимодействия различных подсистем
проводилось исследование возможности замены
формата DBS
на другой формат хранения геометрической информации.
При этом предъявлялись следующие требования:
\begin{itemize}
  \item Текстовый формат для упрощения обработки на актуальных языках программирования
  \item Простота чтения и записи
  \item Возможность хранения геометрической информации, содержащейся в DBS
  \item Самодокументируемость:
  возможность прочитать файл даже без знания спецификации
  \item Расщиряемость:
  возможность расширить состав хранимой информации прозрачным для существующих программ образом
\end{itemize}

В результате сравнения нескольких
наиболее популярных открытых форматов
(JSON~\cite{bi:JSON},
YAML~\cite{bi:YAML},
XML~\cite{bi:XML},
TOML~\cite{bi:TOML},
Tree~\cite{bi:tree.d},
\dots)
был выбран формат
JavaScript Object Notation
(JSON
\cite{bi:JSON}).
К его преимуществам можно отнести:
\begin{itemize}
  \item очень прост:
  всего $\approx 30$ правил в грамматике,
  \item готовые библиотеки для чтения и записи
  для практически всех современных языков,
  \item может формироваться даже
  без использования специализированных библиотек,
  \item является стандартом де-факто во многих
  современных приложениях для обмена данными,
  \item достаточно выразителен:
  позволяет представить любую структуру данных,
  \item расширяем:
  позволяет добавлять данные,
  не нарушая существующую структуру.
\end{itemize}

Формат JSON
поддерживает всего четыре элементарных типа данных:
\begin{enumerate}
  \item число
  \begin{itemize}
    \item как целое
    \item так и вещественное
  \end{itemize}
  \item строка (в двойных кавычках)
  \item Булево значение
  \begin{itemize}
    \item \textit{true}
    \item \textit{false}
  \end{itemize}
  \item отсутствие значения - \textit{null}
\end{enumerate}
и две структуры данных:
\begin{enumerate}
  \item массив $[\,]$ произвольных элементов
  \item объект $\{\,\}$ --- неупорядоченный набор пар ключ--значение
\end{enumerate}

К недостаткам формата JSON
можно отнести прежде всего отсутствие
возможности размещения комментариев,
а также поддержки других
типов данных, в особенности даты и времени.
Однако,
для этой цели существуют так называемые
\textit{расширения} JSON,
которые могут быть при необходимости использованы,
см. например~\cite{bi:JSON5}.

Для хранения и обмена информацией о геометрии деталей
и раскройной карты
была разработана наиболее простая схема JSON,
не включающая сложности с копиями деталей и геометрическими
преобразованиями.

Пример такого файла для простейшей раскройной карты
приведён в Листинге~\ref{lst:json-dbs}.
Она содержит прямоугольный лист размером
$500 \times 700$
и одну деталь в форме кольца диаметрами
$200$ (внешний)
и $100$ (внутренний).
Контуры представлены в виде набора точек
на евклидовой плоскости
$\mathbb R \times \mathbb R$
двумя координатами
($x$ и $y$),
третье число представляет собой кривизну участка контура,
для отрезка прямой она равна $0$,
для дуги в $180^\circ$ --- $\pm 1$,
а в общем случае --- $tg \frac{\varphi}4$,
где $\varphi$ --- центральный угол дуги.

\lstinputlisting[
    language=Java,
    basicstyle=\footnotesize,
    showstringspaces=false,
    numbers=left,
    label={lst:json-dbs},
    captionpos=b,
    caption=JSON-файл с геометрией простейшей раскройной карты
    ]
    {media/nesting.json}

\subsection{Разработка спецификаций JSON}

Опыт практического использования
формата JSON показал преимущества его
использования.
Например,
разработчики
программного обеспечение \textit{RouterManager},
реализующего <<жадный>> алгоритм
\cite{bi:greedy}
и алгоритм на основе динамического программирования
схемы Беллмана
\cite{bi:RoMa},
полностью отказались от использования формата
DBS~\cite{bi:DBS}
и перешли к использованию формата JSON.
Более того,
последний стал использоваться и для других файлов данных:

\begin{itemize}
  \item \textit{Файл задания на резку}.
  Содержит геометрическую информацию,
  подобно формату DBS,
  но в другом представлении ---
  контуры деталей организованы
  не в плоский список,
  а дерево для явного представления
  ограничений предшествования.
  Здесь же описывается набор допустимых
  точек врезки $M_i$,
  полученных дискретизацией контуров,
  поскольку все алгоритмы решают задачу GTSP.
  Кроме того,
  сюда добавлены некоторые параметры
  процесса резки,
  например скорости резки
  $V_{on}$
  и холостого хода
  $V_{off}$,
  координаты начала / конца маршрута и т.п.
  \item \textit{Файл результатов резки}.
  Содержит подробное описание маршрута режущего
  инструмента --- активные сегменты,
  из которых он состоит со ссылками
  на исходные контура деталей,
  сегменты холостого хода,
  интегральные характеристики,
  включая длины реза
  $L_{on}$
  и холостого хода
  $L_{off}$,
  время, количество точек врезки и т.п.,
  необходимые для расчета целевых функций
  \eqref{eq:cut.time} и \eqref{eq:cut.cost}
\end{itemize}

Структура этих файлов оказалась уже заметно сложнее
структуры файла геометрической информации,
особенно с учетом того,
что последняя намеренно была разработана
максимально упрощенной.
Возникла потребность создания формальной спецификации
форматов.
Для этой цели используется механизм
JSON-схем~
\cite{bi:json-schema},
сам основанный на формате JSON
и обладающий довольно широкими
выразительными возможностями
описания семантики используемых
структур данных.

В ходе диссертационной работы были
разработаны JSON-схемы для всех
используемых файлов данных,
они приведены в Приложении~\ref{app:json}
и находятся в свободном доступе в~\cite{bi:dbs-schema}.

Эти схемы могут использоваться во многих целях, таких как:

\begin{enumerate}
  \item Документация на формат файла данных.
  \item Автоматическая проверка корректности файлов данных
  (при помощи специализированных библиотек).
  \item Автоматическая или полуавтоматическая генерация
  программного кода импорта / экспорта файлов данных
  описываемой структуры.
\end{enumerate}

\subsection{Разработка конвертеров форматов файлов данных}
\label{sec:json-dbs.js}

Для расширения поддержки разработанных форматов файлов данных
в ходе диссертационной работы
была разработан пакет утилит командной строки~
\cite{bi:dbs2json}.

Утилиты написаны на языке
JavaScript~\cite{bi:JavaScript}
и его диалекте
CoffeeScript~\cite{bi:CoffeeScript}
и упакованы в исполняемый файл при помощи
Webpack~\cite{bi:webpack}.
В настоящее время они работают только под
операционной системой Microsoft Windows,
хотя возможна сборка для большинства современных платформ.

Утилиты обеспечивают конвертацию между множеством
форматов файлов,
используемых разными подсистемами,
включая:
\begin{itemize}
  \item JSON с геометрической информацией о деталях и раскройной карте
  \item Двоичный файл геометрии DBS, экспорт и импорт
  \item Текстовый файл DXF~\cite{bi:DXF},
  используемый для обмена графической информацией между CAD-системами.
  Обеспечивается экспорт и импорт.
  \item Экспорт геометрической информации в YAML~\cite{bi:YAML}
  \item Запуск автоматического раскроя в САПР <<Сириус>>
  с выбором параметров раскроя.
  \item Запуск автоматической резки в САПР <<Сириус>>
  при помощи RoutingManager
  \cite{bi:greedy,bi:RoMa}.
  \item Экспорт задания и импорт результатов
  в систему фигурного раскроя T-Flex Раскрой~\cite{bi:T-Flex}
  \item Экспорт и импорт в форматы файлов системы автоматического раскроя
  Nesting Factory~\cite{bi:Algomate}.
  Обеспечивается также автоматический запуск раскроя
  и последующий автоматический импорт его результатов.
  \item Экспорт в виде графа для поиска
  Эйлерова цикла~\cite{bi:Makarovskikh2019Jan,Makarovskikh20181171,bi:Makarovskikh2019Other}
  в рамках решения задачи маршрутизации режущего инструмента.
  \item Экспорт в виде HTML-страницы,
  содержащей SVG
  для визуализации,
  см. раздел~\ref{sec:json.view}.
\end{itemize}

Использование открытых форматов позволяет
легко расширять этот список.
Вместе с тем,
более правильным представляется
встраивание экспорта и импорта таких форматов
непосредственно в код используемых подсистем,
однако это не всегда возможно оперативно.
