%!TEX root = ../these.tex

\section{Динамическое программирование}
\label{sec:pcgtsp.dp}

Алгоритм, описанный в разделе~\ref{sec:pcgtsp.bnb},
будучи реализован в ходе данной диссертационной работы,
оказался работоспособным и может применяться как
для оценки качества решений полученных другими способами,
так и для поиска точных решений задачи PCGTSP.

Вместе с тем, дизайн в виде классической схемы ветвей и границ
проявил некоторые недостатки:

\begin{enumerate}
  \item
  Использование для нижней оценки
  одного минимального значения
  $c_{min}$
  вместо полной матрицы
  $D(\sigma)_{ij}$~\eqref{eq.pcgtsp.c_min.dp}
  представляется слишком сильным огрублением.
  \item
  Префиксы $\sigma$ обрабатываются независимо друг от друга,
  тогда как многие из них порождают
  одинаковую вспомогательную задачу
  $\mathcal P(\sigma)$~\eqref{eq:pcgtsp.cut.key}.
  Результаты ее решения требуется запоминать в глобальной переменной,
  что увеличивает расход памяти и затрудняет
  параллельное исполнение алгоритма.
\end{enumerate}
