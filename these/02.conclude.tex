%!TEX root = ../these.tex

\section{Выводы по Главе \ref{ch:ccp}}
\label{sec:ccp.conclude}

\begin{enumerate}
  \item
  Разработана и реализована оригинальная эвристика
  поиска оптимальных положений точек врезки
  на контурах деталей.
  \item
  Эта эвристика может сочетаться с алгоритмами
  дискретной оптимизации для получения
  полного алгоритма решения задачи непрерывной резки
  CCP
  \item
  Полученные решения оказываются вполне сравнимы
  с решениями, получаемыми более традиционными методами,
  путем сведения задачи CCP к задаче GTSP за счёт
  дискретизации контуров деталей.
  Более того, длина холостого хода
  для решений, даваемых разработанным алгоритмом,
  систематически оказывается чуть лучше,
  чем для ранее использовавшихся алгоритмов.
  \item
  Описанный алгоритм может также применяться
  для решения задач более высокого класса ---
  SCCP и GSCCP (непрерывной сегментной резки),
  что открывает дорогу к исследованиям в области
  задачи прерывистой резки (ICP).
  \item
  Переспективными представляются также следующие
  направления исследований:
  \begin{itemize}
    \item
    Использование разработанной эвристики
    в сочетании с другими алгоритмами дискретной оптимизации
    \item
    Учёт других технологических ограничений термической резки,
    кроме ограничения предшествования
  \end{itemize}
\end{enumerate}
