%!TEX root = ../these.tex

\subsection{Получение нижних оценок}
\label{sec:pgstcp.bound}

Используя подход динамического программирования
(Dynamic Programming, DP),
легко найти кратчайший путь
вдоль $\sigma$
между всеми парами вершин
$(v_1, v_r)$,
где
$v_1\in V_{i_1}, v_r\in V_{i_r}$
\begin{equation}
  \label{eq:pcgtsp.Cmin}
  c_{min}(\sigma)=\min_{\substack{v_1\in V_{i_1} \\ v_r\in V_{i_r}} } \left\{
    cost \left(P_{v_1, v_r}\right) \colon
    P_{v_1, v_r} \text{ --- частичный путь в порядке } \sigma \right\}
\end{equation}

Таким образом, исходная постановка задачи PCGTSP
может быть в каждом узле дерева поиска декомпозирована в две
(более простых) задачи:
\begin{enumerate}
  \item вычисление $c_{min}(\sigma)$
  \item задачу PCGTSP $\mathcal P(\sigma)$, полученную релаксацией исходной задачи PCGTSP следующим образом:
  \begin{itemize}
    \item использует те же ограничения предшествования
    \item содержит те же узлы и кластера,
    за исключением внутренних кластеров префикса
    $V_{i_2}, \dots V_{i_{r-1}}$
    \item ребра исходящие из $V_{i_1}$
    и входящие в $V_{i_r}$ удаляются
    \item веса ребер $c(u,v)$ сохраняются,
    \item для всех пар вершин
    $v_1\in V_{i_1}, v_r\in V_{i_r}$
    добавляются ребра нулевого веса
    $c(v_1, v_r)=0$
  \end{itemize}
\end{enumerate}

\begin{proposition}
\begin{multline}
  \label{eq:pcgtsp.decomposition}
  \min_T \{cost(T)| T \text{ начинается с префикса } \sigma\}
  \geqslant \\
  c_{min}(\sigma) + \min_{T}\{cost(T)| T \text{ -- допустимое решение задачи } \mathcal P(\sigma)\}
\end{multline}
\end{proposition}

\begin{proof}
  Любой маршрут $T$,
  начинающийся с префикса $\sigma$,
  делится на две части ---
  одну вдоль $\sigma$,
  вторую от $V_{i_r}$ до ${V_{i_1}}$.
  Первая имеет вес не менее $c_{min}(\sigma)$
  по определению.
  Вторая может быть дополнена ребром,
  идущим из $V_{i_r}$ в ${V_{i_r}}$
  (имеющим нулевой вес)
  и тем самым давая решение задачи
  $\mathcal P(\sigma)$.
\end{proof}

Нижняя оценка решения исходной задачи PCGTSP,
таким образом,
в каждом узле дерева поиска
$\sigma$
оценивается по формуле \eqref{eq:pcgtsp.decomposition}.
Важно заметить,
что многие задачи $\mathcal  P(\sigma)$
оказываются идентичными для разных $\sigma$
(отличающихся порядком посещения внутренних кластеров),
что позволяет значительно уменьшить общее время вычислений.

В то же время,
$\mathcal P(\sigma)$
все еще представляет собой экземпляр
задачи PCGTSP,
хотя и меньше размером,
чем исходная постановка.
Заменим ее решение в формуле~\eqref{eq:pcgtsp.decomposition}
на нижнюю границу,
полученную тем или иным способом
$$
\min_{T}\{cost(T)| T \text{ -- допустимое решение задачи } \mathcal P(\sigma)\}
\geqslant LB(\mathcal P(\sigma))
$$

В данной диссертационной работе для вычисления оценки
$LB(\mathcal P(\sigma))$
вслед за~\cite{SALMAN2020163}
используется трехступенчатая релаксация:
\begin{enumerate}
  \item $\mathcal P(\sigma) \to$ GTSP
  \item GTSP $\to$ ATSP
  \item ATSP $\to \mathcal P_{rel}(\sigma)$
\end{enumerate}
и решение задачи $\mathcal P_{rel}(\sigma)$
принимается за нижнюю границу
$LB(\mathcal P(\sigma))$:
\begin{equation}
  \label{eq:pcgtsp.LB}
  \min_T \{cost(T)| T \text{ начинается с префикса } \sigma\}
  \geqslant
  c_{min}(\sigma) + \mathrm{OPT}(\mathcal P_{rel})
\end{equation}

На всех шагах релаксации,
кроме первого,
используются несколько методов,
так что в общем случае получается \textit{несколько}
оценок
$\mathrm{OPT}(\mathcal P_{rel})$,
в формулу~\eqref{eq:pcgtsp.LB}
подставляется \textit{наибольшая} из них.
Опишем применявшиеся в данной диссертационной работе
методы релаксации задачи
$\mathcal P(\sigma)$
подробнее.

\subsection*{Сведение PCGTSP к GTSP}

Для сведения
$\mathcal P(\sigma)$
к GTSP
необходимо снять ограничения предшествования.
Это можно сделать не полностью, а частично ---
ослабить
путем удаления ребер
$(v_p, v_q)$, которые заведомо запрещены
частичным порядком $\Pi$,
то есть
$(V(v_q), V(v_p))\in A$.
При этом маршрут из $v_q$ в $v_p$
все равно остается возможным,
но через один или несколько промежуточных вершин.

Более строгий анализ показывает,
что можно удалить еще больше ребер.

Так, рассмотрим ребро
$(v_p, v_q)$,
которое формально соответствует частичному порядку $\Pi$
($(V(v_p), V(v_q))\in A$),
но существует промежуточный кластер $V^*$:
$(V(v_p), V^+), (V^+, V(v_q))\in A$.
Тогда ребро
$(v_p, v_q)$
тоже оказывается запрещено ограничениями предшествования,
так как любой допустимый маршрут
$v_p \to v_q$
должен проходить через некоторую вершину
$v^+ \in V^+$.
Для эффективного поиска таких
(запрещенных)
ребер
в данной диссертационной работе
заранее вычисляется
\textit{транзитивное сокращение}
орграфа $\Pi$,
то есть минимальный набор ребер
$A'$,
транзитивное замыкание которого дает $A$.
Например, в библиотеке NetworkX \cite{bi:NetworkX}
для вычисления транзитивного сокращения
имеется функция \textit{transitive\_reduction()}.
Тогда подлежат удалению ребра
$$
\left\{(v_p, v_q)|
  (V(v_p), V(v_q))\in A, (V(v_p), V(v_q))\notin A'\right\}
$$

Кроме того, отдельно следует рассмотреть ребра
$(v_p, v_q)$
в случае, когда
$V(v_q)\in V_1$.
Для них условие удаления меняется на почти обратное:
остаются в графе только те ребра,
которые выходят из <<финальных>> кластеров,
то есть
$\nexists V^+ \colon (V(v_p), V^+)\in A$.

Поскольку мы ослабили ограничения предшествования,
то
$OPT(\mathcal P) \geqslant OPT{GTSP}$.

\subsection{Сведение GTSP к ATSP}
Частично выполнив таким образом ограничения предшествования,
далее сводим получившуюся задачу GTSP
к ATSP. Сделать это можно несколькими способами.

\begin{enumerate}
  \item Преобразование Нуна и Бина~\cite{NoonBean1993},
    сводящее GTSP к ATSP того же размера $n$
  \begin{itemize}
    \item
    Кластеры $V_i$
    (за исключением вырожденного случая $|V_i|=1$)
    замыкаются в циклы нулевого веса,
    то есть для каждой вершины $v_i$ добавляется ребро
    $c(v_i^-,v_i)=0$,
    где $v_i^- \in V(v_i)$ --- вершина, предшествующая $v_i$ в цикле,
    порядок замыкания вершин кластера в цикл --- произвольный
    \item
    Ребро $(v_i, v_j)$
    заменяется на ребро $(v_i^-,v_j)$
    того же веса.
  \end{itemize}
  \item
  \textit{Граф кластеров}
  $H_1 = (\mathcal C, A_1)$,
  индуцированный исходным графом $G$
  по правилу
  $$
    c(V_i, V_j)=\min_{\substack{v_i\in V_i \\ v_j \in V_j}}
    c(v_i,v_j)
  $$
  Тем самым получается задача ATSP размера $m$,
  очевидно
  $LB(\mathcal P) \geqslant LB(H_1)$.
  \item
  Граф кластеров
  $H_2 = (\mathcal C, A_2)$,
  также индуцированный графом $G$,
  но используются веса путей длины 2,
  удовлетворяющих ограничениям предшествования
  $$
    c(V_i, V_j)=\min_{\substack{v_i\in V_i \\ v_j \in V_j \\ v_k\notin V_i, V_j}}
    \frac{c(v_i,v_k)+c(v_k,v_j)}{2}
  $$
  Можно доказать,
  что нижняя оценка на решение задачи ATSP
  для графа $H_2$
  является также нижней оценкой для
  построенной выше задачи GTSP,
  а значит и для исходной задачи
  $\mathcal P$.

  Причина использования графа $H_2$
  заключается в том, что
  граф $H_1$
  фактически не требует,
  чтобы маршрут входил и покидал кластер $V_i$
  через одну и ту же вершину $v_i \in V_i$,
  а использование путей длины 2
  автоматически учитывает это требование,
  хотя бы частично.

  При построении $H_2$
  ограничения предшествования учитываются сложнее,
  чем для преобразования Нуна и Бина или графа $H_1$:
  оба ребра
  $(v_i, v_k)$ и $(v_k, v_j)$
  должны быть допустимы с точки зрения ослабленных
  ограничений предшествования,
  как описано выше.
  \item
  Графы кластеров $H_3, H_4$ и т.д. ---
  строятся аналогично $H_2$,
  но при помощи путей большей длины,
  что позволяет получить альтернативные нижние оценки.
  Ввиду того,
  что построение таких графов заметно сложнее алгоритмически,
  в данной диссертационной работе они не использовались.
\end{enumerate}

\subsection*{Вторая релаксация ATSP }
Теперь задача,
подлежащая решению ещё упростилась,
теперь это просто TSP
(возможно, асимметричная)
и возможно меньшего размера $m<n$,
чем исходная.
Однако она все еще полиномиально сложная,
поэтому еще раз упростим ее,
причем это тоже можно сделать несколькими способами.

