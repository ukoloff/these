%!TEX root = ../these.tex

\section{Выводы по Главе \ref{ch:pcgtsp}}
\label{sec:pcgtsp.conclude}

\begin{enumerate}
  \item
  В ходе диссертационной работы разработан
  и реализован первый специализированный алгоритм ветвей и границ
  для обобщенной задачи коммивояжера с ограничениями предшествования
  PCGTSP,
  развивающий идеи классической схемы динамического программирования
  Хелда и Карпа и подход Салмана к построению нижних оценок.
  \item
  Алгоритм разработан в двух версиях ---
  классической схемы ветвей и границ и динамического программирования ---
  немного отличающихся производительностью.
  Версия динамического программирования естественным образом
  допускает параллельное выполнение.
  \item
  Для оценки производительности алгоритмов проведены численные эксперименты,
  в которых для сравнения использовался передовой коммерческий солвер Gurobi.
  Эксперименты продемонстрировали конкурентоспособность предложенных алгоритмов.
  \item
  В качестве направления дальнейших исследований предполагается:
  \begin{itemize}
    \item
    разработка более точных нижних оценок.
    \item
    дальнейшая оптимизация и распараллеливание
    \item
    интеграция с системой автоматизированного проектирования <<Сириус>>~\cite{Obuhovo},
    предназначенной для оптимизации раскроя листового материала
    на фигурные заготовки и подготовки управляющих программ для машин листовой резки с ЧПУ.
  \end{itemize}
\end{enumerate}
