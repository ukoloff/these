% !TeX root = ..
\section{Численные эксперименты}\label{sec:experiments}

В данном разделе приводятся результаты численных экспериментов
по оценке производительности предлагаемого
алгоритма ветвей и границ
в сравнении со схемой динамического программирования
а также решателем Gurobi,
использующим нашей недавней MILP-моделью
\cite{KKP-optima2020}.

\begin{table}[p]
  \centering
  \caption{Результаты экспериментов}
  \label{t:data}
  \scriptsize
  \def\arraystretch{1.5}
  \begin{tabular}{|r|c*{12}{|r}|}
  \hline
  \multicolumn{5}{|c|}{\textit{Задача}} &
    \multicolumn{3}{c|}{\textit{Gurobi}} &
    \multicolumn{3}{c|}{\textit{Ветвей и границ}} &
    \multicolumn{3}{c|}{\textit{DP}} \\ \hline
    \multicolumn{1}{|c|}{\textit{№}} &
    \multicolumn{1}{c|}{\textit{ID}} &
    \multicolumn{1}{c|}{\textit{n}} &
    \multicolumn{1}{c|}{\textit{m}} &
    \multicolumn{1}{c|}{\textit{UB$_0$}} &
    \multicolumn{1}{c|}{\textit{Время}} &
    \multicolumn{1}{c|}{\textit{LB}} &
    \multicolumn{1}{c|}{\textit{gap, \%}} &
    \multicolumn{1}{c|}{\textit{Время}} &
    \multicolumn{1}{c|}{\textit{LB}} &
    \multicolumn{1}{c|}{\textit{gap, \%}} &
    \multicolumn{1}{c|}{\textit{Время}} &
    \multicolumn{1}{c|}{\textit{LB}} &
    \multicolumn{1}{c|}{\textit{gap, \%}} \\ \hline
    {\bf 1}  & br17.12   & 92   & 17  & 43    & 82.00 & 43    & 0.00  & {\bf 11.2} & \textbf{43}    & {\bf 0.00}    & 27.3   & 43    & 0.00    \\ \hline
    2  & ESC07     & 39   & 8   & 1730  & 0.24   & 1730  & 0.00  & 1.3   & 1726  & 0.23    & 8.37   & 1730  & 0.00    \\ \hline
    3  & ESC12     & 65   & 13  & 1390  & 3.35   & 1390  & 0.00  & 4.3   & 1385  & 0.36    & 14.99  & 1390  & 0.00    \\ \hline
    4  & ESC25     & 133  & 26  & 1418  & 10.61  & 1383  & 0.00  & 32 & 1383    & 0.00    & 60.69  & 1383  & 0.00    \\ \hline
    5  & ESC47     & 244  & 48  & 1399  & 3773   & 1064  & 4.93 & 36000 & 980   & 42.76   & 36000  & 981   & 42.61   \\ \hline
    {\bf 6} & ESC63     & 349  & 64  & 62    & 25.35 & 62    & 0.00  & 1.3   & 62    & 0.00    & {\bf 0.52}   & \textbf{62}    & {\bf 0.00}    \\ \hline
    {\bf 7}  & ESC78     & 414  & 79  & 14872 & 1278.45 & 14630 & 1.66  & 1.3   & 14594 & 1.63    & {\bf 0.68}   & \textbf{14594} & {\bf 1.63}    \\ \hline
    8  & ft53.1    & 281  & 53  & 6194  & 36000  & 5479  & 13.04  & 36000 & 4839  & 28.27   & 36000  & 4839  & 28.27   \\ \hline
    9  & ft53.2    & 274  & 53  & 6653  & 36000  & 5511  & 20.7  & 36000 & 4934  & 34.84   & 36000  & 4940  & 34.68   \\ \hline
    10 & ft53.3    & 281  & 53  & 8446  & 36000  & 6354  & 32.92 & 36000 & 5465  & 54.55   & 36000  & 5465  & 54.55   \\ \hline
    {\bf 11} & ft53.4    & 275  & 53  & 11822 & 20635  & 11259 & 5.00  & 35865 & 11274 & 4.86    & \textbf{2225}   & \textbf{11290} & \textbf{4.71}    \\ \hline
    12 & ft70.1    & 346  & 70  & 32848 &  83.70 & 31521 & 4.21  & 36000 & 31153 & 5.44    & 36000  & 31177 & 5.36    \\ \hline
    13 & ft70.2    & 351  & 70  & 33486 & 36000  & 31787 & 5.35  & 36000 & 31268 & 7.09    & 36000  & 31273 & 7.08    \\ \hline
    14 & ft70.3    & 347  & 70  & 35309 & 36000  & 32775 & 7.73  & 36000 & 32180 & 9.72    & 36000  & 32180 & 9.72    \\ \hline
    {\bf 15} & ft70.4    & 353  & 70  & 44497 & 36000  & 41160 & 8.11  & 36000 & 38989 & 14.13   & \textbf{36000}  & \textbf{41640} & {\bf 6.86}    \\ \hline
    16 & kro124p.1 & 514  & 100 & 33320 & 36000  & 29541 & 12.79 & 36000 & 27869 & 19.56   & 36000  & 27943 & 19.24   \\ \hline
    17 & kro124p.2 & 524  & 100 & 35321 & 36000  & 29983 & 17.80 & 36000 & 28155 & 25.45   & 36000  & 28155 & 25.45   \\ \hline
    18 & kro124p.3 & 534  & 100 & 41340 & 36000 & 30669 & 34.79  & 36000 & 28406 & 45.53   & 36000  & 28406 & 45.53   \\ \hline
    19 & kro124p.4 & 526  & 100 & 62818 & 36000  & 46033 & 36.46 & 36000 & 38137 & 64.72   & 36000  & 38511 & 63.12   \\ \hline
    20 & p43.1     & 203  & 43  & 22545 & 4691   & 21677 & 4.00  & 36000 & 738   & 2954.88 & 36000  & 788   & 2761.04 \\ \hline
    21 & p43.2     & 198  & 43  & 22841 & 36000  & 21357 & 6.94  & 36000 & 749   & 2949.53 & 36000  & 877   & 2504.45 \\ \hline
    22 & p43.3     & 211  & 43  & 23122 & 36000  & 15884 & 45.57 & 36000 & 898   & 2474.83 & 36000  & 906   & 2452.10 \\ \hline
    {\bf 23} & p43.4     & 204  & 43  & 66857 & 36000  & 45198 & 47.92 & 4470  & 66846 & 0.00    & {\bf 333.02} & \textbf{66846} & {\bf 0.00}    \\ \hline
    24 & prob.100  & 510  & 99  & 1474  & 36000  & 805   & 83.10 & 36000 & 632   & 133.23  & 36000  & 632   & 133.23  \\ \hline
    25 & prob.42   & 208  & 41  & 232   & 13310 & 196   & 4.86 & 36000 & 149   & 55.70   & 36000  & 153   & 51.63   \\ \hline
    {\bf 26} & rbg048a   & 255  & 49  & 282   & 24.22  & 282   & 0.00  & 0.9   & 272   & 3.68    & {\bf 0.25}   & \textbf{272}   & \textbf{3.68}    \\ \hline
    {\bf 27} & rbg050c   & 259  & 51  & 378   &  13.83  & 378   & 0.00  & {\bf 0.2}   & \textbf{372}   & \textbf{1.61}    & 0.25   & 372   & 1.61    \\ \hline
    28 & rbg109a   & 573  & 110 & 848   & 6  & 848   & 0.00  & 2407  & 812   & 4.43    & 682    & 809   & 4.82    \\ \hline
    {\bf 29} & rbg150a   & 871  & 151 & 1415  & 15  & 1382  & 2.38  & {\bf 0.4}   & \textbf{1353}  & \textbf{4.58}    & 0.53   & 1353  & 4.58    \\ \hline
    {\bf 30} & rbg174a   & 962  & 175 & 1644  & 27 & 1605  & 2.43  & {\bf 0.4}   & \textbf{1568}  & \textbf{4.85}    & 0.67   & 1568  & 4.85    \\ \hline
    {\bf 31} & rbg253a   & 1389 & 254 & 2376  & 61  & 2307  & 2.99  & {\bf 0.8} & \textbf{2269} & \textbf{4.72} & 1.42   & 2269  & 4.72    \\ \hline
    {\bf 32} & rbg323a   & 1825 & 324 & 2547  & 416  & 2490  & 2.29 & {\bf 2.0}  & \textbf{2448}  & \textbf{4.04} & 3.59   & 2448  & 4.04    \\ \hline
    33 & rbg341a   & 1822 & 342 & 2101  & 18470  & 2033  & 4.97  & 36000 & 1840  & 14.18   & 36000  & 1840  & 14.18   \\ \hline
    34 & rbg358a   & 1967 & 359 & 2080  & 17807  & 1982  & 4.95  & 36000 & 1933  & 7.60    & 36000  & 1933  & 7.60    \\ \hline
    35 & rbg378a   & 1973 & 379 & 2307  & 32205  & 2199  & 4.91  & 36000 & 2032  & 13.53   & 36000  & 2031  & 13.59   \\ \hline
    36 & ry48p.1   & 256  & 48  & 13135 & 36000  & 11965 & 9.78  & 36000 & 10739 & 22.31   & 36000  & 10764 & 22.03   \\ \hline
    37 & ry48p.2   & 250  & 48  & 13802 & 36000  & 12065 & 14.39 & 36000 & 10912 & 26.48   & 36000  & 11000 & 25.47   \\ \hline
    38 & ry48p.3   & 254  & 48  & 16540 & 36000  & 13085 & 26.40 & 36000 & 11732 & 40.98   & 36000  & 11822 & 39.91   \\ \hline
    {\bf 39} & ry48p.4   & 249  & 48  & 25977 & 36000  & 22084 & 17.62 & 18677 & 25037 & 3.75    & {\bf 14001} & \textbf{25043} & {\bf  3.73}    \\ \hline
  \end{tabular}
\end{table}

\subsection{Условия эксперимента}
Все алгоритмы тестировались на общедоступной библиотеке
PCGTSPLIB
\cite{SALMAN2020163}.
Во всех случаях для теплого старта,
всем алгоритмам предоставляется одно и то же
допустимое решение,
полученное эвристическим решателем
PCGLNS
\cite{PCGLNS}.
Для алгоритмов ветвей и границ и динамического программирования,
все вычисления проводятся на одном и том же оборудовании
(16-ядерный Intel Xeon, 128G RAM)
с предельным временем счета 10~часов.
В качестве критерия остановки
мы используем понижение разрыва ниже 5\%,
где разрыв определяется по формуле
\begin{equation}
  \label{eq:gap-stop}
  gap = \frac{\UB - \LB}{\LB}.
\end{equation}

В качестве базы сравнения,
мы воспроизвели численные эксперименты,
представленные в
\cite{KKP-optima2020}
в условиях, описанных выше,
включая время счёта 10 часов и критерий остановки
\eqref{eq:gap-stop}.

Исходный код предложенных алгоритмов
и вспомогательные скрипты доступны в
\cite{GitHub}.

\subsection{Обсуждение}

Полученные результаты эксперимента
представлены в табл.~\ref{t:data},
которая организована следующим образом:
первая группа столбцов описывает
задачу,
включая её обозначение ID,
количество вершин $n$
и кластеров $m$,
а также стоимость стартового решения $\UB_0$,
полученного эвристикой PCGLNS.
Затем следуют три группы столбцов
для решателя Gurobi
и двух предлагаемых алгоритмов.
Каждая группа содержит время
счета в секундах,
наилучшее значение нижней границы
$\LB$
и наилучший разрыв gap
в процентах.
Задачи,
в которых один из предлагаемых
алгоритмов сработал лучше Gurobi,
выделены жирным шрифтом.

Как следует из табл.~\ref{t:data},
для 13 из 39 задач (33\%)
один из наших алгоритмов
показал лучшую производительность.
Из них в 12 случаях лучше время счёта,
а в 7 -- точность.

Заметим, что предложенные алгоритмы
смогли найти оптимальное решение в 6 из 39 случаях
(хотя это не было целью эксперимента).
Для 10 (15) задач, включая одни из самых больших
{\it rbg323a} и {\it rbg358a}
(1825 и 1967 вершин соответственно)
было получено решение с точностью 5\% (10\%).


С другой стороны,
для некоторых задач
(например, {\it p43.1, p43.2} и {\it p43.3}),
результаты наших алгоритмов
оказались крайне слабы по сравнению с Gurobi,
что по видимому объясняется
очень грубыми оценками нижней границы.
В то же время для задач
{\it p43.4} и {\it ry48p.4}
наши алгоритмы сработали гораздо лучше Gurobi.

В целом, хотя Gurobi демонстрирует в среднем
чуть лучшую производительность,
предложенные алгоритмы за редким исключением,
показывают вполне сопоставимые результаты.
Считаем нужным добавить,
что в наших экспериментах
решателю Gurobi было предоставлено,
так же как и тестируемым алгоритмам,
хорошее стартовое решение,
что является не очень обычным способом
организации эксперимента.

