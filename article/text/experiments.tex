% !TeX root = ..
\section{Численные эксперименты}\label{sec:experiments}

В данном разделе приводятся результаты численных экспериментов
по оценке производительности предлагаемого
алгоритма ветвей и границ
в сравнении со схемой динамического программирования
а также решателем Gurobi,
использующим нашей недавней MILP-моделью
\cite{KKP-optima2020}.

\begin{table}[p]
  \centering
  \caption{Результаты экспериментов}
  \label{t:data}
  \scriptsize
  \def\arraystretch{1.5}
  \hspace*{-5ex}
  \begin{tabular}{|r|c*{12}{|r}|}
  \hline
  \multicolumn{5}{|c|}{\textit{Задача}} &
    \multicolumn{3}{c|}{\textit{Gurobi}} &
    \multicolumn{3}{c|}{\textit{Ветвей и границ}} &
    \multicolumn{3}{c|}{\textit{DP}} \\ \hline
    \multicolumn{1}{|c|}{\textit{№}} &
    \multicolumn{1}{c|}{\textit{ID}} &
    \multicolumn{1}{c|}{\textit{n}} &
    \multicolumn{1}{c|}{\textit{m}} &
    \multicolumn{1}{c|}{\textit{UB}} &
    \multicolumn{1}{c|}{\textit{Время}} &
    \multicolumn{1}{c|}{\textit{LB}} &
    \multicolumn{1}{c|}{\textit{gap, \%}} &
    \multicolumn{1}{c|}{\textit{Время}} &
    \multicolumn{1}{c|}{\textit{LB}} &
    \multicolumn{1}{c|}{\textit{gap, \%}} &
    \multicolumn{1}{c|}{\textit{Время}} &
    \multicolumn{1}{c|}{\textit{LB}} &
    \multicolumn{1}{c|}{\textit{gap, \%}} \\ \hline
  {\bf 1}  & br17.12   & 92   & 17  & 43    & 107.28 & 43    & 0.00  & {\bf 11.2} & 43    & {\bf 0.00}    & 27.3   & 43    & 0.00    \\ \hline
  2  & ESC07     & 39   & 8   & 1730  & 0.07   & 1730  & 0.00  & 1.3   & 1726  & 0.23    & 8.37   & 1730  & 0.00    \\ \hline
  3  & ESC12     & 65   & 13  & 1390  & 0.52   & 1390  & 0.00  & 4.3   & 1385  & 0.36    & 14.99  & 1390  & 0.00    \\ \hline
  {\bf 4}  & ESC25     & 133  & 26  & 1418  & 4.45   & 1383  & 2.53  & {\bf 32} & 1383  & {\bf 0.00}    & 60.69  & 1383  & 0.00    \\ \hline
  5  & ESC47     & 244  & 48  & 1399  & 52.01  & 1063  & 31.61 & 36000 & 980   & 42.76   & 36000  & 981   & 42.61   \\ \hline
  {\bf 6} & ESC63     & 349  & 64  & 62    & 380.65 & 62    & 0.00  & 1.3   & 62    & 0.00    & {\bf 0.52}   & 62    & {\bf 0.00}    \\ \hline
  {\bf 7}  & ESC78     & 414  & 79  & 14832 & 43200  & 14581 & 1.72  & 1.3   & 14594 & 1.63    & {\bf 0.68}   & 14594 & {\bf 1.63}    \\ \hline
  8  & ft53.1    & 281  & 53  & 6207  & 42099  & 6022  & 2.96  & 36000 & 4839  & 28.27   & 36000  & 4839  & 28.27   \\ \hline
  9  & ft53.2    & 274  & 53  & 6653  & 42137  & 6184  & 7.58  & 36000 & 4934  & 34.84   & 36000  & 4940  & 34.68   \\ \hline
  10 & ft53.3    & 281  & 53  & 8446  & 42194  & 6936  & 21.77 & 36000 & 5465  & 54.55   & 36000  & 5465  & 54.55   \\ \hline
  11 & ft53.4    & 275  & 53  & 11822 & 23239  & 11822 & 0.00  & 35865 & 11274 & 4.86    & 2225   & 11290 & 4.71    \\ \hline
  {\bf 12} & ft70.1    & 346  & 70  & 32848 & \multicolumn{3}{c|}{Истечение времени}   & 36000 & 31153 & 5.44    & 36000  & 31177 & {\bf 5.36}    \\ \hline
  13 & ft70.2    & 351  & 70  & 33486 & 42021  & 31840 & 5.05  & 36000 & 31268 & 7.09    & 36000  & 31273 & 7.08    \\ \hline
  14 & ft70.3    & 347  & 70  & 35309 & 41173  & 32944 & 7.18  & 36000 & 32180 & 9.72    & 36000  & 32180 & 9.72    \\ \hline
  {\bf 15} & ft70.4    & 353  & 70  & 44497 & 41827  & 41378 & 7.53  & 36000 & 38989 & 14.13   & 36000  & 41640 & {\bf 6.86}    \\ \hline
  16 & kro124p.1 & 514  & 100 & 33320 & 34162  & 29926 & 11.34 & 36000 & 27869 & 19.56   & 36000  & 27943 & 19.24   \\ \hline
  17 & kro124p.2 & 524  & 100 & 35321 & 35379  & 30101 & 17.34 & 36000 & 28155 & 25.45   & 36000  & 28155 & 25.45   \\ \hline
  {\bf 18} & kro124p.3 & 534  & 100 & 41340 & \multicolumn{3}{c|}{Истечение времени}   & 36000 & 28406 & 45.53   & 36000  & 28406 & {\bf 45.53}   \\ \hline
  19 & kro124p.4 & 526  & 100 & 62818 & 41035  & 46704 & 34.50 & 36000 & 38137 & 64.72   & 36000  & 38511 & 63.12   \\ \hline
  20 & p43.1     & 203  & 43  & 22545 & 43150  & 22327 & 0.98  & 36000 & 738   & 2954.88 & 36000  & 788   & 2761.04 \\ \hline
  21 & p43.2     & 198  & 43  & 22841 & 43132  & 22381 & 2.05  & 36000 & 749   & 2949.53 & 36000  & 877   & 2504.45 \\ \hline
  22 & p43.3     & 211  & 43  & 23122 & 43058  & 22540 & 2.57  & 36000 & 898   & 2474.83 & 36000  & 906   & 2452.10 \\ \hline
  {\bf 23} & p43.4     & 204  & 43  & 66857 & 43193  & 45396 & 47.26 & 4470  & 66846 & 0.00    & {\bf 333.02} & 66846 & {\bf 0.00}    \\ \hline
  24 & prob.100  & 510  & 99  & 1474  & 38567  & 800   & 80.25 & 36000 & 632   & 133.23  & 36000  & 632   & 133.23  \\ \hline
  25 & prob.42   & 208  & 41  & 232   & 1292.1 & 202   & 14.85 & 36000 & 149   & 55.70   & 36000  & 153   & 51.63   \\ \hline
  26 & rbg048a   & 255  & 49  & 282   & 64.32  & 282   & 0.00  & 0.9   & 272   & 3.68    & 0.25   & 272   & 3.68    \\ \hline
  {\bf 27} & rbg050c   & 259  & 51  & 378   & {\bf 26.46}  & 378   & {\bf 0.00}  & {\bf 0.2}   & 372   & {\bf 1.61}    & 0.25   & 372   & 1.61    \\ \hline
  28 & rbg109a   & 573  & 110 & 848   & 83.23  & 848   & 0.00  & 2407  & 812   & 4.43    & 682    & 809   & 4.82    \\ \hline
  {\bf 29} & rbg150a   & 871  & 151 & 1415  & {\bf 29095}  & 1414  & {\bf 0.07}  & {\bf 0.4}   & 1353  & {\bf 4.58}    & 0.53   & 1353  & 4.58    \\ \hline
  30 & rbg174a   & 962  & 175 & 1644  & 5413.8 & 1641  & 0.18  & 0.4   & 1568  & 4.85    & 0.67   & 1568  & 4.85    \\ \hline
  {\bf 31} & rbg253a   & 1389 & 254 & 2376  & 43159  & 2369  & {\bf 0.13}  & {\bf 0.8} & 2269  & {\bf 4.72} & 1.42   & 2269  & 4.72    \\ \hline
  {\bf 32} & rbg323a   & 1825 & 324 & 2547  & 40499  & 2533  & {\bf 0.55}  & {\bf 2}  & 2448  & {\bf 4.04} & 3.59   & 2448  & 4.04    \\ \hline
  33 & rbg341a   & 1822 & 342 & 2101  & 30687  & 2064  & 1.41  & 36000 & 1840  & 14.18   & 36000  & 1840  & 14.18   \\ \hline
  34 & rbg358a   & 1967 & 359 & 2080  & 32215  & 2021  & 2.38  & 36000 & 1933  & 7.60    & 36000  & 1933  & 7.60    \\ \hline
  35 & rbg378a   & 1973 & 379 & 2307  & 42279  & 2231  & 2.38  & 36000 & 2032  & 13.53   & 36000  & 2031  & 13.59   \\ \hline
  36 & ry48p.1   & 256  & 48  & 13135 & 43141  & 12125 & 8.33  & 36000 & 10739 & 22.31   & 36000  & 10764 & 22.03   \\ \hline
  37 & ry48p.2   & 250  & 48  & 13802 & 43033  & 12130 & 13.78 & 36000 & 10912 & 26.48   & 36000  & 11000 & 25.47   \\ \hline
  38 & ry48p.3   & 254  & 48  & 16540 & 43102  & 13096 & 26.30 & 36000 & 11732 & 40.98   & 36000  & 11822 & 39.91   \\ \hline
  {\bf 39} & ry48p.4   & 249  & 48  & 25977 & 43057  & 22266 & 16.67 & 18677 & 25037 & 3.75    & {\bf 14001} & 25043 & {\bf  3.73}    \\ \hline
  \end{tabular}
\end{table}

\subsection{Условия эксперимента}
Все алгоритмы тестировались на общедоступной библиотеке
PCGTSPLIB
\cite{SALMAN2020163}. 
Во всех случаях для теплого старта,
всем алгоритмам предоставляется одно и то же
допустимое решение,
полученное эвристическим решателем
PCGLNS
\cite{PCGLNS}. 
Для алгоритмов ветвей и границ и динамического программирования,
все вычисления проводятся на одном и том же оборудовании
(16-ядерный Intel Xeon, 128G RAM)
с предельным временем счета 10~часов.
В качестве критерия остановки 
мы используем понижение разрыва ниже 5\%,
где разрыв определяется по формуле
$$
gap = \frac{\UB - \LB}{\LB}.
$$

В качестве базы сравнения,
мы используем результаты численных экспериментов,
представленные в 
\cite{KKP-optima2020}
для тех же самых экземпляров задач
(и на том же оборудовании),
полученные при помощи решателя Gurobi и
эвристики PCGLNS
(предельное время счета 12 часов).

\subsection{Обсуждение}

Полученные результаты эксперимента
представлены в табл.~\ref{t:data},
которая организована следующим образом:
первая группа столбцов описывает
задачу,
включая её обозначение ID,
количество вершин $n$
и кластеров $m$,
а также стоимость стартового решения $\UB$,
полученного эвристикой PCGLNS.
Затем следуют три группы столбцов
для решателя Gurobi
и двух предлагаемых алгоритмов.
Каждая группа содержит время
счета в секундах,
наилучшее значение нижней границы
$\LB$
и наилучший разрыв gap
в процентах.
Задачи,
в которых один из предлагаемых
алгоритмов сработал лучше Gurobi,
выделены жирным шрифтом.

Как следует из табл.~\ref{t:data},
для 13 из 39 задач (33\%)
один из наших алгоритмов
показал лучшую производительность
(либо в точности решения, либо во времени счета).
Особенно мы хотим выделить задачи
{\it ft70.1} и {\it kro124p.3},
где наши алгоритмы смогли найти допустимое решение,
тогда как Gurobi не смог этого сделать.

Кроме того,
для некоторых задач,
стартовое решение,
полученное эвристикой PCGLNS
оказалось настолько близким к оптимальному,
что наши алгоритмы завершают работу
почти немедленно.

С другой стороны,
для некоторых задач
(например, {\it p43.1, p43.2} и {\it p43.3}),
результаты наших алгоритмов
оказались крайне слабы по сравнению с Gurobi,
что по видимому объясняется
очень грубыми оценками нижней границы.

Считаем нужным добавить,
что в наших экспериментах
решателю Gurobi было предоставлено,
так же как и тестируемым алгоритмам,
хорошее стартовое решение,
что является не очень обычным способом
организации эксперимента.

