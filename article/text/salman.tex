% !TeX root = ..
\section{Алгоритм ветвей и границ}\label{sec:SA}

Для решения задачи PCGTSP 
$(G,\CX,\Pi)$,
мы обходим дерево поиска в ширину
(Breadth First Search),
см.~Алгоритм~\ref{alg:bnb}.
Каждый узел этого дерева связан с префиксом
$\sigma = \left(V_{i_1}, V_{i_2}, \dots V_{i_r} \right)$,
где
$V_{i_j} \in \CX$,
$V_{i_1} = V_1$,
и
$r \in \{{1, \dots m}\}$.

\begin{algorithm}
\caption{BnB :: Главная процедура}\label{alg:bnb}
\hspace*{\algorithmicindent}{\bf Вход}: орграф $G$, кластеры $\CX$, частичный порядок $\Pi$ \\
\hspace*{\algorithmicindent}{\bf Выход}: маршрут и его стоимость
\begin{algorithmic}[1]
    \STATE инициализация $Q =$ empty queue
    \STATE начинаем с $Root = V_1$
    \STATE $Q$.push($Root$)
    \WHILE{\NOT $Q$.empty()}
        \STATE берём следующий префикс для обработки: $\sigma = Q$.pop()
        \STATE $process = Bound(\sigma)$
        \IF{\NOT $process$}
            \STATE префикс отсекается; {\bf continue}
        \ENDIF
        \STATE $UpdateLowerBound(\sigma)$
        \FORALL{$child \in Branch(\sigma)$}
            \STATE помещаем префикс в очередь на обработку $Q$.push($child$)
        \ENDFOR
    \ENDWHILE
\end{algorithmic}
\end{algorithm}

\begin{algorithm}
\caption{BnB :: Bound}\label{alg:bnb:bound}
\hspace*{\algorithmicindent}{\bf Вход}: префикс $\sigma$ \\
\hspace*{\algorithmicindent}{\bf Выход}: признак того, что префикс подлежит обработке
\begin{algorithmic}[1]
    \STATE {\bf global} $D_{ij}^{\mathcal T}$
    \STATE {\bf global} $Opt^{\mathcal T}$
    \STATE вычисляем кортеж $\mathcal T = \left(V_{i_1},
        \left\{V_{i_1}, V_{i_2}, \dots V_{i_r}\right\}, V_{i_r} \right)$
        \label{alg:bnb:bound:key}
    \STATE $D_{ij} = MinCosts(\sigma)$
        \label{alg:bnb:bound:pfx}
    \IF{$D_{ij}^{(\sigma)} \ge D_{ij}^{\mathcal T}[\mathcal T], \forall i, j$}
        \RETURN \FALSE
    \ENDIF
    \STATE  обновляем веса маршрутов $D_{ij}^{\mathcal T}[\mathcal T]  = \min \left(
        D_{ij}^{\mathcal T}[\mathcal T], D_{ij} \right),
        \forall i, j$
    \STATE  $c_{min} = \min\limits_{i, j} D_{ij}$
    \IF{$\mathcal T \notin Opt^{\mathcal T}$}
        \STATE вычисляем нижнюю границу $Opt^{\mathcal T}[\mathcal T] = \max\left(L_1(\sigma), L_2(\sigma))\right)$
        \label{alg:bnb:bound:sfx}
    \ENDIF
    \STATE $\LB = c_{min} + Opt^{\mathcal T}[\mathcal T]$
        \label{alg:bnb:bound:lb}
    \IF{$\LB > \UB$}
        \RETURN \FALSE
    \ENDIF
    \RETURN \TRUE
\end{algorithmic}
\end{algorithm}

К каждому узлу дерева поиска
мы применяем процедуру отсечения
Bound
(Алгоритм~\ref{alg:bnb:bound}),
которая выполняет следующие действия:
\begin{itemize}
    \item
    для префикса $\sigma$,
    мы находим кортеж
    % $\mathcal T(\sigma)$
$$
\mathcal T(\sigma) = \left(V_{i_1},
        \left\{V_{i_1}, V_{i_2}, \dots V_{i_r}\right\}, V_{i_r} \right)
$$
    \item
    на шаге~\ref{alg:bnb:bound:pfx},
    мы вычисляем матрицу $D(\sigma)$ 
    минимальных попарных весов по формуле:
    $$
    \hspace{-2ex}
    D(\sigma)_{vu} = \min\left\{
      cost(P_{v,u})\colon
      v \in V_{i_1},
      u \in V_{i_r},
      P_{v,u} \text{ путь $v$-$u$ в порядке } \sigma
    \right\}.
    $$
    Эта матрица удобно вычисляется инкрементально
    на основе матрицы
    $D(\sigma')$
    родительского узла дерева поиска
    \item
    если, для некоторого $\sigma_1$, $\mathcal T(\sigma) = \mathcal T(\sigma_1)$ 
    и
    $$
    D(\sigma)_{vu} \ge D(\sigma_1)_{vu}, \quad
    (v \in V_{i_1}, u \in V_{i_r}),
    $$
    то,
    префикс $\sigma$ имеет веса в матрице$D(\sigma)$
    больше, чем для префикса $\sigma_1$
    и подлежит отсечению
    \item
    на шаге~\ref{alg:bnb:bound:sfx},
    мы рассчитываем оценки $L_1$ и $L_2$,
    см. табл.~\ref{t:LBs} 
    и сохраняем их в глобальной переменной
    $Opt^{\mathcal T}$,
    используя формулу
    $$
    Opt^{\mathcal T(\sigma)} = \max(L_1, L_2)
    $$
    \item
    для текущего узла 
    $\sigma$,
    рассчитываем нижнюю границу по формуле
    $$
    \LB(\sigma) = \min_{vu}D(\sigma)_{vu} + Opt^{\mathcal T(\sigma)}
    $$
    на шаге~\ref{alg:bnb:bound:lb}
    \item
    наконец, 
    узел $\sigma$ отсекается, если $\LB > \UB$.
\end{itemize}

\begin{algorithm}
\caption{BnB :: Branch}\label{alg:bnb:branch}
\hspace*{\algorithmicindent}{\bf Вход}: префикс $\sigma$ \\
\hspace*{\algorithmicindent}{\bf Выход}: список потомков префикса для обработки
\begin{algorithmic}[1]
    \STATE инициализация $R =$ empty queue
    \FORALL{$V \in \CX$}
        \STATE $valid =$ \TRUE
        \FORALL{$W \in \sigma$}
            \IF{$W=V$ \OR $(V, W) \in \Pi$}
                \STATE $valid =$ \FALSE
                \STATE {\bf break}
            \ENDIF
        \ENDFOR
        \IF{$valid$}
            \STATE добавляем новый префикс $R$.push($\sigma+V$)
        \ENDIF
    \ENDFOR
    \RETURN $R$
\end{algorithmic}
\end{algorithm}

Префиксы,
которые избежали отсечения,
обрабатываются процедурой 
$Branch$
(Алгоритм~\ref{alg:bnb:branch}),
которая пытается удлинить префикс 
$\sigma$
на один кластер,
соблюдая при этом ограничение предшествования
$\Pi$.
