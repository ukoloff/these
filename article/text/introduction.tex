% !TeX root = ..
\section{Введение}\label{sec:intro}
Обобщенная задача коммивояжера (GTSP) -- 
это хорошо известная задача комбинаторной оптимизации, 
представленная в основополагающей статье \cite{SKGS1969} 
Сриваставы и др. и привлекшая внимание многих исследователей
(см. обзор в~\cite{GutinPunnen2007}).

В GTSP для данного взвешенного орграфа 
$ G = (V, E, c) $ 
и разбиения $ V_1 \cup \ldots \cup V_m $ набора узлов $ V $ 
на непустые взаимно непересекающиеся кластеры требуется найти замкнутый тур с минимальной стоимостью $ T $, 
который посещает каждый кластер $ V_i $ ровно один раз.

В этой статье мы рассматриваем обобщенную задачу коммивояжера 
с ограничением предшествования (PCGTSP), 
в которой кластеры следует посещать в соответствии с некоторым заданным частичным порядком. 
Эта расширенная версия GTSP имеет множество практический применений, включая
\begin{itemize}
	\item
	оптимизация траектории инструмента для станков с числовым программным управлением (ЧПУ) 
	\cite{CASTELINO2003173}
	\item
	минимизация времени {\it холостого хода} при раскрое листового металла 
	\cite{Petunin2018, Makarovskikh20181171}
	\item
	координатно-измерительное оборудование 
	\cite{SALMAN2016138} 
	\item
	оптимизация траектории при многоствольном бурении
	\cite{DEWIL2019}.
\end{itemize}

\subsection{Связанные работы} 
GTSP - это расширение классической задачи коммивояжера (TSP). 
Следовательно, если оценивать размер задачи количеством кластеров 
$ m $, 
задача становится NP-сложной даже на евклидовой плоскости \cite{Papa77}. 
С другой стороны, 
хорошо известная схема динамического программирования Хелда и Карпа~\cite{HeldKarp1962}, 
адаптированная к GTSP, 
имеет временную сложность 
$ O (n ^ 3m ^ 2 \cdot 2 ^ m) $, 
то есть этот алгоритм принадлежит FPT,
будучи параметризован количеством кластеров. 
Следовательно, 
оптимальное решение
GTSP может быть найдено за полиномиальное время при условии 
$ m = O (\log n) $.

Обзор литературы показывает, 
что алгоритмическое проектирование GTSP развивалось по нескольким направлениям.

Первый подход основан на сведении исходной задачи к некоторой задаче асимметричной TSP, 
после чего этот вспомогательный экземпляр может быть решен с помощью алгоритмов, 
разработанного для ATSP 
(\cite{LaporteSemet1999, NoonBean1993}). 
Несмотря на математическую элегантность, 
этот подход страдает несколькими недостатками:
\begin{enumerate}
\item
полученные экземпляры ATSP устроены довольно необычно, 
что затрудняет их решение даже для современных решателей MIP, 
таких как Gurobi и CPLEX.
\item
близкие к оптимальным решения задачи ATSP
могут соответствовать недопустимым решениям исходной задачи 
\cite{KaraGut2012}.
\end{enumerate}

Другой подход связан с разработкой точных алгоритмов для частных случаев 
и алгоритмов аппроксимации с теоретическими гарантиями производительности. 
Среди них есть алгоритмы ветвей и границ и ветвей и разрезов 
(см., например, \cite{FishGonToth1997, Yuan2020}) 
и приближенные схемы полиномиального времени (PTAS) 
для некоторых специальных случаев 
\cite{FerGriSit2006, KhN-PSIM2017}.

Наконец, 
третий подход заключается в разработке 
различных эвристик и метаэвристик. 
Так, Г.~Гутин и Д.~Карапетян \cite{Gutin-2010} 
предложили эффективный меметический алгоритм, 
в \cite{Helsgaun-2015} знаменитый эвристический решатель 
Лина-Кернигана-Хельсгауна был расширен до GTSP, 
а в \cite{SMITH20171} была разработана мощная метаэвристика 
Adaptive Large Neighborhood Search (ALNS), 
которая на сегодняшний день является наиболее эффективной.

К сожалению, в случае PCGTSP 
алгоритмические результаты все еще остаются довольно малочисленными. 
Насколько нам известно, в открытых источниках доступны только
\begin{enumerate}
	\item
	эффективные алгоритмы для специальных ограничений предшествования типа Баласа 
	\cite {Balas-Sim2001, ChenKhKh2016, CKK-IFAC2016} 
	и ограничения предшествования, приводящие к квази- и псевдопирамидальным оптимальным обходам	
	\cite{KhN-AMAI-2020}
	\item
	общие идеи о специализированном (PCGTSP) алгоритме ветвей и границ
	\cite{SALMAN2020163}
	\item
	недавняя эвристика PCGLNS, предложенная авторами 
	\cite{KKP-optima2020} 
	как расширение результатов
	\cite{SMITH20171}. 
\end{enumerate} 

В этой статье мы пытаемся восполнить этот пробел.

\subsection{Новизна данной работы}
\begin{itemize}
	\item
	расширяя идею, предложенную в {SALMAN20163}, 
	мы разрабатываем и реализуем первый специализированный алгоритм для PCGTSP
	\item
	расширяя классический подход к ветвлению \cite{MorinMarsten1976}, 
	мы реализуем схему динамического программирования Хелда и Карпа, 
	дополненную оригинальной ограничивающей стратегией
	\item
	для оценки предложенных и реализованных алгоритмов 
	мы проводим численные эксперименты на общедоступной библиотеке PCGTSPLIB~\cite{SALMAN2020163}. 
	Мы сравниваем производительность этих алгоритмов с современным MIP-решателем Gurobi, 
	оснащенным лучшей моделью MIPS. 
	Чтобы уравнять всех конкурентов, 
	мы во всех случаях используем в качестве теплого старта приближенное решение, 
	полученное нашей недавней эвристикой PCGLNS~\cite{PCGLNS}.
\end{itemize}  
