% !TeX root = ..
\section{Заключение}\label{sec:summary}

В данной работе
разработан и реализован
первый специализированный алгоритм
ветвей и границ
для обобщенной задачи коммивояжера
с ограничениями предшествования.
Он развивает идеи классической схемы
динамического программирования
Хелда-Карпа
и схемы Салмана.

Для оценки производительности
предложенных алгоритмов,
проведены численные эксперименты,
в качестве базы сравнения использован
решатель Gurobi.
Эксперименты продемонстрировали,
что наши алгоритмы
вполне конкурентноспособны на уровне
современных MIP-решателей.

В качестве направления дальнейших исследований
мы предполагаем разработку более
точных нижних оценок.
Кроме того,
мы полагаем,
что дальнейшая оптимизация
и распараллеливание
могут существенно улучшить производительность
наших алгоритмов.

В настоящее время разработанное программное обеспечение
интегрируется с системой автоматизированного проектирования СИРИУС
\cite{bi:Sirius},
предназначенной для оптимизация раскроя
листового материала на фигурные заготовки
и подготовки управляющих программ
для машин листовой резки с ЧПУ.

\subsection*{Благодарность}

Работа выполнена
в ходе исследований
Уральского Математического Центра
при финансовой поддержке Министерства науки и высшего образования РФ,
соглашение № 075-02-2021-1383.
