\documentclass{article}

% \documentclass[runningheads]{llncs}

\usepackage{cmap}

\usepackage[utf8]{inputenc}
\usepackage[T2A]{fontenc}
\usepackage[english,russian]{babel}

\usepackage{indentfirst}
\setlength{\parindent}{2.5em}

% Убираем переносы в заголовках
\usepackage[raggedright]{titlesec}

% Короткое тире для ненумерованых списков
% ГОСТ 2.105-95, пункт 4.1.7, требует дефиса, но так лучше смотрится
\renewcommand{\labelitemi}{\normalfont\bfseries{--}}

%%% Выравнивание и переносы %%%
%% http://tex.stackexchange.com/questions/241343/what-is-the-meaning-of-fussy-sloppy-emergencystretch-tolerance-hbadness
%% http://www.latex-community.org/forum/viewtopic.php?p=70342#p70342
\tolerance 1414
\hbadness 1414
\emergencystretch 1.5em % В случае проблем регулировать в первую очередь
\hfuzz 0.3pt
\vfuzz \hfuzz
%\raggedbottom
%\sloppy                 % Избавляемся от переполнений

% Висячие строки
\clubpenalty=10000
\widowpenalty=10000
\brokenpenalty=4991
\raggedbottom

% Надписи рисунков и таблиц
\usepackage{caption}

\captionsetup{
  font=small,
  labelsep=period,
  justification=centering,
}

% https://tex.stackexchange.com/a/78777
% Подписи к таблицам
\DeclareCaptionFormat{hfillstart}{\hfill#1#2#3}
\captionsetup[table]
{
  format=hfillstart,
  labelsep=newline,
}

\usepackage{listings}

\usepackage{url}

\usepackage{tabularx}

% \usepackage{flafter}

\usepackage{enumitem}

\usepackage{lipsum}

% Should be the LAST package to use
\usepackage{hyperref}

\newcommand{\theseAuthor}
  {Уколов Станислав Сергеевич}

\newcommand{\theseYear}
  {2021}

\newcommand{\theseTitle}
  {Разработка алгоритмов оптимальной маршрутизации
  режущего инструмента
  для~машин термической резки с~ЧПУ}

\newcommand{\theseSupervisor}
  {Петунин Александр Александрович}

\newcommand{\theseSvRegalia}
  {доктор технических наук, доцент}

\newcommand{\theseCode}
  {05.13.12}

\newcommand{\theseCodeDesq}
  {Системы автоматизации проектирования (промышленность)}

\newcommand{\theseCity}
  {Екатеринбург}


\hypersetup{
  linktocpage=true,
  pdfborder={0 0 0},
  pdftitle={\theseTitle},
  pdfauthor={\theseAuthor},
  pdfsubject={\theseCode\ \theseCodeDesq},
  pdflang={ru},
}

% Вывести информацию о версиях используемых библиотек в лог сборки
\listfiles


% !TeX root = ..

\usepackage{algorithm}
\usepackage{algorithmic}
\usepackage{graphicx}
\usepackage[caption=false]{subfig}

\floatname{algorithm}{Algorithm}
\newcommand{\ITP}[1]{#1^{\mathrm{ITP}}}
\newcommand{\DP}[1]{#1^{\mathrm{DP}}}

\newcommand{\APP}[1]{#1_{\mathrm{APP}}}
\newcommand{\CITP}[1]{#1_{\mathrm{CITP}}} 
\newcommand{\OPT}[1]{\mathrm{OPT}({#1})}
% \newcommand{\APP}[1]{\mathrm{APP}({#1})}
\newcommand{\OP}{\mathrm{OPT}}
\newcommand{\AP}{\mathrm{APP}} 
\renewcommand{\le}{\leq}
\renewcommand{\ge}{\geq}  

\newcommand{\R}{\mathfrak{R}}
\newcommand{\I}{\mathfrak{I}}
\newcommand{\CX}{\mathcal{C}}
\renewcommand{\L}{\mathcal{L}}
\renewcommand{\P}{\mathcal{P}}
\newcommand{\LB}{\mathrm{LB}}
\newcommand{\UB}{\mathrm{UB}}

\renewcommand{\algorithmicforall}{\textbf{for each}}



\author{М.Ю. Хачай, С.С. Уколов, А.А. Петунин}
\title{
Специализированный алгоритм ветвей и границ
для обобщённой задачи коммивояжера
с ограничениями предшествования
}
% \titlerunning{BnB algorithms for PCGTSP}


% \author{Michael Khachay\inst{1}\orcidID{0000-0003-3555-0080} \and\\ Stanislav Ukolov\inst{2}\orcidID{0000-0002-9946-6446} \and\\ Alexander Petunin\inst{1,2}\orcidID{0000-0003-2540-1305}}

% \institute{Krasovsky Institute of Mathematics and Mechanics, Ekaterinburg, Russia \and Ural Federal University, Ekaterinburg, Russia\\
%   \email{mkhachay@imm.uran.ru},\ \  \email{s.s.ukolov@urfu.ru}, \  \ \email{aapetunin@gmail.com}}

	

\begin{document}

\maketitle

\begin{abstract}
The Generalized Traveling Salesman Problem (GTSP) is a well-known combinatorial optimization problem having numerous valuable practical applications in operations research. In the Precedence Constrained GTSP (PCGTSP), any feasible tour is restricted to visit all the clusters according to some given partial order. Unlike the common setting of the GTSP, the PCGTSP appears still weakly studied in terms of algorithmic design and implementation.
To the best of our knowledge, all the known algorithmic results for this problem can be exhausted by Salmans's general branching framework, a few MILP models, and the PCGLNS meta-heuristic proposed by the authors recently. In this paper, we present the first problem-specific branch-and-bound algorithm designed with an extension of Salman's approach and exploiting PCGLNS as a powerful primal heuristic.   Using the public PCGTSPLIB testbench, we evaluate the performance of the proposed algorithm against the classic Held-Karp dynamic programming scheme with branch-and-bound node fathoming strategy and Gurobi state-of-the-art solver armed by our recently proposed MILP model and PCGLNS-based warm start. 
% \keywords{Generalized Traveling Salesman Problem \and precedence constraints \and Branch-and-Bound algorithm}
\end{abstract}  

% !TeX root = ..
\section{Введение}\label{sec:intro}
Обобщенная задача коммивояжера (GTSP) --
это хорошо известная задача комбинаторной оптимизации,
представленная в основополагающей статье \cite{SKGS1969}
Сриваставы и др. и привлекшая внимание многих исследователей
(см. обзор в~\cite{GutinPunnen2007}).

В GTSP для данного взвешенного орграфа
$ G = (V, E, c) $
и разбиения $ V_1 \cup \ldots \cup V_m $ набора узлов $ V $
на непустые взаимно непересекающиеся кластеры требуется найти замкнутый тур с минимальной стоимостью $ T $,
который посещает каждый кластер $ V_i $ ровно один раз.

В этой статье мы рассматриваем обобщенную задачу коммивояжера
с ограничением предшествования (PCGTSP),
в которой кластеры следует посещать в соответствии с некоторым заданным частичным порядком.
Эта расширенная версия GTSP имеет множество практический применений, включая
\begin{itemize}
	\item
	оптимизация траектории инструмента для станков с числовым программным управлением (ЧПУ)
	\cite{CASTELINO2003173}
	\item
	минимизация времени {\it холостого хода} при раскрое листового металла
	\cite{Petunin2018, Makarovskikh20181171}
	\item
	координатно-измерительное оборудование
	\cite{SALMAN2016138}
	\item
	оптимизация траектории при многоствольном бурении
	\cite{DEWIL2019}.
\end{itemize}

\subsection{Связанные работы}
GTSP - это расширение классической задачи коммивояжера (TSP).
Следовательно, если оценивать размер задачи количеством кластеров
$ m $,
задача становится NP-сложной даже на евклидовой плоскости \cite{Papa77}.
С другой стороны,
хорошо известная схема динамического программирования Хелда и Карпа~\cite{HeldKarp1962},
адаптированная к GTSP,
имеет временную сложность
$ O (n ^ 3m ^ 2 \cdot 2 ^ m) $,
то есть этот алгоритм принадлежит FPT,
будучи параметризован количеством кластеров.
Следовательно,
оптимальное решение
GTSP может быть найдено за полиномиальное время при условии
$ m = O (\log n) $.

Обзор литературы показывает,
что алгоритмическое проектирование GTSP развивалось по нескольким направлениям.

Первый подход основан на сведении исходной задачи к некоторой задаче асимметричной TSP,
после чего этот вспомогательный экземпляр может быть решен с помощью алгоритмов,
разработанного для ATSP
(\cite{LaporteSemet1999, NoonBean1993}).
Несмотря на математическую элегантность,
этот подход страдает несколькими недостатками:
\begin{enumerate}
\item
полученные экземпляры ATSP устроены довольно необычно,
что затрудняет их решение даже для современных решателей MIP,
таких как Gurobi и CPLEX.
\item
близкие к оптимальным решения задачи ATSP
могут соответствовать недопустимым решениям исходной задачи
\cite{KaraGut2012}.
\end{enumerate}

Другой подход связан с разработкой точных алгоритмов для частных случаев
и алгоритмов аппроксимации с теоретическими гарантиями производительности.
Среди них есть алгоритмы ветвей и границ и ветвей и разрезов
(см., например, \cite{FishGonToth1997, Yuan2020})
и приближенные схемы полиномиального времени (PTAS)
для некоторых специальных случаев
\cite{FerGriSit2006, KhN-PSIM2017}.

Наконец,
третий подход заключается в разработке
различных эвристик и метаэвристик.
Так, Г.~Гутин и Д.~Карапетян \cite{Gutin-2010}
предложили эффективный меметический алгоритм,
в \cite{Helsgaun-2015} знаменитый эвристический решатель
Лина-Кернигана-Хельсгауна был расширен до GTSP,
а в \cite{SMITH20171} была разработана мощная метаэвристика
Adaptive Large Neighborhood Search (ALNS),
которая на сегодняшний день является наиболее эффективной.

К сожалению, в случае PCGTSP
алгоритмические результаты все еще остаются довольно малочисленными.
Насколько нам известно, в открытых источниках доступны только
\begin{enumerate}
	\item
	эффективные алгоритмы для специальных ограничений предшествования типа Баласа
	\cite {Balas-Sim2001, ChenKhKh2016, CKK-IFAC2016}
	и ограничения предшествования, приводящие к квази- и псевдопирамидальным оптимальным обходам
	\cite{KhN-OPTA2018,KhN-AMAI-2020}
	\item
	общие идеи о специализированном (PCGTSP) алгоритме ветвей и границ
	\cite{SALMAN2020163}
	\item
	недавно разработанный авторами данной статьи метаэвристический солвер PCGLNS
  \cite{KKP-optima2020, PCGLNS},
  развивающий результаты, полученные в
	\cite{SMITH20171}
  для GTSP.
\end{enumerate}

В этой статье мы пытаемся восполнить этот пробел.

\subsection{Новизна данной работы}
\begin{itemize}
	\item
	расширяя идею, предложенную в \cite{SALMAN2020163},
	мы разрабатываем и реализуем первый специализированный алгоритм для PCGTSP
	\item
	расширяя классический подход к ветвлению \cite{MorinMarsten1976},
	мы реализуем схему динамического программирования Хелда и Карпа,
	дополненную оригинальной ограничивающей стратегией
	\item
	проведенные численные эксперименты показывают,
  что производительность предложенных алгоритмов
  сравнима как в смысле скорости, так и точности получаемых решений
  с современным решателем Gurobi,
  использующим лучшую в настоящее время MILP-модель
  и стартовое решение MIP

\end{itemize}

% \input{text/related-work}
% !TeX root = ..
\section{Problem Statement}\label{sec:PS}
We consider the general setting of the Precedence Constrained Generalized Traveling Problem (PCGTSP). An instance of this problem is given by a triplet $(G,\CX,\Pi)$, where
\begin{itemize}
	\item[-] an edge-weighted digraph $G=(V,E,c)$ defines a groundset network supplemented with transportation costs $c(u,v)$ for any arc $(u,v)\in E$
	\item[-] a partition $\CX=\{V_1,\ldots,V_m\}$ splits the nodeset $V$ of the graph $G$ into $m$ non-empty pairwise-disjoint \textit{clusters}
	\item[-] a directed acyclic graph $\Pi=(\CX,A)$ defines a partial order (\textit{precedence constraints}) on the set of clusters $\CX$.      
\end{itemize}

For any node $v\in V$, by $V(v)$ we denote the (only) cluster $V_p\in\CX$, such that $v\in V_p$. Further, without loss of generality, we assume $\Pi$ to be \textit{transitively closed} (i.e.  $(V_i,V_j)\in A$ and $(V_j,V_k)\in A$ imply $(V_i,V_k)\in A$) and that $(V_1,V_p)\in A$ for any $p\in\{2,\ldots,m\}$.

A closed $m$-tour $T$ is called a \textit{feasible} solution of the PCGTSP, if it
\begin{itemize}
	\item[-] departs for and arrives at some node $v_1\in V_1$
	\item[-] visits each cluster $V_p\in\CX$
	\item[-] each arc $(v_i, v_j)$ of  $T$ (except the arc $(v_m,v_1)$) fulfills the precedence constraints, i.e. $(V(v_i),V(v_j))\in A$.
\end{itemize} 

To any tour $T=v_1, v_2, \ldots, v_m$, we assign its cost
$$
	cost(T) = c(v_m,v_1) + \sum_{i=1}^{m-1} c(v_i,v_{i+1}). 
$$ 
The goal is to find a feasible tour $T$ of the minimum cost $cost(T)$. 
% !TeX root = ..
\section{Preliminaries}\label{sec:pre}
Both algorithms, the branch-and-bound and dynamic programming designed and implemented in this paper exploit the similar main idea. 

\subsection{Instance decomposition}
At any node of the search tree, we decompose the initial instance as follows: 
\begin{itemize}
	\item[(i)] consider a subset $\CX'\subset\CX$, such that $V_1\in\CX'$, fix some cluster $V_l\in\CX'$ and nodes $v\in V_1$ and $u\in V_l$, respectively
	\item[(ii)] let $c_{\min}$ be a lower bound for the minimum cost of $v-u$-paths traversing all the clusters in $\CX'$ and fulfilling the precedence constraints\footnote{In our dynamic programming, this bound is tight} 
	\item[(iii)] excluding from $\CX'$ all the inner clusters and connecting $V_1$ with $V_l$ directly by a zero-cost arc(s), we consider a smaller auxiliary subproblem $\P$, which inherits other transportation costs, clustering, and precedence constraints from the initial instance 
	\item[(iv)] taking 
	\begin{equation}\label{e:bounds}
		\LB = c_{\min} + \OPT{\P_{rel}}
	\end{equation}
	as a lower bound, we fathom the current node each time when $\LB > \UB$. Here, $\OPT{\P_{rel}}$ is the optimum of some efficiently solvable relaxation of $\P$ and $\UB$ is the cost of the best known feasible solution.
\end{itemize}
\subsection{Lower bounds}\label{ssec:LBs}
In this subsection, we compare the lower bounds obtained by several relaxations of the auxiliary problem $\P$. To relax $\P$, we use the two-stage approach proposed in \cite{SALMAN2020163}. 

At the first stage, we reduce $\P$ to the appropriate ATSP instance by one of the following ways:
\begin{itemize}
	\item[(i)] relax the initial precedence constraints by exclusion all the arcs $(v',v'')\in E$, for which $(V(v''),V(v'))\in A$. Then, reduce the obtained instance to ATSP using the classic Noon and Been transformation \cite{NoonBean1993}
	\item[(ii)] after the same relaxation of the precedence constraints, reduce the relaxed problem to the ATSP instance defined by the auxiliary \textit{cluster} graph $H_1=(\tilde\CX',A_1, c_1)$, where
	\[
		\tilde\CX'=\CX\setminus\CX'\cup\{V_1,V_l\},
	\]
	\[\hspace*{-3ex}
	A_1=\{(V_1,V_l)\}\cup\{(V_i,V_j)\ |\  i>2, \{V_i,V_j\}\subset\tilde\CX', \exists (v'\in V_i, v''\in V_j)\colon (v',v'')\in E\},
	\]
	\[
	c_1(V_1,V_l) = 0,\ c_1(V_i,V_j) = \min\{c(v',v'')\colon v'\in V_i, v''\in V_j, (v',v'')\in E\}
	\]
	\item[(iii)] reduce the initial problem to the instance of ATSP defined by the digraph $H_2=(\tilde\CX',A_2, c_2)$, for which
	\begin{multline*}
	A_2=\{(V_1,V_l)\}\cup\{(V_i, V_k)\ |\ i>2,\\ \exists (j>1): \{V_i,V_j,V_k\}\subset\tilde\CX' \wedge (\{(V_j,V_i), (V_k,V_j), (V_k, V_i)\}\cap A=\varnothing)\\
	 \wedge \exists (v'\in V_i, v''\in V_j, v'''\in V_k)\colon (\{(v',v''),(v'',v''')\}\subset E)\}\\
	\cup \{(V_i, V_k)\ |\ i>2, (\{V_i,V_k\}\subset\tilde\CX')\wedge ((V_k,V_i)\not\in A)\\
	\wedge \exists (v'\in V_i, v_1\in V_1, v''\in V_k)\colon \{(v',v_1), (v_1,v'')\}\subset E\},	
	\end{multline*}
	i.e., for any $V_i\in \tilde\CX'\setminus\{V_1\}$, the ordered pair $(V_i,V_k)\in A_2$, if there exists $V_j\in\tilde\CX'$ and nodes $v'\in V_i, v''\in V_j$ and $v'''\in V_k$, such that the path $\pi = v',v'', v'''$ is consistent with the initial preference constraints. 

	Then,  
	\[
	\hspace*{-3ex} c_2(V_1,V_l) = 0,\ c_2(V_i,V_k) = \min\{c(v',v'')+c(v'',v''')\colon \pi=v',v'', v''' \text{ is consistent}\}.
	\]
\end{itemize}

At the second stage, relaxing the obtained ATSP instance by reduction either to the Minimum Spanning Arborescence Problem (MSAP) or to the Assignment Problem (AP), we compute the appropriate lower bounds by equation \eqref{e:bounds}. In addition,  to increase the tightness of our lower bounds, we compute optimum values for some ATSP instances obtained by option (ii), using the solver Gurobi. For convenience, we present the designations for all the used lower bounds in Table \ref{t:LBs}\subref{t:matrix}. 

\begin{table}[h!]

\centering
	\scriptsize
    \caption{Lower bounds: (a) bound names; (b) for each bound we present 95\%-confidence interval for its averaged ratio to  $L_3$}\label{t:LBs}
    \subfloat[]{\label{t:matrix}
    \begin{tabular}[t]{|c|c|c|c|}
    \hline
    & \textbf{Noon-Bean} & $\mathbf{H_1}$ & $\mathbf{H_2}$ \\
    \hline \hline
  \textbf{AP} & $E_1$ & $\mathbf{L_1}$ & $\mathbf{L_2}$\\
  \textbf{MSAP} & $E_2$ & $E_3$ & $E_4$\\
  \textbf{Gurobi} & $E_5$ & $\mathbf{L_3}$ & $E_6$\\ 
  \hline 
    \end{tabular}
    }
   \quad
   \subfloat[]{\label{t:confid}
   \begin{tabular}[t]{|*{3}{c|}}
   \hline
   $E_1$ & $E_2 = E_3$ &$E_4$ \\ \hline
   $0.48 \pm 0.03$ & $0.54 \pm 0.01$ & $0.60 \pm 0.002$\\ \hline \hline
   $L_1$ & $L_2$ & $L_3$ \\ \hline
   $0.91\pm 0.02$ & $0.97\pm 0.02$ & 1.00 \\
   \hline
   \end{tabular}
   }
\end{table}

Relying on results of the exploratory experiments, we shorten the list of lower bounds employed in the subsequent evaluation 
(see Table \ref{t:LBs}\subref{t:confid}). Indeed, the bounds  $L_1$-$L_3$ appear to be tighter than others, which is statistically significant with a 95\% confidence level. Also, we skip bounds $E_5$ and $E_6$, whose computation leads to extremely high time consumption. Thus, in Section \ref{sec:experiments}, we restrict ourselves to the bounds 
\[
	\LB_i = c_{\min} + L_i,\ i\in\{1,2,3\}.
\]    


%!TEX root = KUP-optima2021.tex
\section{Branch-and-bound algorithm}\label{sec:SA}

To solve the PCGTSP instance
$(G,\C,\Pi)$,
we traverse the search tree in
Breadth First Search order
(see Algorithm~\ref{alg:bnb}).
Each node of this tree is associated with
a prefix
$\sigma = \left(V_{i_1}, V_{i_2}, \dots V_{i_r} \right)$,
where
$V_{i_j} \in \C$,
$V_{i_1} = V_1$,
and
$r \in \{{1, \dots m}\}$.

\begin{algorithm}[h!]
\caption{BnB :: Main}\label{alg:bnb}
\hspace*{\algorithmicindent}{\bf Input}: the graph $G$, clusters $\C$, the DAG $\Pi$ \\
\hspace*{\algorithmicindent}{\bf Output}: the tour and cost of optimal solution
\begin{algorithmic}[1]
    \STATE initialize $Q =$ empty queue
    \STATE start from $Root = V_1$
    \STATE $Q$.push($Root$)
    \WHILE{not $Q$.empty()}
        \STATE get prefix to process: $\sigma = Q$.pop()
        \STATE $process = Bound(\sigma)$
        \IF{\NOT $process$}
            \STATE prefix is fathomed; {\bf continue}
        \ENDIF
        \STATE $UpdateLowerBound(\sigma)$
        \FORALL{$child \in Branch(\sigma)$}
            \STATE queue child prefix $Q$.push($child$)
        \ENDFOR
    \ENDWHILE
\end{algorithmic}
\end{algorithm}

\begin{algorithm}[h!]
\caption{BnB :: Bounding procedure}\label{alg:bnb:bound}
\hspace*{\algorithmicindent}{\bf Input}: the prefix $\sigma$ \\
\hspace*{\algorithmicindent}{\bf Output}: the flag if the prefix survives or is fathomed
\begin{algorithmic}[1]
    \STATE {\bf global} $D_{ij}^{\mathcal T}$
    \STATE {\bf global} $Opt^{\mathcal T}$
    \STATE calculate tuple $\mathcal T = \left(V_{i_1},
        \left\{V_{i_1}, V_{i_2}, \dots V_{i_r}\right\}, V_{i_r} \right)$
        \label{alg:bnb:bound:key}
    \STATE $D_{ij} = MinCosts(\sigma)$
        \label{alg:bnb:bound:pfx}
    \IF{$D_{ij}^{(\sigma)} \ge D_{ij}^{\mathcal T}[\mathcal T], \forall i, j$}
        \RETURN \FALSE
    \ENDIF
    \STATE  update best weights $D_{ij}^{\mathcal T}[\mathcal T]  = \min \left(
        D_{ij}^{\mathcal T}[\mathcal T], D_{ij} \right),
        \forall i, j$
    \STATE  $c_{min} = \min\limits_{i, j} D_{ij}$
    \IF{$\mathcal T \notin Opt^{\mathcal T}$}
        \STATE calculate bounds $Opt^{\mathcal T}[\mathcal T] = \max\left(L_1(\sigma), L_2(\sigma))\right)$
        \label{alg:bnb:bound:sfx}
    \ENDIF
    \STATE $\LB = c_{min} + Opt^{\mathcal T}[\mathcal T]$
        \label{alg:bnb:bound:lb}
    \IF{$\LB > \UB$}
        \RETURN \FALSE
    \ENDIF
    \RETURN \TRUE
\end{algorithmic}
\end{algorithm}

For each node of the search tree
we apply
the Bounding procedure
(Algorithm~\ref{alg:bnb:bound})
to perform the following actions:
\begin{itemize}
    \item
    for the prefix $\sigma$,
    we assign the tuple
    % $\mathcal T(\sigma)$
$$
\mathcal T(\sigma) = \left(V_{i_1},
        \left\{V_{i_1}, V_{i_2}, \dots V_{i_r}\right\}, V_{i_r} \right)
$$
    \item
    at step~\ref{alg:bnb:bound:pfx},
    we compute the matrix $D(\sigma)$ of minimal pairwise costs by the following formula:
    $$
    \hspace{-2ex}
    D(\sigma)_{vu} = \min\left\{
      cost(P_{v,u})\colon
      v \in V_{i_1},
      u \in V_{i_r},
      P_{v,u} \text{ is a partial $v$-$u$ path along } \sigma
    \right\}.
    $$
    This can be easily calculated
    incrementally using
    matrix $D(\sigma')$
    of parent tree node
    \item
    if, for some $\sigma_1$, $\mathcal T(\sigma) = \mathcal T(\sigma_1)$ and
    $$
    D(\sigma)_{vu} \ge D(\sigma_1)_{vu}, \quad
    (v \in V_{i_1}, u \in V_{i_r}),
    $$
    then,
    prefix $\sigma$ is dominated by $\sigma_1$
    and is fathomed
    \item
    at step~\ref{alg:bnb:bound:sfx},
    we calculate bounds $L_1$ and $L_2$,
    see Table~\ref{t:LBs} and assign the global variable
    $Opt^{\mathcal T}$ by the formula
    $$
    Opt^{\mathcal T(\sigma)} = \max(L_1, L_2)
    $$
    \item
    for current node $\sigma$,
    its lower bound is
    calculated by the formula
    $$
    \LB(\sigma) = \min_{vu}D(\sigma)_{vu} + Opt^{\mathcal T(\sigma)}
    $$
    at step~\ref{alg:bnb:bound:lb}
    \item
    finally, the node $\sigma$ is fathomed if $\LB > \UB$.
\end{itemize}

\begin{algorithm}
\caption{BnB :: Branching procedure}\label{alg:bnb:branch}
\hspace*{\algorithmicindent}{\bf Input}: the prefix $\sigma$ \\
\hspace*{\algorithmicindent}{\bf Output}: the list of children prefixes to process
\begin{algorithmic}[1]
    \STATE initialize $R =$ empty queue
    \FORALL{$V \in \C$}
        \STATE $valid =$ \TRUE
        \FORALL{$W \in \sigma$}
            \IF{$W=V$ \OR $(V, W) \in \Pi$}
                \STATE $valid =$ \FALSE
                \STATE {\bf break}
            \ENDIF
        \ENDFOR
        \IF{$valid$}
            \STATE append new prefix $R$.push($\sigma+V$)
        \ENDIF
    \ENDFOR
    \RETURN $R$
\end{algorithmic}
\end{algorithm}

Nodes that survived are subjected to the
$Branch$ procedure
(Algorithm~\ref{alg:bnb:branch}),
where we try to enlarge the current prefix
$\sigma$
taking into account the precedence constraint $\Pi$.

%!TEX root = pcgtsp.tex
\section{Dynamic Programming}\label{sec:DP}
The branch-and-bound algorithm proposed in Section \ref{sec:SA} appears to be closely related to the classic Dynamic Programming (DP) scheme of Held and Karp  \cite{HeldKarp1962} adapted to take into account precedence constraints and augmented with one of the bounding strategies introduced in the seminal paper \cite{MorinMarsten1976}.

\begin{algorithm}[t]
\caption{DP ::  inductive construction of the lookup table}\label{alg:A2}
\hspace*{\algorithmicindent}\textbf{Input:} the graph $G$, the DAG $\Pi$, the layer $\L_k$ of the lookup table, and the current best upper bound $\UB$\\
% \hspace*{\algorithmicindent} \textbf{Parameters:} \textit{number\_of\_trials}, \textit{acceptance\_criterion}, \textit{termination\_criterion}\\
\hspace*{\algorithmicindent}\textbf{Output:} the $(k+1)$-th layer $\L_{k+1}$
\begin{algorithmic}[1]
\STATE initialize $\L_{k+1}=\varnothing$
\FORALL{$\C'\in\I_k$}
  \FORALL{cluster $V_l\in\C\setminus\C'$, s.t. $\C'\cup \{V_l\}\in\I_{k+1}$}
    \FORALL{$v\in V_1$ and $u\in V_l$}
      \IF{there exists a state $S=(\C',U,v,w)\in\L_k$, s.t. $(w,u)\in E$}
      \STATE define new state $S'=(\C'\cup\{V_l\}, V_l, v, u)$
      \STATE $S'[cost] = \min\{S[cost] + c(w,u)\colon S=(\C',U,v,w)\in\L_k\}$
      \STATE $S'[pred] = \arg\min\{S[cost] + c(w,u)\colon S=(\C',U,v,w)\in\L_k\}$
      \STATE $S'[LB] = S'[cost] + \max\{L_1,L_2,L_3\}$
      \IF{$S'[\LB] \leq \UB$}
        \STATE append $S'$ to $\L_{k+1}$
      \ENDIF
    \ENDIF
    \ENDFOR
  \ENDFOR
\ENDFOR
\RETURN $\L_{k+1}$
\end{algorithmic}
\end{algorithm}

Therefore, in this paper, we implement the revised version of this scheme to examine numerically the performance of our B-n-B algorithm.
Like to the classic DP, our algorithm consists of two main stages.
\begin{itemize}
  \item[(i).] At this stage, the lookup table is constructed incrementally, in the forward direction, layer by layer. The optimum of the instance to be solved is computed after the construction of the last $m$-th layer.
  \item[(ii).] Here the optimal tour is reconstructed on the lookup table, in the backward direction.
\end{itemize}

Each DP state  (entry of the lookup table) corresponds to a partial $v$-$u$-path and is indexed by a tuple $(\C',V_l,v, u)$, where
\begin{itemize}
  \item[(i)] $\C'\subset \C$ is an \textit{ideal} of the partially ordered set of clusters $\C$, i.e.
  \[
    \forall (V\in\C', V'\in\C)\   (V',V)\in A) \Rightarrow (V'\in\C');
  \]
  obviously, in our setting, $V_1$ belongs to an arbitrary ideal $\C'\subset\C$

  \item[(ii)] $V_l\subset\C'$, for which there is no $V\in \C'$, such that $(V_l,V)\in A$
  \item[(iii)] $v\in V_1$, $u\in V_l$.
\end{itemize}
Content of each DP entry $S$ consists of the reference $S[pred]$ to the predecessing state, the local lower bound $S[LB]$, and the cost $S[cost]$ of the corresponding partial $v$-$u$-path.

Let $\I_k$ be a subset of ideals of the same size $k\in\{1,\ldots,m\}$. Evidently, $\I_1=\{\{V_1\}\}$, therefore, the 1st layer $\L_1$ of the lookup table can be constructed trivially. Inductive construction of other layers is defined in Algorithm \ref{alg:A2}.

\subsection{Remarks}
(i). The optimum of the given instance can be found by the classic Bellman's equation
\[
  \mathrm{OPT}=\min_{v\in V_1}\min\{S[cost]+c(u,v)\colon S=(\C',V_l, v, u)\in\L_m\}
\]

\noindent
(ii). By construction, the size of the lookup table is $O(n^2m\cdot |\I|)$. Therefore, the running time of our algorithm is $O(n^3m^2\cdot |\I|)$. In particular, in the case of a partial order of any fixed \textit{width} $w$, $|\I|=O(m^w)$ \cite{Steiner-1990}. Therefore, the PCGTSP can be solved to optimality in a polynomial time, even without state fathoming at Steps 10-12.

\noindent
(iii) After construction of any current layer $\L_k$, we recalculate the global lower bound value, which leads to a decrease in the overall gap.

\noindent
(iv) In our implementation, to speed up the algorithm, we compute the bound $L_3$ at Step 9 only for a small number of states, with the smallest lower bounds.


%!TEX root = KUP-optima2021.tex
\section{Numerical evaluation}\label{sec:experiments}

In this section, we report the results of numerical performance evaluation of the proposed branch-and-bound algorithm in comparison with the DP scheme and the Gurobi solver supplemented with our recent MILP model \cite{KKP-optima2020}.

\subsection{Experimental setup}
All the algorithms are tested against the public PCGTSPLIB testbench library \cite{SALMAN2020163}. To perform a warm start on each testing instance, all algorithms are supplied by the same feasible solution obtained by the PCGLNS heuristic solver \cite{PCGLNS}. For the B-n-B and DP algorithms, all computations are carried out on the same hardware (16-core Intel Xeon 128G RAM) within the same time limit of 10 hours. As a stop criteria, we use 5\% gap tolerance, where  
$$
gap = \frac{\UB - \LB}{\LB}.
$$

As a baseline, we use the results of computational experiments presented in \cite{KKP-optima2020} for the same instances (and the same hardware resources) by Gurobi and PCGLNS MIP-start solutions (time limit is 12 hours). 

\begin{table}[p]
    \centering
    \caption{Experimental results}
    \label{t:data}
    \scriptsize
    \def\arraystretch{1.5}
    \hspace*{-5ex}
    \begin{tabular}{|r|c*{12}{|r}|}
    \hline
    \multicolumn{5}{|c|}{\textit{Instance}} &
      \multicolumn{3}{c|}{\textit{Gurobi}} &
      \multicolumn{3}{c|}{\textit{Branch   \& Bound}} &
      \multicolumn{3}{c|}{\textit{DP}} \\ \hline
    \multicolumn{1}{|c|}{\textit{\#}} &
      \multicolumn{1}{c|}{\textit{ID}} &
      \multicolumn{1}{c|}{\textit{n}} &
      \multicolumn{1}{c|}{\textit{m}} &
      \multicolumn{1}{c|}{\textit{UB}} &
      \multicolumn{1}{c|}{\textit{time (sec)}} &
      \multicolumn{1}{c|}{\textit{LB}} &
      \multicolumn{1}{c|}{\textit{gap (\%)}} &
      \multicolumn{1}{c|}{\textit{time (sec)}} &
      \multicolumn{1}{c|}{\textit{LB}} &
      \multicolumn{1}{c|}{\textit{gap (\%)}} &
      \multicolumn{1}{c|}{\textit{time (sec)}} &
      \multicolumn{1}{c|}{\textit{LB}} &
      \multicolumn{1}{c|}{\textit{gap (\%)}} \\ \hline
    {\bf 1}  & br17.12   & 92   & 17  & 43    & 107.28 & 43    & 0.00  & {\bf 11.2} & 43    & {\bf 0.00}    & 27.3   & 43    & 0.00    \\ \hline
    2  & ESC07     & 39   & 8   & 1730  & 0.07   & 1730  & 0.00  & 1.3   & 1726  & 0.23    & 8.37   & 1730  & 0.00    \\ \hline
    3  & ESC12     & 65   & 13  & 1390  & 0.52   & 1390  & 0.00  & 4.3   & 1385  & 0.36    & 14.99  & 1390  & 0.00    \\ \hline
    {\bf 4}  & ESC25     & 133  & 26  & 1418  & 4.45   & 1383  & 2.53  & {\bf 32} & 1383  & {\bf 0.00}    & 60.69  & 1383  & 0.00    \\ \hline
    5  & ESC47     & 244  & 48  & 1399  & 52.01  & 1063  & 31.61 & 36000 & 980   & 42.76   & 36000  & 981   & 42.61   \\ \hline
    {\bf 6} & ESC63     & 349  & 64  & 62    & 380.65 & 62    & 0.00  & 1.3   & 62    & 0.00    & {\bf 0.52}   & 62    & {\bf 0.00}    \\ \hline
    {\bf 7}  & ESC78     & 414  & 79  & 14832 & 43200  & 14581 & 1.72  & 1.3   & 14594 & 1.63    & {\bf 0.68}   & 14594 & {\bf 1.63}    \\ \hline
    8  & ft53.1    & 281  & 53  & 6207  & 42099  & 6022  & 2.96  & 36000 & 4839  & 28.27   & 36000  & 4839  & 28.27   \\ \hline
    9  & ft53.2    & 274  & 53  & 6653  & 42137  & 6184  & 7.58  & 36000 & 4934  & 34.84   & 36000  & 4940  & 34.68   \\ \hline
    10 & ft53.3    & 281  & 53  & 8446  & 42194  & 6936  & 21.77 & 36000 & 5465  & 54.55   & 36000  & 5465  & 54.55   \\ \hline
    11 & ft53.4    & 275  & 53  & 11822 & 23239  & 11822 & 0.00  & 35865 & 11274 & 4.86    & 2225   & 11290 & 4.71    \\ \hline
    {\bf 12} & ft70.1    & 346  & 70  & 32848 & \multicolumn{3}{c|}{Time limit}   & 36000 & 31153 & 5.44    & 36000  & 31177 & {\bf 5.36}    \\ \hline
    13 & ft70.2    & 351  & 70  & 33486 & 42021  & 31840 & 5.05  & 36000 & 31268 & 7.09    & 36000  & 31273 & 7.08    \\ \hline
    14 & ft70.3    & 347  & 70  & 35309 & 41173  & 32944 & 7.18  & 36000 & 32180 & 9.72    & 36000  & 32180 & 9.72    \\ \hline
    {\bf 15} & ft70.4    & 353  & 70  & 44497 & 41827  & 41378 & 7.53  & 36000 & 38989 & 14.13   & 36000  & 41640 & {\bf 6.86}    \\ \hline
    16 & kro124p.1 & 514  & 100 & 33320 & 34162  & 29926 & 11.34 & 36000 & 27869 & 19.56   & 36000  & 27943 & 19.24   \\ \hline
    17 & kro124p.2 & 524  & 100 & 35321 & 35379  & 30101 & 17.34 & 36000 & 28155 & 25.45   & 36000  & 28155 & 25.45   \\ \hline
    {\bf 18} & kro124p.3 & 534  & 100 & 41340 & \multicolumn{3}{c|}{Time limit}   & 36000 & 28406 & 45.53   & 36000  & 28406 & {\bf 45.53}   \\ \hline
    19 & kro124p.4 & 526  & 100 & 62818 & 41035  & 46704 & 34.50 & 36000 & 38137 & 64.72   & 36000  & 38511 & 63.12   \\ \hline
    20 & p43.1     & 203  & 43  & 22545 & 43150  & 22327 & 0.98  & 36000 & 738   & 2954.88 & 36000  & 788   & 2761.04 \\ \hline
    21 & p43.2     & 198  & 43  & 22841 & 43132  & 22381 & 2.05  & 36000 & 749   & 2949.53 & 36000  & 877   & 2504.45 \\ \hline
    22 & p43.3     & 211  & 43  & 23122 & 43058  & 22540 & 2.57  & 36000 & 898   & 2474.83 & 36000  & 906   & 2452.10 \\ \hline
    {\bf 23} & p43.4     & 204  & 43  & 66857 & 43193  & 45396 & 47.26 & 4470  & 66846 & 0.00    & {\bf 333.02} & 66846 & {\bf 0.00}    \\ \hline
    24 & prob.100  & 510  & 99  & 1474  & 38567  & 800   & 80.25 & 36000 & 632   & 133.23  & 36000  & 632   & 133.23  \\ \hline
    25 & prob.42   & 208  & 41  & 232   & 1292.1 & 202   & 14.85 & 36000 & 149   & 55.70   & 36000  & 153   & 51.63   \\ \hline
    26 & rbg048a   & 255  & 49  & 282   & 64.32  & 282   & 0.00  & 0.9   & 272   & 3.68    & 0.25   & 272   & 3.68    \\ \hline
    {\bf 27} & rbg050c   & 259  & 51  & 378   & {\bf 26.46}  & 378   & {\bf 0.00}  & {\bf 0.2}   & 372   & {\bf 1.61}    & 0.25   & 372   & 1.61    \\ \hline
    28 & rbg109a   & 573  & 110 & 848   & 83.23  & 848   & 0.00  & 2407  & 812   & 4.43    & 682    & 809   & 4.82    \\ \hline
    {\bf 29} & rbg150a   & 871  & 151 & 1415  & {\bf 29095}  & 1414  & {\bf 0.07}  & {\bf 0.4}   & 1353  & {\bf 4.58}    & 0.53   & 1353  & 4.58    \\ \hline
    30 & rbg174a   & 962  & 175 & 1644  & 5413.8 & 1641  & 0.18  & 0.4   & 1568  & 4.85    & 0.67   & 1568  & 4.85    \\ \hline
    {\bf 31} & rbg253a   & 1389 & 254 & 2376  & 43159  & 2369  & {\bf 0.13}  & {\bf 0.8} & 2269  & {\bf 4.72} & 1.42   & 2269  & 4.72    \\ \hline
    {\bf 32} & rbg323a   & 1825 & 324 & 2547  & 40499  & 2533  & {\bf 0.55}  & {\bf 2}  & 2448  & {\bf 4.04} & 3.59   & 2448  & 4.04    \\ \hline
    33 & rbg341a   & 1822 & 342 & 2101  & 30687  & 2064  & 1.41  & 36000 & 1840  & 14.18   & 36000  & 1840  & 14.18   \\ \hline
    34 & rbg358a   & 1967 & 359 & 2080  & 32215  & 2021  & 2.38  & 36000 & 1933  & 7.60    & 36000  & 1933  & 7.60    \\ \hline
    35 & rbg378a   & 1973 & 379 & 2307  & 42279  & 2231  & 2.38  & 36000 & 2032  & 13.53   & 36000  & 2031  & 13.59   \\ \hline
    36 & ry48p.1   & 256  & 48  & 13135 & 43141  & 12125 & 8.33  & 36000 & 10739 & 22.31   & 36000  & 10764 & 22.03   \\ \hline
    37 & ry48p.2   & 250  & 48  & 13802 & 43033  & 12130 & 13.78 & 36000 & 10912 & 26.48   & 36000  & 11000 & 25.47   \\ \hline
    38 & ry48p.3   & 254  & 48  & 16540 & 43102  & 13096 & 26.30 & 36000 & 11732 & 40.98   & 36000  & 11822 & 39.91   \\ \hline
    {\bf 39} & ry48p.4   & 249  & 48  & 25977 & 43057  & 22266 & 16.67 & 18677 & 25037 & 3.75    & {\bf 14001} & 25043 & {\bf  3.73}    \\ \hline
    \end{tabular}
    \end{table}

\subsection{Results}

The obtained numerical results are reported in Table~\ref{t:data}.
It is organized as follows: 
The first column group describes problem instance
with its ID, number of nodes ($n$) and clusters ($m$),
and the weight of the start solution, given by PCGLNS heuristic ($\UB$).
Then goes three groups of columns for Gurobi solver and
two proposed algorithms. 
Each group reports time to run (in seconds),
the best lower bound ($\LB$), and the final gap.
Each instance, where one of the proposed algorithms
outperforms Gurobi, is highlighted in bold.

As it follows from Table~\ref{t:data},
in 13 out of 39 instances (33\%)
some of our algorithms is a best performer
(either by accuracy or running time).
Especially we would like to emphasize
instances {\it ft70.1} and {\it kro124p.3} where
our algorithms managed to find a rather good feasible solution
whereas Gurobi failed.

In addition, for some instances,
start  solution given by PCGLNS was so close to optimal,
that our algorithms stopped almost immediately.

On the other hand,
for some instances
(eg. {\it p43.1, p43.2} and {\it p43.3}),
our algorithms are defeated by Gurobi,
we guess the reason is untight lower bounds.

To be fair,
we should note,
that Gurobi was provided
with very good MIP-start solution,
which is rather unusual in such experiments.


% !TeX root = ..
\section{Conclusion}\label{sec:summary}

In this paper,
we designed and implemented
the first problem-specific branch-and-bound
algorithm for precedence constrained GTSP.
The algorithm evolves ideas of the classic DP scheme of Held and Karp
and Salman's framework.

To evaluate performance of the proposed algorithms,
we carried out numerical experiments
in comparison with Gurobi solver
which show that our algorithms appear
to be quite competitive with
state-of-the-art MIP-solver.

To the future work we postpone
design of more tight lower bounds.
In addition, we believe that
further optimization and parallelization
can significantly speed up 
the implementation of our algorithms.

\subsection*{Acknowledgments}
The work was performed as a part of research carried out
in the Ural Mathematical Center with the financial support
of the Ministry of Science and Higher Education of the Russian Federation
(Agreement number 075-02-2021-1383).


% \clearpage
% \bibliographystyle{splncs04}
\bibliographystyle{plain}
\bibliography{pcgtsp}

\end{document}

