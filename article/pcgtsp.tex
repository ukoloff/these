\documentclass{article}

% \documentclass[runningheads]{llncs}

% !TeX root = ..

\usepackage[utf8]{inputenc}
\usepackage[T2A]{fontenc}
\usepackage[english,russian]{babel}

\usepackage{indentfirst}
\setlength{\parindent}{2.5em}

% !TeX root = ..

\usepackage{amssymb}
\usepackage{amsmath}
\usepackage{epsfig}
\usepackage{float}
\usepackage[caption=false]{subfig} 

\usepackage{multirow}
% \usepackage{array}
% \newcolumntype{P}[1]{>{\centering\arraybackslash}p{#1}}
\usepackage{longtable}


% !TeX root = ..

\usepackage{algorithm}
\usepackage{algorithmic}

\floatname{algorithm}{Алгоритм}
\newcommand{\ITP}[1]{#1^{\mathrm{ITP}}}
\newcommand{\DP}[1]{#1^{\mathrm{DP}}}

\newcommand{\APP}[1]{#1_{\mathrm{APP}}}
\newcommand{\CITP}[1]{#1_{\mathrm{CITP}}} 
\newcommand{\OPT}[1]{\mathrm{OPT}({#1})}
% \newcommand{\APP}[1]{\mathrm{APP}({#1})}
\newcommand{\OP}{\mathrm{OPT}}
\newcommand{\AP}{\mathrm{APP}} 
\renewcommand{\le}{\leq}
\renewcommand{\ge}{\geq}  

\newcommand{\R}{\mathfrak{R}}
\newcommand{\I}{\mathfrak{I}}
\newcommand{\CX}{\mathcal{C}}
\renewcommand{\L}{\mathcal{L}}
\renewcommand{\P}{\mathcal{P}}
\newcommand{\LB}{\mathrm{LB}}
\newcommand{\UB}{\mathrm{UB}}

\renewcommand{\algorithmicforall}{\textbf{for each}}

\usepackage{csquotes}

\usepackage[%
  parentracker=true,
  style=gost-numeric,
  defernumbers=true,
  movenames=false,
  maxnames=100,
  % sorting=none,
]{biblatex}

\toggletrue{bbx:gostbibliography}

\addbibresource{bib/self.bib}
\addbibresource{bib/others.bib}
\addbibresource{article/pcgtsp.bib}


\author{М.Ю. Хачай, С.С. Уколов, А.А. Петунин}
\title{
Специализированный алгоритм ветвей и границ
для обобщённой задачи коммивояжера
с ограничениями предшествования
}
% \titlerunning{BnB algorithms for PCGTSP}


% \author{Michael Khachay\inst{1}\orcidID{0000-0003-3555-0080} \and\\ Stanislav Ukolov\inst{2}\orcidID{0000-0002-9946-6446} \and\\ Alexander Petunin\inst{1,2}\orcidID{0000-0003-2540-1305}}

% \institute{Krasovsky Institute of Mathematics and Mechanics, Ekaterinburg, Russia \and Ural Federal University, Ekaterinburg, Russia\\
%   \email{mkhachay@imm.uran.ru},\ \  \email{s.s.ukolov@urfu.ru}, \  \ \email{aapetunin@gmail.com}}

	

\begin{document}

\maketitle

\begin{abstract}
Обобщенная задача коммивояжера (Generalized Traveling Salesman Problem, GTSP) -- 
это хорошо известная задача комбинаторной оптимизации, 
имеющая множество важных практических приложений при исследовании операций. 
В GTSP с ограничением предшествования (PCGTSP) 
накладываются дополнительные ограничения на порядок 
посещения кластеров в соответствии 
с некоторым заранее заданным частичным порядком.
В отличие от классической GTSP, 
PCGTSP все еще слабо исследована с точки зрения разработки и реализации алгоритмов.
Насколько нам известно, 
все известные алгоритмические подходы для этой проблемы 
исчерпываются общей структурой ветвления Салмана, 
несколькими моделями MILP 
и недавно предложенной авторами метаэвристикой PCGLNS. 
В данной работе мы представляем первый специализированный алгоритм ветвей и границ, 
разработанный с расширением подхода Салмана и использующий PCGLNS 
в качестве мощной первичной эвристики. 
Используя общедоступную тестовую библиотеку PCGTSPLIB, 
мы оцениваем производительность предложенного алгоритма 
по сравнению с классической схемой динамического программирования Хелда-Карпа,
дополненный стратегией ветвления и границ,
и современным решателем Gurobi, 
использующим нашу недавнюю моделью MILP 
и горячий старт на основе PCGLNS.

% \keywords{Generalized Traveling Salesman Problem \and precedence constraints \and Branch-and-Bound algorithm}
\end{abstract}  

%!TEX root = pcgtsp.tex
\section{Introduction}\label{sec:intro}
The Generalized Traveling Salesman Problem (GTSP) is a well-known combinatorial optimization problem introduced in the seminal paper \cite{SKGS1969} by S. Srivastava et al. and attracted the attention of many researchers~(see the survey in~\cite{GutinPunnen2007}). 

In the GTSP, for a given weighted digraph $G=(V,E,c)$ and partition $V_1\cup\ldots\cup V_m$ of the nodeset $V$ into non-empty mutually disjoint clusters, it is required to find a minimum cost closed tour $T$ that visits each cluster $V_i$ exactly once. 

In this paper, we consider the Precedence Constrained Generalized Traveling Salesman Problem (PCGTSP), where the clusters should be visited according to some given partial order. This extended version of the GTSP has numerous relevant industrial applications including 
\begin{itemize}
	\item[-] toolpath optimization for Computer Numerical Control (CNC) machines \cite{CASTELINO2003173}
	\item[-] \textit{air} time minimization in metal sheet cutting \cite{Petunin2018,Makarovskikh20181171}
	\item[-] coordinate measuring machinery \cite{SALMAN2016138} 
	\item[-] path optimization in multi-hole drilling \cite{DEWIL2019}.
\end{itemize}

\subsubsection{Related work.} The GTSP is an extension of the classic Traveling Salesman Problem (TSP). Therefore, any time when the number of clusters $m$ is a part of the input, the problem is strongly NP-hard even on the Euclidean plane \cite{Papa77}. On the other hand, the well-known Held and Karp dynamic programming scheme \cite{HeldKarp1962} adapted to the GTSP has running-time bound $O(n^3m^2\cdot 2^m)$, i.e. this algorithm belongs to FPT being parameterized by the number of clusters. Furthermore, the GTSP can be solved to optimality in polynomial time, provided $m=O(\log n)$.

As it follows from the literature, algorithmic design for the GTSP developed in several ways. 

The first approach is based on the reduction of the initial problem to some corresponding instance of the Asymmetric TSP, after that this auxiliary instance can be solved by an algorithm designed to the ATSP (\cite{LaporteSemet1999,NoonBean1993}). Despite its mathematical elegance, this approach suffers from a couple of shortcomings: 
\begin{itemize}
\item[(i)] the resulting ATSP instances have a rather unusual shape making their solution hard even for the state-of-the-art MIP solvers like Gurobi and CPLEX 
\item[(ii)] close-to-optimal solutions of these instances can produce infeasible solutions of the initial problem \cite{KaraGut2012}.
\end{itemize}

Another approach deals with developing problem-specific exact algorithms and approximation algorithms with theoretical performance guarantees. Among them are branch-and-bound and branch-and-cut algorithms (see, e.g. \cite{FishGonToth1997,Yuan2020}) and Polynomial Time Approximation Schemes (PTAS) for several special settings \cite{FerGriSit2006,KhN-PSIM2017}.  

Finally, the third approach is about designing various heuristics and meta-heuristics. Thus, G.Gutin and D.Karapetyan \cite{Gutin-2010} proposed an efficient memetic algorithm, in \cite{Helsgaun-2015}, the famous Lin-Kernighan-Helsgaun heuristic solver was extended to the GTSP, and in \cite{SMITH20171} the powerful Adaptive Large Neighborhood Search (ALNS) meta-heuristic was developed, which appear to be a best-performer to date. 

Unfortunately, for the PCGTSP, algorithmic results still remain quite rare. To the best of our knowledge, the published results are exhausted by
\begin{itemize}
	\item[(i)] efficient algorithms for the Balas-type special precedence constraints \cite{Balas-Sim2001,ChenKhKh2016,CKK-IFAC2016} and the precedence constraints leading to quasi- and pseudo-pyramidal optimal tours \cite{KhN-AMAI-2020}
	\item[(ii)] general idea of a problem-specific branch-and-bound algorithm for this problem \cite{SALMAN2020163}
	\item[(iii)] recent PCGLNS heuristic proposed by the authors \cite{KKP-optima2020} as an extension of the results of \cite{SMITH20171}. 
\end{itemize} 

In this paper, we try to bridge this gap.
\subsubsection{Contribution  of this paper} is three-fold:
\begin{itemize}
	\item[(i)] extending the idea proposed in \cite{SALMAN2020163}, we design and implement the first problem-specific algorithm for the PCGTSP
	\item[(ii)] extending the classic branching approach \cite{MorinMarsten1976}, we implement the Held and Karp dynamic programming scheme supplemented with an original bounding strategy
	\item[(iii)] to evaluate the proposed and implemented algorithms, we carry out numerical experiments on the public PCGTSPLIB test-bench \cite{SALMAN2020163}. We compare the performance of these algorithms against the state-of-the-art Gurobi MIP-solver armed with the best MIPS model. To put all the competitors on an equal footing, for any instance, we use the same approximate solution obtained by our recent heuristic PCGLNS \cite{PCGLNS} for the warm start.
\end{itemize}  
% \input{text/related-work}
%!TEX root = pcgtsp.tex
\section{Problem Statement}\label{sec:PS}
We consider the general setting of the Precedence Constrained Generalized Traveling Problem (PCGTSP). An instance of this problem is given by a triplet $(G,\C,\Pi)$, where
\begin{itemize}
	\item[-] an edge-weighted digraph $G=(V,E,c)$ defines a groundset network supplemented with transportation costs $c(u,v)$ for any arc $(u,v)\in E$
	\item[-] a partition $\C=\{V_1,\ldots,V_m\}$ splits the nodeset $V$ of the graph $G$ into $m$ non-empty pairwise-disjoint \textit{clusters}
	\item[-] a directed acyclic graph $\Pi=(\C,A)$ defines a partial order (\textit{precedence constraints}) on the set of clusters $\C$.      
\end{itemize}

For any node $v\in V$, by $V(v)$ we denote the (only) cluster $V_p\in\C$, such that $v\in V_p$. Further, without loss of generality, we assume $\Pi$ to be \textit{transitively closed} (i.e.  $(V_i,V_j)\in A$ and $(V_j,V_k)\in A$ imply $(V_i,V_k)\in A$) and that $(V_1,V_p)\in A$ for any $p\in\{2,\ldots,m\}$.

A closed $m$-tour $T$ is called a \textit{feasible} solution of the PCGTSP, if it
\begin{itemize}
	\item[-] departs for and arrives at some node $v_1\in V_1$
	\item[-] visits each cluster $V_p\in\C$
	\item[-] each arc $(v_i, v_j)$ of  $T$ (except the arc $(v_m,v_1)$) fulfills the precedence constraints, i.e. $(V(v_i),V(v_j))\in A$.
\end{itemize} 

To any tour $T=v_1, v_2, \ldots, v_m$, we assign its cost
$$
	cost(T) = c(v_m,v_1) + \sum_{i=1}^{m-1} c(v_i,v_{i+1}). 
$$ 
The goal is to find a feasible tour $T$ of the minimum cost $cost(T)$. 
% !TeX root = ..
\section{Общие соображения}\label{sec:pre}
Оба алгоритма,
разработанные и реализованные в данной работе,
используют общие основные идеи.

\subsection{Разбиение задачи}
В каждой вершине дерева поиска,
мы разделяем исходную задачу
на две (более простых) подзадачи
следующим образом:
\begin{enumerate}
	\item
	Рассмотрим подмножество
	$\CX'\subset\CX$,
	такое что $V_1\in\CX'$,
	зафиксируем некоторый кластер $V_l\in\CX'$
	и вершины $v\in V_1$ и $u\in V_l$
	соответственно
	\item
	пусть $c_{\min}$ -- нижняя граница стоимости $v-u$-путей,
	проходящих через все кластеры $\CX'$
	и удовлетворяющих ограничению предшествования\footnote{В нашем варианте динамического программирования эта граница будет точной}
	\item
	исключая из $\CX'$ все внутренние кластеры и соединяя
	$V_1$ с $V_l$ напрямую ребром нулевого (0) веса,
	мы тем самым создаём подзадачу $\P$,
	имеющую все те же остальные веса путей, разбиение на кластеры и ограничения предшествования,
	что и исходная
	\item
	принимая
	\begin{equation}\label{e:bounds}
		\LB = c_{\min} + \OPT{\P_{rel}}
	\end{equation}
	за нижнюю границу,
	мы отсекаем все узлы,
	для которых
	$\LB > \UB$.
	Здесь, $\OPT{\P_{rel}}$
	это вес некоторого эффективно находимого решения упрощённой задачи $\P$
	и $\UB$ -- стоимость наилучшего известного допустимого решения исходной задачи.
\end{enumerate}

\subsection{Нижние границы}\label{ssec:LBs}
В этом разделе мы сравним разные способы получения нижних границ
для вспомогательной проблемы
$\P$.
Рассмотрим несколько способов упростить
$\P$,
используя двухэтапный подход,
предложенный в
\cite{SALMAN2020163}.

На первом этапе мы упрощаем
$\P$,
превращая ее в задачу ATSP
одним из следующих способов:
\begin{enumerate}
	\item
	ослабляя исходное ограничение предшествования,
	исключаем все ребра
	$(v',v'')\in E$,
	для которых
	$(V(v''),V(v'))\in A$.
	Затем,
	сводим полученную задачу к ATSP,
	используя классическую трансформацию Нуна-Бина
	\cite{NoonBean1993}
	\item
	тем же способом ослабив исходное ограничение предшествования,
	сводим полученную задачу к ATSP,
	определённой на вспомогательном графе {\it кластеров}
	$H_1=(\tilde\CX',A_1, c_1)$, где
	\[
		\tilde\CX'=\CX\setminus\CX'\cup\{V_1,V_l\},
	\]
	\[\hspace*{-3ex}
	A_1=\{(V_1,V_l)\}\cup\{(V_i,V_j)\ |\  i>2, \{V_i,V_j\}\subset\tilde\CX', \exists (v'\in V_i, v''\in V_j)\colon (v',v'')\in E\},
	\]
	\[
	c_1(V_1,V_l) = 0,\ c_1(V_i,V_j) = \min\{c(v',v'')\colon v'\in V_i, v''\in V_j, (v',v'')\in E\}
	\]
	\item
	сводим исходную задачу к ATSP,
	определённой на ориентированном графе
	$H_2=(\tilde\CX',A_2, c_2)$,
	где
	\begin{multline*}
	A_2=\{(V_1,V_l)\}\cup\{(V_i, V_k)\ |\ i>2,\\ \exists (j>1): \{V_i,V_j,V_k\}\subset\tilde\CX' \wedge (\{(V_j,V_i), (V_k,V_j), (V_k, V_i)\}\cap A=\varnothing)\\
	 \wedge \exists (v'\in V_i, v''\in V_j, v'''\in V_k)\colon (\{(v',v''),(v'',v''')\}\subset E)\}\\
	\cup \{(V_i, V_k)\ |\ i>2, (\{V_i,V_k\}\subset\tilde\CX')\wedge ((V_k,V_i)\not\in A)\\
	\wedge \exists (v'\in V_i, v_1\in V_1, v''\in V_k)\colon \{(v',v_1), (v_1,v'')\}\subset E\},
	\end{multline*}
	то есть,
	для любого
	$V_i\in \tilde\CX'\setminus\{V_1\}$,
	упорядоченная пара $(V_i,V_k)\in A_2$,
	если существует $V_j\in\tilde\CX'$
	и вершины $v'\in V_i, v''\in V_j$
	и
	$v'''\in V_k$,
	такие, что путь $\pi = v',v'', v'''$
	не запрещён исходным ограничением предшествования
	$\Pi$.
	Далее,
	\[
	\hspace*{-3ex} c_2(V_1,V_l) = 0,
	\ c_2(V_i,V_k) = \min\{c(v',v'')+c(v'',v''')\colon \pi=v',v'', v''' \text{ -- допустимый путь}\}.
	\]
\end{enumerate}

На втором этапе,
мы находим приближенное решение полученной задачи ATSP,
путем нахождения либо минимального остовного дерева
(Minimum Spanning Arborescence Problem, MSAP),
либо решения задачи о назначениях
(Assignment Problem, AP)
и тем самым получаем значение нижней границы по формуле
\eqref{e:bounds}.
Кроме того,
мы можем посчитать ещё более точную
нижнюю границу,
прямо решая вспомогательную задачу ATSP
при помощи солвера Gurobi
(на практике только для задач, полученных способом 2).
Для удобства
все способы получения нижних границ сведены в
табл.~\ref{t:LBs}\subref{t:matrix}.

\begin{table}[h!]

\centering
	\scriptsize
    \caption{Нижние границы}\label{t:LBs}
    \subfloat[Обозначения]{\label{t:matrix}
    \begin{tabular}[t]{|c|c|c|c|}
    \hline
    & \textbf{Noon-Bean} & $\mathbf{H_1}$ & $\mathbf{H_2}$ \\
    \hline \hline
  \textbf{AP} & $E_1$ & $\mathbf{L_1}$ & $\mathbf{L_2}$\\
  \textbf{MSAP} & $E_2$ & $E_3$ & $E_4$\\
  \textbf{Gurobi} & $E_5$ & $\mathbf{L_3}$ & $E_6$\\
  \hline
    \end{tabular}
    }
   \quad
   \subfloat[Оценки по сравнению с $L_3$]{\label{t:confid}
   \begin{tabular}[t]{|*{3}{c|}}
   \hline
   $E_1$ & $E_2 = E_3$ &$E_4$ \\ \hline
   $0.48 \pm 0.03$ & $0.54 \pm 0.01$ & $0.60 \pm 0.002$\\ \hline \hline
   $L_1$ & $L_2$ & $L_3$ \\ \hline
   $0.91\pm 0.02$ & $0.97\pm 0.02$ & 1.00 \\
   \hline
   \end{tabular}
   }
\end{table}

На основе результатов численных экспериментов,
мы сократили полный список методов оценки нижней границы,
см. табл.~\ref{t:LBs}\subref{t:confid}.
На практике оценки
$L_1$--$L_3$
оказываются почти всегда наиболее строгими,
статистически значимо
с доверительным интервалом 95\%.
Мы также отказались от использования оценок
$E_5$ и $E_6$
ввиду того,
что они требуют гораздо большего времени счета.
Таким образом,
в разделе~\ref{sec:experiments}
мы ограничились использованием оценок
\[
	\LB_i = c_{\min} + L_i,\ i\in\{1,2,3\}.
\]


% !TeX root = ..
\section{Алгоритм ветвей и границ}\label{sec:SA}

Для решения задачи PCGTSP 
$(G,\CX,\Pi)$,
мы обходим дерево поиска в ширину
(Breadth First Search),
см.~Алгоритм~\ref{alg:bnb}.
Каждый узел этого дерева связан с префиксом
$\sigma = \left(V_{i_1}, V_{i_2}, \dots V_{i_r} \right)$,
где
$V_{i_j} \in \CX$,
$V_{i_1} = V_1$,
и
$r \in \{{1, \dots m}\}$.

\begin{algorithm}[]
\caption{BnB :: Главная процедура}\label{alg:bnb}
\hspace*{\algorithmicindent}{\bf Вход}: орграф $G$, кластеры $\CX$, частичный порядок $\Pi$ \\
\hspace*{\algorithmicindent}{\bf Выход}: маршрут и его стоимость
\begin{algorithmic}[1]
    \STATE инициализация $Q =$ empty queue
    \STATE начинаем с $Root = V_1$
    \STATE $Q$.push($Root$)
    \WHILE{\NOT $Q$.empty()}
        \STATE берём следующий префикс для обработки: $\sigma = Q$.pop()
        \STATE $process = Bound(\sigma)$
        \IF{\NOT $process$}
            \STATE префикс отсекается; {\bf continue}
        \ENDIF
        \STATE $UpdateLowerBound(\sigma)$
        \FORALL{$child \in Branch(\sigma)$}
            \STATE помещаем префикс в очередь на обработку $Q$.push($child$)
        \ENDFOR
    \ENDWHILE
\end{algorithmic}
\end{algorithm}

\begin{algorithm}[]
\caption{BnB :: Bound}\label{alg:bnb:bound}
\hspace*{\algorithmicindent}{\bf Вход}: префикс $\sigma$ \\
\hspace*{\algorithmicindent}{\bf Выход}: признак того, что префикс подлежит обработке
\begin{algorithmic}[1]
    \STATE {\bf global} $D_{ij}^{\mathcal T}$
    \STATE {\bf global} $Opt^{\mathcal T}$
    \STATE вычисляем кортеж $\mathcal T = \left(V_{i_1},
        \left\{V_{i_1}, V_{i_2}, \dots V_{i_r}\right\}, V_{i_r} \right)$
        \label{alg:bnb:bound:key}
    \STATE $D_{ij} = MinCosts(\sigma)$
        \label{alg:bnb:bound:pfx}
    \IF{$D_{ij}^{(\sigma)} \ge D_{ij}^{\mathcal T}[\mathcal T], \forall i, j$}
        \RETURN \FALSE
    \ENDIF
    \STATE  обновляем веса маршрутов $D_{ij}^{\mathcal T}[\mathcal T]  = \min \left(
        D_{ij}^{\mathcal T}[\mathcal T], D_{ij} \right),
        \forall i, j$
    \STATE  $c_{min} = \min\limits_{i, j} D_{ij}$
    \IF{$\mathcal T \notin Opt^{\mathcal T}$}
        \STATE вычисляем нижнюю границу $Opt^{\mathcal T}[\mathcal T] = \max\left(L_1(\sigma), L_2(\sigma))\right)$
        \label{alg:bnb:bound:sfx}
    \ENDIF
    \STATE $\LB = c_{min} + Opt^{\mathcal T}[\mathcal T]$
        \label{alg:bnb:bound:lb}
    \IF{$\LB > \UB$}
        \RETURN \FALSE
    \ENDIF
    \RETURN \TRUE
\end{algorithmic}
\end{algorithm}

К каждому узлу дерева поиска
мы применяем процедуру отсечения
Bound
(Алгоритм~\ref{alg:bnb:bound}),
которая выполняет следующие действия:
\begin{itemize}
    \item
    для префикса $\sigma$,
    мы находим кортеж
    % $\mathcal T(\sigma)$
$$
\mathcal T(\sigma) = \left(V_{i_1},
        \left\{V_{i_1}, V_{i_2}, \dots V_{i_r}\right\}, V_{i_r} \right)
$$
    \item
    на шаге~\ref{alg:bnb:bound:pfx},
    мы вычисляем матрицу $D(\sigma)$ 
    минимальных попарных весов по формуле:
    $$
    \hspace{-2ex}
    D(\sigma)_{vu} = \min\left\{
      cost(P_{v,u})\colon
      v \in V_{i_1},
      u \in V_{i_r},
      P_{v,u} \text{ путь $v$-$u$ в порядке } \sigma
    \right\}.
    $$
    Эта матрица удобно вычисляется инкрементально
    на основе матрицы
    $D(\sigma')$
    родительского узла дерева поиска
    \item
    если, для некоторого $\sigma_1$, $\mathcal T(\sigma) = \mathcal T(\sigma_1)$ 
    и
    $$
    D(\sigma)_{vu} \ge D(\sigma_1)_{vu}, \quad
    (v \in V_{i_1}, u \in V_{i_r}),
    $$
    то,
    префикс $\sigma$ имеет веса в матрице$D(\sigma)$
    больше, чем для префикса $\sigma_1$
    и подлежит отсечению
    \item
    на шаге~\ref{alg:bnb:bound:sfx},
    мы рассчитываем оценки $L_1$ и $L_2$,
    см. табл.~\ref{t:LBs} 
    и сохраняем их в глобальной переменной
    $Opt^{\mathcal T}$,
    используя формулу
    $$
    Opt^{\mathcal T(\sigma)} = \max(L_1, L_2)
    $$
    \item
    для текущего узла 
    $\sigma$,
    рассчитываем нижнюю границу по формуле
    $$
    \LB(\sigma) = \min_{vu}D(\sigma)_{vu} + Opt^{\mathcal T(\sigma)}
    $$
    на шаге~\ref{alg:bnb:bound:lb}
    \item
    наконец, 
    узел $\sigma$ отсекается, если $\LB > \UB$.
\end{itemize}

\begin{algorithm}[]
\caption{BnB :: Branch}\label{alg:bnb:branch}
\hspace*{\algorithmicindent}{\bf Вход}: префикс $\sigma$ \\
\hspace*{\algorithmicindent}{\bf Выход}: список потомков префикса для обработки
\begin{algorithmic}[1]
    \STATE инициализация $R =$ empty queue
    \FORALL{$V \in \CX$}
        \STATE $valid =$ \TRUE
        \FORALL{$W \in \sigma$}
            \IF{$W=V$ \OR $(V, W) \in \Pi$}
                \STATE $valid =$ \FALSE
                \STATE {\bf break}
            \ENDIF
        \ENDFOR
        \IF{$valid$}
            \STATE добавляем новый префикс $R$.push($\sigma+V$)
        \ENDIF
    \ENDFOR
    \RETURN $R$
\end{algorithmic}
\end{algorithm}

Префиксы,
которые избежали отсечения,
обрабатываются процедурой 
$Branch$
(Алгоритм~\ref{alg:bnb:branch}),
которая пытается удлинить префикс 
$\sigma$
на один кластер,
соблюдая при этом ограничение предшествования
$\Pi$.

% !TeX root = ..
\section{Динамическое программирование}\label{sec:DP}
Алгоритм ветвей и границ,
описанный в разделе~\ref{sec:SA},
оказывается сильно связан с классической схемой,
использующей динамическое программирование (DP)
и носящей имя Хелда-Карпа
\cite{HeldKarp1962},
адаптированной для учёта ограничения предшествования
и дополненной стратегией отсечения,
представленной в основополагающей статье~\cite{MorinMarsten1976}.

\begin{algorithm}[t]
\caption{DP ::  индуктивное построение таблицы поиска}\label{alg:A2}
\hspace*{\algorithmicindent}\textbf{Вход:} орграф $G$, частичный порядок $\Pi$, слой таблицы поиска $\L_k$, верхняя граница $\UB$\\
% \hspace*{\algorithmicindent} \textbf{Parameters:} \textit{number\_of\_trials}, \textit{acceptance\_criterion}, \textit{termination\_criterion}\\
\hspace*{\algorithmicindent}\textbf{Выход:} $(k+1)$-ый слой $\L_{k+1}$
\begin{algorithmic}[1]
\STATE инициализация $\L_{k+1}=\varnothing$
\FORALL{$\CX'\in\I_k$}
  \FORALL{кластер $V_l\in\CX\setminus\CX'$, s.t. $\CX'\cup \{V_l\}\in\I_{k+1}$}
    \FORALL{$v\in V_1$ и $u\in V_l$}
      \IF{есть состояние $S=(\CX',U,v,w)\in\L_k$, s.t. $(w,u)\in E$}
      \STATE создаем новое состояние $S'=(\CX'\cup\{V_l\}, V_l, v, u)$
      \STATE $S'[cost] = \min\{S[cost] + c(w,u)\colon S=(\CX',U,v,w)\in\L_k\}$
      \STATE $S'[pred] = \arg\min\{S[cost] + c(w,u)\colon S=(\CX',U,v,w)\in\L_k\}$
      \STATE $S'[LB] = S'[cost] + \max\{L_1,L_2,L_3\}$
      \IF{$S'[\LB] \leqslant \UB$}
        \STATE добавляем $S'$ к $\L_{k+1}$
      \ENDIF
    \ENDIF
    \ENDFOR
  \ENDFOR
\ENDFOR
\RETURN $\L_{k+1}$
\end{algorithmic}
\end{algorithm}

В данной работе мы реализуем уточненную версию этой схемы 
для численной оценки производительности нашего алгоритма ветвей и границ.
Подобно классическому DP,
наш алгоритм состоит из двух этапов.
\begin{enumerate}
  \item
  На этом этапе таблица поиска строится инкрементально,
  в прямом направлении, 
  слой за слоем.
  Оптимальная стоимость для решаемой задачи
  находится после вычисления последнего $m$-го слоя.   
  \item
  Оптимальный маршрут реконструируется обратным просмотром 
  на основе данных, 
  хранящихся в таблице поиска.
\end{enumerate}

Каждое состояние DP
(запись в таблице поиска)
соответствует частичному 
$v$-$u$-пути
и индексируется кортежем
$(\CX',V_l,v, u)$, где
\begin{enumerate}
  \item
  $\CX'\subset \CX$ представляет собой \textit{идеал} частично упорядоченного множества кластеров $\CX$, то есть
  \[
    \forall (V\in\CX', V'\in\CX)\   (V',V)\in A) \Rightarrow (V'\in\CX');
  \]
  очевидно,
  в наших условиях,
  $V_1$ 
  принадлежит произвольному идеалу
  $\CX'\subset\CX$

  \item
  $V_l\subset\CX'$, 
  для которого нет 
  $V\in \CX'$, 
  такого, что
  $(V_l,V)\in A$
  \item
  $v\in V_1$, $u\in V_l$.
\end{enumerate}
Содержимое каждой записи DP
$S$ 
состоит из ссылки
$S[pred]$
на предшествующее состояние,
локальной нижней границы
$S[LB]$
и стоимости
$S[cost]$
соответствующего частичного 
$v$-$u$-пути.

Пусть $\I_k$
-- подмножество идеалов одного размера
$k\in\{1,\ldots,m\}$. 
Очевидно,
$\I_1=\{\{V_1\}\}$, 
а значит первый слой
$\L_1$
таблицы поиска строится тривиально.
Индуктивное построение остальных слоев
описано в Алгоритме~\ref{alg:A2}.

\subsection{Замечания}
\begin{enumerate}
  \item 
  Оптимум для решаемой задачи дается классическим уравнением Беллмана
  \[
    \mathrm{OPT}=\min_{v\in V_1}\min\{S[cost]+c(u,v)\colon S=(\CX',V_l, v, u)\in\L_m\}
  \]
  \item 
  По построению,
  размер таблицы поиска
  $O(n^2m\cdot |\I|)$. 
  Значит, время работы нашего алгоритма
  $O(n^3m^2\cdot |\I|)$. 
  В частности, 
  в случае частичного порядка фиксированной {\it ширины}
  $w$, $|\I|=O(m^w)$ \cite{Steiner-1990}. 
  Следовательно,
  оптимальное решение 
  PCGTSP 
  может быть найдено в этом случае за полиномиальное время
  даже без применения отсечения на шагах 10--12.

  \item
  После построения любого из слоев
  $\L_k$, 
  мы обновляем глобальное значение нижней границы,
  что приводит к сокращению общего разрыва.

  \item
  В нашей реализации,
  для повышения быстродействия
  мы вычисляем оценку $L_3$ 
  на шаге 9 
  только для небольшого количества состояний
  с наименьшей нижней границей.
\end{enumerate}


% !TeX root = ..
\section{Численные эксперименты}\label{sec:experiments}

В данном разделе приводятся результаты численных экспериментов
по оценке производительности предлагаемого
алгоритма ветвей и границ
в сравнении со схемой динамического программирования
а также решателем Gurobi,
использующим нашей недавней MILP-моделью
\cite{KKP-optima2020}.

\begin{table}[p]
  \centering
  \caption{Результаты экспериментов}
  \label{t:data}
  \scriptsize
  \def\arraystretch{1.5}
  \hspace*{-5ex}
  \begin{tabular}{|r|c*{12}{|r}|}
  \hline
  \multicolumn{5}{|c|}{\textit{Задача}} &
    \multicolumn{3}{c|}{\textit{Gurobi}} &
    \multicolumn{3}{c|}{\textit{Ветвей и границ}} &
    \multicolumn{3}{c|}{\textit{DP}} \\ \hline
    \multicolumn{1}{|c|}{\textit{№}} &
    \multicolumn{1}{c|}{\textit{ID}} &
    \multicolumn{1}{c|}{\textit{n}} &
    \multicolumn{1}{c|}{\textit{m}} &
    \multicolumn{1}{c|}{\textit{UB}} &
    \multicolumn{1}{c|}{\textit{Время}} &
    \multicolumn{1}{c|}{\textit{LB}} &
    \multicolumn{1}{c|}{\textit{gap, \%}} &
    \multicolumn{1}{c|}{\textit{Время}} &
    \multicolumn{1}{c|}{\textit{LB}} &
    \multicolumn{1}{c|}{\textit{gap, \%}} &
    \multicolumn{1}{c|}{\textit{Время}} &
    \multicolumn{1}{c|}{\textit{LB}} &
    \multicolumn{1}{c|}{\textit{gap, \%}} \\ \hline
  {\bf 1}  & br17.12   & 92   & 17  & 43    & 107.28 & 43    & 0.00  & {\bf 11.2} & 43    & {\bf 0.00}    & 27.3   & 43    & 0.00    \\ \hline
  2  & ESC07     & 39   & 8   & 1730  & 0.07   & 1730  & 0.00  & 1.3   & 1726  & 0.23    & 8.37   & 1730  & 0.00    \\ \hline
  3  & ESC12     & 65   & 13  & 1390  & 0.52   & 1390  & 0.00  & 4.3   & 1385  & 0.36    & 14.99  & 1390  & 0.00    \\ \hline
  {\bf 4}  & ESC25     & 133  & 26  & 1418  & 4.45   & 1383  & 2.53  & {\bf 32} & 1383  & {\bf 0.00}    & 60.69  & 1383  & 0.00    \\ \hline
  5  & ESC47     & 244  & 48  & 1399  & 52.01  & 1063  & 31.61 & 36000 & 980   & 42.76   & 36000  & 981   & 42.61   \\ \hline
  {\bf 6} & ESC63     & 349  & 64  & 62    & 380.65 & 62    & 0.00  & 1.3   & 62    & 0.00    & {\bf 0.52}   & 62    & {\bf 0.00}    \\ \hline
  {\bf 7}  & ESC78     & 414  & 79  & 14832 & 43200  & 14581 & 1.72  & 1.3   & 14594 & 1.63    & {\bf 0.68}   & 14594 & {\bf 1.63}    \\ \hline
  8  & ft53.1    & 281  & 53  & 6207  & 42099  & 6022  & 2.96  & 36000 & 4839  & 28.27   & 36000  & 4839  & 28.27   \\ \hline
  9  & ft53.2    & 274  & 53  & 6653  & 42137  & 6184  & 7.58  & 36000 & 4934  & 34.84   & 36000  & 4940  & 34.68   \\ \hline
  10 & ft53.3    & 281  & 53  & 8446  & 42194  & 6936  & 21.77 & 36000 & 5465  & 54.55   & 36000  & 5465  & 54.55   \\ \hline
  11 & ft53.4    & 275  & 53  & 11822 & 23239  & 11822 & 0.00  & 35865 & 11274 & 4.86    & 2225   & 11290 & 4.71    \\ \hline
  {\bf 12} & ft70.1    & 346  & 70  & 32848 & \multicolumn{3}{c|}{Истечение времени}   & 36000 & 31153 & 5.44    & 36000  & 31177 & {\bf 5.36}    \\ \hline
  13 & ft70.2    & 351  & 70  & 33486 & 42021  & 31840 & 5.05  & 36000 & 31268 & 7.09    & 36000  & 31273 & 7.08    \\ \hline
  14 & ft70.3    & 347  & 70  & 35309 & 41173  & 32944 & 7.18  & 36000 & 32180 & 9.72    & 36000  & 32180 & 9.72    \\ \hline
  {\bf 15} & ft70.4    & 353  & 70  & 44497 & 41827  & 41378 & 7.53  & 36000 & 38989 & 14.13   & 36000  & 41640 & {\bf 6.86}    \\ \hline
  16 & kro124p.1 & 514  & 100 & 33320 & 34162  & 29926 & 11.34 & 36000 & 27869 & 19.56   & 36000  & 27943 & 19.24   \\ \hline
  17 & kro124p.2 & 524  & 100 & 35321 & 35379  & 30101 & 17.34 & 36000 & 28155 & 25.45   & 36000  & 28155 & 25.45   \\ \hline
  {\bf 18} & kro124p.3 & 534  & 100 & 41340 & \multicolumn{3}{c|}{Истечение времени}   & 36000 & 28406 & 45.53   & 36000  & 28406 & {\bf 45.53}   \\ \hline
  19 & kro124p.4 & 526  & 100 & 62818 & 41035  & 46704 & 34.50 & 36000 & 38137 & 64.72   & 36000  & 38511 & 63.12   \\ \hline
  20 & p43.1     & 203  & 43  & 22545 & 43150  & 22327 & 0.98  & 36000 & 738   & 2954.88 & 36000  & 788   & 2761.04 \\ \hline
  21 & p43.2     & 198  & 43  & 22841 & 43132  & 22381 & 2.05  & 36000 & 749   & 2949.53 & 36000  & 877   & 2504.45 \\ \hline
  22 & p43.3     & 211  & 43  & 23122 & 43058  & 22540 & 2.57  & 36000 & 898   & 2474.83 & 36000  & 906   & 2452.10 \\ \hline
  {\bf 23} & p43.4     & 204  & 43  & 66857 & 43193  & 45396 & 47.26 & 4470  & 66846 & 0.00    & {\bf 333.02} & 66846 & {\bf 0.00}    \\ \hline
  24 & prob.100  & 510  & 99  & 1474  & 38567  & 800   & 80.25 & 36000 & 632   & 133.23  & 36000  & 632   & 133.23  \\ \hline
  25 & prob.42   & 208  & 41  & 232   & 1292.1 & 202   & 14.85 & 36000 & 149   & 55.70   & 36000  & 153   & 51.63   \\ \hline
  26 & rbg048a   & 255  & 49  & 282   & 64.32  & 282   & 0.00  & 0.9   & 272   & 3.68    & 0.25   & 272   & 3.68    \\ \hline
  {\bf 27} & rbg050c   & 259  & 51  & 378   & {\bf 26.46}  & 378   & {\bf 0.00}  & {\bf 0.2}   & 372   & {\bf 1.61}    & 0.25   & 372   & 1.61    \\ \hline
  28 & rbg109a   & 573  & 110 & 848   & 83.23  & 848   & 0.00  & 2407  & 812   & 4.43    & 682    & 809   & 4.82    \\ \hline
  {\bf 29} & rbg150a   & 871  & 151 & 1415  & {\bf 29095}  & 1414  & {\bf 0.07}  & {\bf 0.4}   & 1353  & {\bf 4.58}    & 0.53   & 1353  & 4.58    \\ \hline
  30 & rbg174a   & 962  & 175 & 1644  & 5413.8 & 1641  & 0.18  & 0.4   & 1568  & 4.85    & 0.67   & 1568  & 4.85    \\ \hline
  {\bf 31} & rbg253a   & 1389 & 254 & 2376  & 43159  & 2369  & {\bf 0.13}  & {\bf 0.8} & 2269  & {\bf 4.72} & 1.42   & 2269  & 4.72    \\ \hline
  {\bf 32} & rbg323a   & 1825 & 324 & 2547  & 40499  & 2533  & {\bf 0.55}  & {\bf 2}  & 2448  & {\bf 4.04} & 3.59   & 2448  & 4.04    \\ \hline
  33 & rbg341a   & 1822 & 342 & 2101  & 30687  & 2064  & 1.41  & 36000 & 1840  & 14.18   & 36000  & 1840  & 14.18   \\ \hline
  34 & rbg358a   & 1967 & 359 & 2080  & 32215  & 2021  & 2.38  & 36000 & 1933  & 7.60    & 36000  & 1933  & 7.60    \\ \hline
  35 & rbg378a   & 1973 & 379 & 2307  & 42279  & 2231  & 2.38  & 36000 & 2032  & 13.53   & 36000  & 2031  & 13.59   \\ \hline
  36 & ry48p.1   & 256  & 48  & 13135 & 43141  & 12125 & 8.33  & 36000 & 10739 & 22.31   & 36000  & 10764 & 22.03   \\ \hline
  37 & ry48p.2   & 250  & 48  & 13802 & 43033  & 12130 & 13.78 & 36000 & 10912 & 26.48   & 36000  & 11000 & 25.47   \\ \hline
  38 & ry48p.3   & 254  & 48  & 16540 & 43102  & 13096 & 26.30 & 36000 & 11732 & 40.98   & 36000  & 11822 & 39.91   \\ \hline
  {\bf 39} & ry48p.4   & 249  & 48  & 25977 & 43057  & 22266 & 16.67 & 18677 & 25037 & 3.75    & {\bf 14001} & 25043 & {\bf  3.73}    \\ \hline
  \end{tabular}
\end{table}

\subsection{Условия эксперимента}
Все алгоритмы тестировались на общедоступной библиотеке
PCGTSPLIB
\cite{SALMAN2020163}. 
Во всех случаях для теплого старта,
всем алгоритмам предоставляется одно и то же
допустимое решение,
полученное эвристическим решателем
PCGLNS
\cite{PCGLNS}. 
Для алгоритмов ветвей и границ и динамического программирования,
все вычисления проводятся на одном и том же оборудовании
(16-ядерный Intel Xeon, 128G RAM)
с предельным временем счета 10~часов.
В качестве критерия остановки 
мы используем понижение разрыва ниже 5\%,
где разрыв определяется по формуле
$$
gap = \frac{\UB - \LB}{\LB}.
$$

В качестве базы сравнения,
мы используем результаты численных экспериментов,
представленные в 
\cite{KKP-optima2020}
для тех же самых экземпляров задач
(и на том же оборудовании),
полученные при помощи решателя Gurobi и
эвристики PCGLNS
(предельное время счета 12 часов).

\subsection{Обсуждение}

Полученные результаты эксперимента
представлены в табл.~\ref{t:data},
которая организована следующим образом:
первая группа столбцов описывает
задачу,
включая её обозначение ID,
количество вершин $n$
и кластеров $m$,
а также стоимость стартового решения $\UB$,
полученного эвристикой PCGLNS.
Затем следуют три группы столбцов
для решателя Gurobi
и двух предлагаемых алгоритмов.
Каждая группа содержит время
счета в секундах,
наилучшее значение нижней границы
$\LB$
и наилучший разрыв gap
в процентах.
Задачи,
в которых один из предлагаемых
алгоритмов сработал лучше Gurobi,
выделены жирным шрифтом.

Как следует из табл.~\ref{t:data},
для 13 из 39 задач (33\%)
один из наших алгоритмов
показал лучшую производительность
(либо в точности решения, либо во времени счета).
Особенно мы хотим выделить задачи
{\it ft70.1} и {\it kro124p.3},
где наши алгоритмы смогли найти допустимое решение,
тогда как Gurobi не смог этого сделать.

Кроме того,
для некоторых задач,
стартовое решение,
полученное эвристикой PCGLNS
оказалось настолько близким к оптимальному,
что наши алгоритмы завершают работу
почти немедленно.

С другой стороны,
для некоторых задач
(например, {\it p43.1, p43.2} и {\it p43.3}),
результаты наших алгоритмов
оказались крайне слабы по сравнению с Gurobi,
что по видимому объясняется
очень грубыми оценками нижней границы.

Считаем нужным добавить,
что в наших экспериментах
решателю Gurobi было предоставлено,
так же как и тестируемым алгоритмам,
хорошее стартовое решение,
что является не очень обычным способом
организации эксперимента.


% !TeX root = ..
\section{Заключение}\label{sec:summary}

В данной работе
разработан и реализован
первый специализированный алгоритм
ветвей и границ
для обобщенной задачи коммивояжера
с ограничениями предшествования.
Он развивает идеи классической схемы
динамического программирования
Хелда-Карпа
и схемы Салмана.

Для оценки производительности
предложенных алгоритмов,
проведены численные эксперименты,
в качестве базы сравнения использован
решатель Gurobi.
Эксперименты продемонстрировали,
что наши алгоритмы 
вполне конкурентноспособны на уровне
современных MIP-решателей.

В качестве направления дальнейших исследований
мы предполагаем разработку более 
жестких нижних границ.
Кроме того,
мы полагаем,
что дальнейшая оптимизация
и распараллеливание
могут существенно улучшить производительность
наших алгоритмов.

\subsection*{Благодарность}
The work was performed as a part of research carried out
in the Ural Mathematical Center with the financial support
of the Ministry of Science and Higher Education of the Russian Federation
(Agreement number 075-02-2021-1383).

Работа выполнена
в ходе исследований
Уральского Математического Центра
при финансовой поддержке Министерства науки и высшего образования РФ, 
соглашение № 075-02-2021-1383.


% \clearpage
% \bibliographystyle{splncs04}
% \bibliographystyle{plain}
% \bibliography{pcgtsp}
\printbibliography[]

\end{document}

