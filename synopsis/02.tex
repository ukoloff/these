%!TEX root = ../synopsis.tex

Во {\bf второй главе} 
рассматривается задача непрерывной резки {\it CCP}
на Эвклидовой плоскости
$\mathbb R \times \mathbb R$.
Возьмём
$N$
попарно непересекающихся плоских контуров
$\{C_1, C_2, ... C_N\}$,
ограничивающих
$n$
деталей
$\{A_1, A_2, ... A_n\}$.
В общем случае
$n \leqslant N$.
В данной работе рассматриваются только контуры 
$C_i$,
состоящие из
(конечного числа)
отрезков прямых линий и дуг окружностей,
так как именно такие геометрические примитивы
поддерживаются программным обеспечением
современных машин термической резки с ЧПУ.
Выберем также две точки 
$M_0$, $M_{N + 1}$
(почти всегда $M_0 = M_{N + 1}$),
которые будут использоваться
как начало и конец
маршрута резки.
Задача непрерывной резки
({\it Continuous Cutting Problem, CCP})
состоит в поиске:
\begin{enumerate}
\item
$N$ точек врезки $M_i \in C_i, i \in \overline{1, N}$
\item
Последовательности обхода контуров
$C_i$,
то есть перестановки
$N$
элементов
$I = (i_1, i_2, ... i_N)$
\end{enumerate}
Результатом решения задачи будет являться маршрут
\begin{equation}
  \{M_0, M_{i_1}, M_{i_2}, \dots M_{i_N}, M_{N + 1}\}
\end{equation}
Целевая функция в данном случае 
и сводится фактически к минимизации длины холостого хода:
\begin{equation}
  \mathcal{L} = \sum_{j=0}^N|M_{i_j}M_{i_{j+1}}|
  \label{air-move-length}
\end{equation}
$$
\mathcal{L} \to \min
$$
где, для простоты записи мы полагаем
$M_{i_0} = M_0$,
$M_{i_{N + 1}} = M_{N + 1}$.
