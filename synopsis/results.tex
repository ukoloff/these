В соответствии с целью и задачами исследования получены следующие
научные и практические результаты:

\begin{enumerate}
    \item
    Разработана схема эффективного учёта ограничений предшествования для
    задачи непрерывной резки.
    \item
    Разработана основанная на геометрических соображениях эвристика
    размещения точек врезки на плоских контурах.
    \item
    Доказано, что данная эвристика доставляет локальный минимум длины холостого хода
    и сформулированы два набора достаточных условий того,
    что полученное решение является глобальным минимумом.
    \item
    Эти две схемы могут соединяться с различными алгоритмами дискретной оптимизации,
    тем самым получается полный алгоритм решения задачи непрерывной резки.
    В данной работе для дискретной оптимизации используется
    метод переменных окрестностей.
    \item
    Полученный алгоритм даёт решения хорошего качества за разумное время.
    В случае, когда известно точное решение соответствующей обобщённой
    задачи коммивояжера, они визуально совпадают, но длина холостого
    хода для решения задачи непрерывной резки оказывается короче за счёт
    отсутствия дискретизации.
    \item
    Продемонстрировано, что полученный алгоритм может использоваться для
    решения задач сегментной непрерывной резки (SCCP)
    и обобщённой сегментной непрерывной резки (GSCCP),
    тем самым открывая подход к решению общей задачи прерывистой резки
    (ICP)
    \item
    Разработан алгоритм Branch-and-Bound для точного решения
    обобщённой задачи коммивояжёра с ограничениями предшествования,
    рассчитывающий нижнюю границу
    \item
    Алгоритм способен находить точные решения для задач большего размера,
    чем другие алгоритмы.
    В проведённых экспериментах было найдено решение для
    задачи со 151 кластером.
    \item
    Данный алгоритм также решает важную задачу оценки качества
    решений, полученных другими алгоритмами,
    даже в случае, когда нахождение точного решения непрактично.
    \item
    Алгоритм может быть реализован в классической схеме,
    а также в парадигме динамического программирования.
    В последнем случае он естественным образом допускает
    распараллеливание и демонстрирует лучшую производительность.
    \item
    Разработаны форматы данных для обмена геометрической и маршрутной
    информацией и визуализации для использования в CAD/CAM-системах,
    а также алгоритмы преобразования на примере САПР <<Сириус>>.
    Аналогичные конвертеры могут легко разрабатываться для других САПР.
    \item
    Разработано программное обеспечение для реализации всех алгоритмов
    на языках C, Python и JavaScript.
\end{enumerate}

\paragraph*{Перспективы дальнейшей разработки темы.}
Можно выделить следующие направления дальнейшего развития и совершенствования алгоритмического и
программного обеспечения САПР УП для оборудования листовой фигурной резки с ЧПУ:

\begin{enumerate}
    \item
    Учёт дополнительных ограничений в алгоритме решения задачи
    непрерывной резки, в частности того,
    что точка врезки располагается не на самом контуре,
    а на некотором расстоянии от него, а также того, что
    существуют зоны, где не могут размещаться зоны врезки,
    а также других технологических ограничений,
    порождаемых современным оборудованием термической резки с ЧПУ
    \item
    Использование других алгоритмов дискретной оптимизации в задаче
    непрерывной резки и оценка производительности и качества получаемых
    алгоритмов.
    \item
    Разработка других методов получения оценок для частичных подзадач GTSP
    с целью повышения нижней границы;
    например, за счёт более точного учёта расстояний между точками,
    а не только между кластерами.
\end{enumerate}
