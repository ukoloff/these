В соответствии с целью и задачами исследования получены следующие
научные и практические результаты:

\begin{enumerate}
  \item
  Разработан алгоритм ветвей и границ для точного решения
  обобщённой задачи коммивояжёра с ограничениями предшествования.
  Он может быть реализован в классической схеме,
  а также в парадигме динамического программирования,
  при этом он допускает
  распараллеливание и демонстрирует лучшую производительность.
  \item
  Предложенный алгоритм способен находить точные решения для задач большего размера,
  чем известные алгоритмы.
  В проведённых экспериментах было найдено решение для
  задачи со 151 кластером.
  \item
  Данный алгоритм также решает важную задачу оценки качества
  решений, полученных другими алгоритмами.
  \item
  Разработана схемы эффективного учёта ограничений предшествования
  для задач маршрутизации как в дискретной,
  так и в непрерывной схеме оптимизации.
  \item
  Разработана основанная на геометрических соображениях эвристика
  оптимального размещения точек врезки на плоских контурах.
  \item
  Доказано, что данная эвристика доставляет локальный минимум длины холостого хода
  и сформулированы два набора достаточных условий того,
  что полученное решение является глобальным минимумом.
  \item
  Разработанные алгоритмы могут использоваться для
  решения задач сегментной непрерывной резки (SCCP и GSCCP)
  тем самым открывая подход к решению общей задачи прерывистой резки
  (ICP)
  \item
  Разработаны форматы данных
  и алгоритмические схемы
  для обмена геометрической и маршрутной
  информацией и визуализации для использования в CAD/CAM-системах,
  а также алгоритмы преобразования формата файлов,
  что позволило интегрировать разработанное ПО
  с САПР~<<Сириус>>
  и T-Flex CAD.
  \item
  Разработано программное обеспечение для реализации всех алгоритмов
  на языках C, Python и JavaScript.
\end{enumerate}

\paragraph*{Перспективы дальнейшей разработки темы.}
Можно выделить следующие направления дальнейшего развития и совершенствования алгоритмического и
программного обеспечения САПР УП для оборудования листовой фигурной резки с ЧПУ:

\begin{enumerate}
    \item
    Разработка методов получения нижних оценок для частичных подзадач GTSP;
    например, за счёт более точного учёта расстояний между узлами,
    а не только между кластерами.
    \item
    Разработка метаэвристических алгоритмов дискретной оптимизации в задаче
    непрерывной резки и оценка производительности и качества получаемых
    алгоритмов.
    \item
    Учет технологических требований термической резки
    \item
    Интеграция разработанных алгоритмов с отечественными САПР
    для проектирования УП машин листовой резки
\end{enumerate}
