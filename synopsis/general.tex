\paragraph*{Актуальность темы исследования.}

Современное производство предъявляет высокие требования к качеству заготовок,
технико-экономическому уровню выпускаемой продукции,
что приводит к увеличению затрат на проектирование и технологическую подготовку производства.
Одним из направлений повышения эффективности использования
производственных ресурсов является совершенствование безотходных технологий
в~металлообрабатывающих производствах и~возрастание степени их автоматизации.

Раскройно-заготовительные операции,
являясь началом большинства производственных процессов,
оказывают существенное влияние на трудоёмкость
и экономичность изготовления детали.
Для получения заготовок сложной
геометрической формы из листового материала в условиях мелкосерийного и
единичного производства широко применяются машины фигурной резки с
числовым программным управлением
(ЧПУ).
К данному типу оборудования
относятся станки газовой, лазерной, плазменной, электроэрозионной
и~гидроабразивной резки металла.
Станки листовой резки имеют множество преимуществ:
возможность обработки многих видов материалов различной толщины,
высокая скорость резки, возможность обработки контуров различной сложности,
адаптация к постоянным изменениям номенклатуры выпускаемой продукции.
Использование оборудования с ЧПУ, предполагает применение
средств автоматизации проектирования управляющих программ
(CAM-систем).
При использовании современных CAD/CAM систем, предназначенных для
автоматизированного проектирования раскроя и подготовки
управляющих программ
(далее -- УП)
для машины с ЧПУ, возникает несколько различных взаимосвязанных задач,
поэтому обычно
проектирование УП для технологического оборудования листовой резки
состоит из нескольких этапов.
Первый этап предполагает предварительное геометрическое моделирование заготовок
и разработку раскройной карты листового материала.
На этапе проектирования раскроя возникает известная задача оптимизации фигурного раскроя листового материала,
которая с точки зрения геометрической оптимизации
относится к классу трудно решаемых проблем раскроя-упаковки
(\textit{Cutting \& Packing}).
На следующем этапе проектирования УП осуществляется процесс назначения траектории
перемещения режущего инструмента
(маршрута резки)
для полученного на первом этапе варианта раскроя.
На этом этапе возникают актуальные научно-практические задачи
оптимизации маршрута режущего инструмента.
Их целью обычно является минимизация стоимости и / или времени процесса резки,
связанного с обработкой требуемых контуров деталей из листового материала,
за счет определения оптимальной последовательности вырезки контуров
и выбора необходимых точек для термической врезки в листовой материал
и направления движения резака
с учетом технологических ограничений процесса резки.
Следует отметить, что современные специализированные САПР предоставляют
базовый инструментарий для решения задач рационального раскроя материалов
и подготовки УП для технологического оборудования листовой резки с ЧПУ.
Вместе с тем разработчики систем автоматизированного проектирования УП
для оборудования листовой резки с ЧПУ не уделяют должного внимания
проблеме оптимизации маршрута резки.
Существующее программное обеспечение САПР не гарантирует
получение оптимальных траекторий перемещения инструмента
при одновременном соблюдении технологических требований резки,
обусловленных необходимостью уменьшения термических деформаций материала,
которые могут приводить к существенным искажениям геометрии вырезаемых деталей.
Зачастую пользователи САПР используют интерактивный режим проектирования УП.
При этом при проектировании маршрута резки как правило применяется
стандартная техника резки <<по замкнутому контуру>> и путь инструмента строится
только с точки зрения минимизации холостых переходов режущего инструмента.
В связи с этим актуальным направлением
исследования является применение эвристических и метаэвристических подходов,
которые позволяют получить решение задачи
оптимальной маршрутизации
режущего инструмента
за приемлемое время.

\paragraph*{Степень разработанности темы исследования.}

Методы проектирования технологических процессов раскроя,
включая методы формирования маршрута резки, исследовались в работах как отечественных так и зарубежных ученых.
Хотя задача раскроя не входит в круг рассматриваемых в диссертационной работе задач,
тем не менее, следует упомянуть о значительном вкладе советских и российских исследователей
в теорию оптимизации раскроя-упаковки.
Работы в этой предметной области были начаты выдающимися учёными
В.А.~Залгаллером и Л.В.~Канторовичем
и продолжены в уфимской научной школе
Э.А.~Мухачевой и ее учениками:
А.Ф.~Валеевой, М.А.~Верхотуровым, В.М.~Картаком, В.В.~Мартыновым, А.С.~Филипповой и др.
Методологические и теоретические основы создания САПР листового раскроя были заложены
Н.И.~Гилем, А.А.~Петуниным, Ю.Г.~Стояном, В.Д.~Фроловским.

Разработкой алгоритмов для маршрутизации инструмента машин листовой резки с ЧПУ занимались,
в частности, следующие российские исследователи:
М.А.~Верхотуров, Т.А.~Макаровских, Р.Т.~Мурзакаев, А.А.~Петунин, А.Г.~Ченцов,
П.А.~Ченцов, В.Д.~Фроловский, М.Ю.~Хачай и др.,
а также зарубежные исследователи:
E.~Arkin, N.~Ascheuer, D.~Cattrysse, R.~Dewil, L.~Gambardella, J.~Hoeft, Y.~Jing, Y.~Kim , M.~Lee, S.U.~Sherif, W.~Yang и др.
В подавляющем большинстве работ
используется дискретизация граничных контуров деталей,
что позволяет применять различные хорошо разработанные математические модели дискретной оптимизации.
Можно отметить только отдельные публикации,
где оптимизационные алгоритмы ориентированы на поиск решений среди континуальных множеств.

Общеизвестно,
что задачи маршрутизации инструмента машин листовой резки с ЧПУ относятся к
NP-трудным задачам.
При этом следует отметить, что до настоящего времени не существует единой математической модели
проблемы оптимизации траектории инструмента для технологического оборудования листовой резки с ЧПУ.
Имеются отдельные группы ученых, которые занимаются исследованием частных случаев этой проблемы.
Кроме того, в рамках CAD/CAM систем,
предназначенных для проектирования раскроя и управляющих программ для машин листовой резки с ЧПУ,
есть отдельные модули, которые позволяют решать некоторые оптимизационные задачи,
например минимизацию холостого хода инструмента,
однако при этом не обеспечивают соблюдение технологических требований резки материала на машинах с ЧПУ
и не позволяют получать маршруты резки,
близкие к оптимальным с точки зрения критерия стоимости резки с учетом рабочего хода инструмента,
затрат на врезку и т.д.
К тому же, реализованные в коммерческом программном обеспечении алгоритмы
как правило не описываются в научной литературе.

В общей проблеме маршрутизации инструмента машин листовой резки с ЧПУ
можно выделить несколько классов задач:
задача непрерывной резки (CCP),
задача резки с конечными точками (ECP),
задача прерывистой резки (ICP),
задача обхода многоугольников (TPP),
задача коммивояжера (TSP)
и обобщенная задача коммивояжера (GTSP).
Любая задача оптимизации термической резки может рассматриваться как ICP,
тем не менее, литература по ICP очень скудна,
и большинство программных и научных статей вводят искусственные ограничения,
которые упрощают ICP до задач других классов.
Поиск хороших алгоритмов оптимизации или эффективного упрощения ICP
мог бы заполнить явный и существующий пробел в исследованиях.

В целом можно отметить,
что за рамками исследований отечественных и зарубежных коллег остаются 3 принципиальных момента:
\begin{enumerate}
    \item
    Разработка алгоритмов, обеспечивающих получение глобального оптимума
    оптимизационной задачи маршрутизации инструмента.
    \item
    Учет тепловых искажений заготовок при термической резке
    с целевой функцией стоимости, что приводит к не технологичным решениям
    и искажению геометрии получаемых заготовок.
    \item
    Рассмотрение задач маршрутизации из класса ICP
    и задач с набором возможных точек врезки из континуального множества.
\end{enumerate}

Применение эффективных классических метаэвристических алгоритмов дискретной оптимизации
(метод ветвей и границ, метод эмуляции отжига, метод муравьиной колонии, эволюционные алгоритмы, метод переменных окрестностей и др.)
для дискретных моделей оптимизации траектории инструмента машин с ЧПУ
возможно только при адаптации этих алгоритмов к требованиям технологических ограничений листовой резки.
Таким образом,
необходимость в создании специализированных оптимизационных постановок задач,
алгоритмов и программного обеспечения представляется
доминантой развития методов решения исследуемой оптимизационной проблемы
маршрутизации инструмента машин листовой резки с~ЧПУ.

\paragraph*{Цель работы}
заключается в разработке алгоритмов решения задачи оптимальной
маршрутизации режущего инструмента
и методик применения данных алгоритмов
в системах автоматизированного проектирования УП для машин термической резки с ЧПУ.
Для достижения поставленной в работе цели необходимо решить следующие
\textbf{задачи}:

\begin{itemize}
    \item
    Разработать эвристику поиска оптимального положения
    точек врезки в контур детали в процессе решения задачи
    непрерывной резки
    \item
    Подобрать алгоритм поиска решения задачи дискретной оптимизации
    допускающий совместное использование
    с разработанной эвристикой поиска точек врезки
    \item
    Разработать точный алгоритм решения
    обобщённой задачи коммивояжера
    с ограничениями предшествования (PCGTSP),
    позволяющий оценивать качество решений
    на основе вычисления нижней оценки
    \item
    Разработать программное обеспечение,
    реализующее разработанные алгоритмы
    \item
    Разработать методику использования алгоритмов
    оптимальной маршрутизации режущего инструмента
    в CAD/CAM-системах на примере САПР~<<Сириус>>
\end{itemize}

\paragraph*{Научная новизна результатов.}

\begin{enumerate}
    \item
    Разработан эвристический алгоритм непосредственного решения
    задачи непрерывной резки без сведения к обобщённой
    задаче коммивояжера и применения дискретизации
    \item
    Предложена схема учёта ограничений предшествования на основе
    геометрических соображений,
    эффективно уменьшающая вычислительную сложность задачи
    \item
    Разработан точный алгоритм,
    позволяющий находить нижние оценки для
    обобщённой задачи коммивояжера с ограничениями предшествования
    \item
    Сформулированы более строгие ограничения,
    позволяющие сократить дерево перебора для обобщённой
    задачи коммивояжера с ограничениями предшествования
\end{enumerate}

\paragraph*{Теоретическая и практическая значимость работы} заключается:

\begin{enumerate}
    \item
    Прямое решение задачи непрерывной резки позволяет уменьшить длину
    холостого хода режущего инструмента по сравнению с используемыми
    в настоящее время технологиями на основе решения задачи GTSP
    и её аналогов
    \item
    Новая схема учёта ограничений предшествования для задачи
    непрерывной резки, уменьшая размерность задачи и время счёта,
    позволяет получать решения задач большего размера,
    в особенности, имеющих большую вложенность контуров
    \item
    Кроме ускорения процесса проектирования УП,
    алгоритм решения задачи непрерывной резки также может применяться
    как составная часть других алгоритмов,
    что в свою очередь позволяет использовать его для решения задач
    класса GSCCP (обобщённая сегментная непрерывная резка),
    то есть искать подходы к решению самой общей задачи
    маршрутизации режущего инструмента -- ICP
    (прерывистой резки)
    \item
    Предложенные схемы учёта ограничений предшествования и поиска точек врезки
    могут сочетаться с другими алгоритмами дискретной оптимизации,
    кроме использованного в данной работе метода переменных окрестностей,
    что может привести к созданию новых алгоритмов решения задачи
    непрерывной резки
    \item
    Представленные в данной диссертационной работе необходимые условия
    глобального минимума могут быть использованы при разработке новых
    алгоритмов, например, для ограничения перебора или наоборот,
    его возобновления с целью улучшения качества решения.
    \item
    Предложенный алгоритм решения обобщённой задачи коммивояжера
    с ограничениями предшествования способен находить точные решения
    для задач, которые были ранее недоступны для точного решения
    \item
    За счёт вычисления нижних оценок,
    данный алгоритм может использоваться также для
    оценки качества решений, полученных при помощи других алгоритмов
    \item
    В предложенной схеме использованы несколько способов получения нижней оценки,
    что позволяет получать более точные результаты и сокращать объём перебора
    \item
    Допустимо использование также других способов получения оценок,
    что открывает дорогу новым теоретическим и практическим исследованиям
    \item
    Предложенные алгоритмы могут использоваться для подготовки УП для
    машин термической резки с ЧПУ в составе САПР <<Сириус>>,
    а также других CAD/CAM-систем
\end{enumerate}

\paragraph*{Методология и методы исследования.}

Методологическую базу исследования составили
фундаментальные и прикладные работы отечественных и зарубежных ученых
в области автоматизированного проектирования маршрута резки для машин листовой резки с ЧПУ,
геометрического моделирования,
разработки алгоритмов оптимальной маршрутизации,
методы вычислительной геометрии и компьютерной графики.
В качестве инструментов исследования использовались следующие методы:
анализ, синтез, классификация, формализация, математические методы обработки данных.
Оценка эффективности предложенных
методов и алгоритмов осуществлялась с помощью вычислительных экспериментов
на различных раскройных картах и тестовых примерах.
Проводилось их сравнение с результатами,
полученными при работе алгоритмов
других авторов.

\paragraph*{Положения, выносимые на~защиту:}

\begin{enumerate}
    \item
    Схема сведения задачи непрерывной резки
    с ограничениями предшествования к аналогичной
    меньшего размера и не имеющей ограничениями
    \item
    Эвристика поиска точек врезки в плоские контура,
    не использующая дискретизации
    \item
    Достаточные условия, при которых полученный маршрут
    доставляет глобальный минимум длины холостого хода инструмента
    \item
    Схема использования ограничений предшествования
    для сокращения перебора в обобщённой задаче коммивояжера
    \item
    Точный алгоритм решения обобщённой задачи коммивояжера
    с ограничениями предшествования
    с обновлением нижней границы
    \item
    Форматы файлов для обмена геометрической и маршрутной информацией
    и визуализации для использования алгоритмов оптимальной маршрутизации
    в CAD/CAM-системах
\end{enumerate}

\paragraph*{Достоверность результатов}
диссертационной работы подтверждается результатами экспериментальных исследований,
приведенными в ряде публикаций и полученными при использовании методик, алгоритмов и программных средств,
созданных при непосредственном участии соискателя.
Основные положения диссертации были представлены на международных и всероссийских научных конференциях,
опубликованы в изданиях ВАК, Scopus, WoS,
известны в научном сообществе и положительно оценены специалистами.

\paragraph*{Апробация результатов работы.}
Основные результаты работы докладывались и обсуждались на всероссийских и международных конференциях, в том числе:

\begin{itemize}
    \item
    \textit{Applications of Mathematics in Engineering and Economics},
    Созополь, Болгария, 2016 год
    \item
    \textit{MiM2016: on Manufacturing, Modelling, Management \& Control},
    Труа, Франция, 2016 год
    \item
    \textit{ASRTU 2017 International Conference on Intellectual Manufacturing},
    Харбин, Китайская Народная Республика, 2017 год
    \item
    \textit{Mathematical Optimization Theory And Operations Research},
    Екатеринбург, Россия, 2019 год
    \item
    \textit{Manufacturing Modelling, Management and Control - 9th MIM 2019},
    Берлин, Германия, 2019 год
    \item
    \textit{XVI Всероссийская научно-практическая конференция
    <<Перспективные системы и задачи управления>>},
    Домбай, Россия, 2021 год
\end{itemize}

\paragraph*{Личный вклад автора}
состоит в проведении теоретических и экспериментальных исследований
по теме диссертационной работы,
проведении аналитических расчетов на основе полученных результатов.
В опубликованных совместных работах постановка и разработка алгоритмов для
решения задач осуществлялись совместными усилиями соавторов
при непосредственном активном участии соискателя.

\paragraph*{По теме диссертационной работы}
опубликовано
8 научных работ
в рецензируемых научных журналах,
определенных ВАК РФ и
Аттестационным советом УрФУ,
из них
7 публикаций проиндексировано в международных базах данных
WoS и Scopus.
