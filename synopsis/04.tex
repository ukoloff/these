%!TEX root = ../synopsis.tex

В {\bf четвёртой главе}
рассматривается методология
использования алгоритмов решения
разных классов задачи резки в существующих
CAD/CAM-системах на примере
САПР~<<Сириус>>.
Поскольку требуется обеспечить
совместную работу программного обеспечения,
разработанного в разное время разными командами разработчиков,
чрезвычайно важными становятся вопросы
организации эффективных программных интерфейсов.

Например, для хранения и обмена геометрической информацией
в САПР~<<Сириус>>
используется унаследованный двоичный формат DBS
\autocite{bi:DBS},
который обладает важными достоинствами:
\begin{itemize}
  \item
  эффективное хранение больших данных за счёт эффективного хранения массивов вещественных
  \item
  возможность добавления новых типов записей для хранения ранее не предусмотренной информации;
  расширяемость формата
  \item
  механизм создания копий деталей и геометрических преобразований над ними.
\end{itemize}

В то же время,
работа с ним сопряжена с рядом сложностей,
прежде всего:
\begin{itemize}
  \item
  Сложность чтения двоичного формата,
  особенно в некоторых языках программирования
  \item
  Структура DBS-файла,
  предназначенная для эффективного хранения,
  сильно отличается от удобного внутреннего представления
  геометрии;
  требуется нетривиальное преобразование при чтении файла
  \item
  Формат DBS создавался в том числе для экономии памяти,
  как дисковой, так и оперативной,
  что более неактуально;
  отказ от этого позволяет резко упростить процедуры экспорта--импорта.
\end{itemize}

В рамках данной диссертационной работы
в целях упрощения взаимодействия различных подсистем
было принято решение использовать по возможности
открытые текстовые форматы для хранения и передачи данных.
В качестве основного формата был выбран формат
JavaScript Object Notation
\autocite{bi:JSON}
(JSON),
ввиду того, что
он с одной стороны имеет готовые библиотеки для чтения и записи
для практически всех современных языков
программирования,
является стандартом де-факто во многих
современных приложениях для обмена данными,
довольно прост,
настолько, что может например,
формироваться даже без использования специализированных библиотек,
но при этом достаточно выразителен.


% \lstinputlisting[
%     language=Java,
%     basicstyle=\footnotesize,
%     showstringspaces=false,
%     numbers=left,
%     captionpos=b,
%     caption=JSON-Схема для DBS-файлов
%     ]
%     {media/dbs.json}
