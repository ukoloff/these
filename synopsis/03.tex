%!TEX root = ../synopsis.tex

В {\bf третьей главе} 
рассматривается обобщённая задача коммивояжера с
ограничениями предшествования ({\it PCGTSP}) --
это хорошо известная задача комбинаторной оптимизации,
имеющая множество приложений помимо
оптимизации траектории режущего инструмента 
и привлекшая внимание многих исследователей.
К сожалению,
в отличие от задачи GTSP,
алгоритмические подходы к решению 
именно PCGTSP всё ещё малочисленны.
Алгоритм, предложенный в данной диссертационной работе,
на основе синтеза нескольких идей других исследователей,
представляет собой попытку
создать первый точный алгоритм решения
задачи PCGTSP в общем виде.

Задача определяется тройкой
$(G,\mathcal C,\Pi)$, 
где
$G=(V,E,c)$ -- взвешенный ориентированный граф на $n$
вершинах,
задающий веса $c(u,v)$ для всех своих ребер
$(u,v)\in E, u, v \in V$;
$\mathcal C=\{V_1,\ldots,V_m\}$ -- разбиение вершин
графа $G$ на $m$ кластеров;
$ \Pi = (\mathcal C, A) $ -- частичный порядок,
заданный на множестве кластеров.
