%!TEX root = ../synopsis.tex

Во {\bf второй главе}
рассматривается
применение дискретных моделей оптимизации
для проектирования УП для машин
фигурной резки с~ЧПУ,
при этом исследуется
обобщённая задача коммивояжера с
ограничениями предшествования ({\it PCGTSP}) --
хорошо известная задача комбинаторной оптимизации,
имеющая множество приложений помимо
оптимизации траектории режущего инструмента.
К сожалению,
в отличие от задачи GTSP,
алгоритмические подходы к решению
именно PCGTSP всё ещё малочисленны.
Алгоритм, предложенный в данной диссертационной работе,
на основе синтеза нескольких идей других исследователей,
представляет собой попытку
создать первый точный алгоритм
ветвей и границ для
решения задачи PCGTSP в общем виде
\cite{bi:pcgtsp2021}.

Задача определяется тройкой
$(G,\mathcal C,\Pi)$,
где
$G=(V,E,c)$ -- взвешенный ориентированный граф на $n$
вершинах,
задающий веса $c(u,v)$ для всех своих ребер
$(u,v)\in E, u, v \in V$;
$\mathcal C=\{V_1,\ldots,V_m\}$ -- разбиение вершин
графа $G$ на $m$ кластеров;
$ \Pi = (\mathcal C, A) $ -- частичный порядок,
заданный на множестве кластеров.
Для каждой вершины
$v\in V$, за $V(v)$
обозначим (единственный) кластер
$V_p\in\mathcal C$,
такой что
$v\in V_p$.

Допустимым решением задачи
$(G,\mathcal C,\Pi)$
называется тур (замкнутый путь) $T = v_1, v_2, \dots v_m$,
удовлетворяющий нескольким условиям:
  имеет длину $|T|=m$;
  начинается и заканчивается в некоторой вершине $v_1\in V_1$;
  посещает каждый кластер $V_p\in\mathcal C$ в произвольной вершине $v_p \in V_p$;
  \textit{соответствует} частичному порядку $\Pi$,
  то есть любой кластер $V_p$
  посещается маршрутом $T$
  только \textit{после}
  всех кластеров, предшествующих ему в
  $\Pi$.

Каждому решению
$T$
мы назначаем стоимость
\begin{equation}
    \label{eq:pctgsp-cost}
	cost(T) = c(v_m,v_1) + \sum_{i=1}^{m-1} c(v_i,v_{i+1})
\end{equation}

Требуется найти допустимый тур
$ T $
с минимальной стоимостью
$
cost (T) \to \min_T
$.

Ключевой идеей алгоритма является построение нижней оценки стоимости решения.
Для этого в каждой вершине дерева поиска исходная задача разделяется на две:
\begin{enumerate}
    \item
    Фиксируется {\it префикс}
    маршрута $\sigma=\{V_1, \dots, V_r\}$.
    Обозначим $c_{min}$ нижнюю границу длины
    всех путей, проходящих из вершин кластера $V_1$
    в вершины кластера $V_r$ строго в порядке $\sigma$.
    \item
    Строится вспомогательная задача $\mathcal P$,
    удалением из исходной задачи всех кластеров,
    являющиеся внутренними в $\sigma$ и
    соединением всех вершины из $V_1$ с вершинами $V_r$
    ребрами нулевого веса.
    Полученная таким образом задача всё ещё сложна,
    однако она может быть несколькими способами
    упрощена (релаксирована) $\mathcal P \to \mathcal P_{rel}$
    и для неё найдено оптимальное решение $\mathrm{OPT}(\mathcal P_{rel})$
\end{enumerate}
Нижняя оценка $LB$
находится как
\begin{equation}
    \label{eq:pcgtsp-lb}
    LB = c_{\min} + \mathrm{OPT}(\mathcal P_{rel})
\end{equation}

В качестве релаксации $\mathcal P_{rel}$ задачи $\mathcal P$
могут применяться:
\begin{itemize}
    \item
    хорошо известная трансформация Нуна и Бина,
    преобразующая задачу GTSP в обычную задачу
    коммивояжера TSP
    \item
    построение вспомогательного {\it графа кластеров} $H_1$
    с весами, индуцированными весами исходного графа $G$:
    $$
    c(V_i, V_j) = \min_{v_i \in V_i, v_j \in V_j} c(v_i, v_j)
    $$
    \item
    построение графа кластеров $H_2$,
    веса в котором также индуцированы весами путей длины 2 в исходном графе $G$:
    $$
    c(V_i, V_j) =
        \min_{\substack{v_i \in V_i, v_j \in V_j\\v_k \notin V_i \cup V_j}  }
        \frac{c(v_i, v_k) + c(v_k, v_j)}{2}
    $$
    при этом маршрут $v_i, v_k, v_j$ должен удовлетворять ограничению предшествования $\Pi$.
    Этот способ позволяет точнее учитывать взаимное расположение контуров на плоскости
    в случае Евклидовой задачи PCGTSP
    \item
    по аналогии с $H_2$ могут строиться кластерные графы с весами на основе
    путей большей ($\geqslant 3$) длины,
    однако алгоритм их расчёта существенно сложнее и
    не входит в рамки данной диссертационной работы
\end{itemize}

Наконец, для (быстрого) поиска
нижней оценки на решение задачи $TSP(\mathcal P_{rel})$
можно использовать:
\begin{itemize}
    \item
    Решение задачи о минимальном остовном дереве
    (Minimal Spanning Arborescence, MSAP),
    так как $MSAP(\mathcal P) \leqslant TSP(\mathcal P)$
    \item
    Решение задачи о цикловом покрытии (Cycle cover);
    оно находится как решение задачи о назначениях
    (Assignment Problem, AP)
    для двудольного графа,
    обе доли которого представляют $\mathcal P_{rel}$;
    опять $AP(\mathcal P) \leqslant TSP(\mathcal P)$
    \item
    в некоторых случаях можно прямо решить задачу
    коммивояжера $TSP(\mathcal P)$.
    В данной работе для этого использовался решатель Gurobi
    и MIP-модель ATSPxy
    \autocite{SARIN2005}.
\end{itemize}

Все возможные методы оценки нижней границы сведены в табл.~\ref{t:LBs},
используемые в описываемой реализации обозначены
$L_1$--$L_3$,
не используемые --
$E_1$--$E_6$.
Проведённые численные эксперименты показывают,
что часть методов
($E_1$--$E_4$)
систематически дают более слабые оценки,
а часть
($E_1$--$E_2$, $E_5$--$E_6$)
требуют значительных временных затрат по сравнению с выбранными.

\begin{table}
    \centering
    \caption{Методы оценки нижней границы}\label{t:LBs}
    \begin{tabular}[t]{*{5}{|c}|}
        \hline
        & \textbf{Нун и Бин} & $\mathbf{H_1}$ & $\mathbf{H_2}$ & $\mathbf{H_{3+}}$\\
        \hline \hline
      \textbf{AP} & $E_1$ & $\mathbf{L_1}$ & $\mathbf{L_2}$ & - \\
      \textbf{MSAP} & $E_2$ & $E_3$ & $E_4$ & - \\
      \textbf{Gurobi} & $E_5$ & $\mathbf{L_3}$ & $E_6$ & -\\
      \hline
    \end{tabular}
\end{table}

Получив описанными методами оценку нижней границы для данного префикса по формуле~\eqref{eq:pcgtsp-lb},
алгоритм принимает решение об отсечении ветви,
порождаемой текущим префиксом $\sigma$, при условии
\begin{equation}
    \label{eq:pcgtsp:cut}
    LB > UB,
\end{equation}
где $UB$ -- стоимость наилучшего известного допустимого решения исходной задачи.
В данной работе для получения $UB$
используется эвристика PCGLNS
\autocite{KKP-optima2020},
позволяющая буквально за несколько секунд получить
близкое к оптимальному решение исходной задачи PCGTSP,
что резко сокращает перебор,
позволяя отбрасывать $\approx 50\%$--$90\%$
ветвей.

Предлагаемый алгоритм ветвей и границ обходит дерево поиска,
начиная с корня
$\sigma=\{V_1\}$
в ширину,
для каждого узла
находя оценку нижней границы~\eqref{eq:pcgtsp-lb},
проверяя условие отсечения~\eqref{eq:pcgtsp:cut}
и для <<выживших>> узлов применяя процедуру ветвления.
Для этого он находит все кластера,
которые можно добавить в конец текущему префиксу,
не нарушая ограничение предшествования $\Pi$.
По окончании подсчёта префиксов одной длины,
алгоритм обновляет текущую оценку нижней границы:
$$
LB = \max(LB, LB_r)
$$
$$
LB_r = \min_{|\sigma|=r} LB(\sigma),
$$
в этот момент алгоритм может быть остановлен
по достижении нужной точности.
Если же алгоритм доходит до обработки
префиксов длины
$|\sigma|=m$,
то для них вместо оценки~\eqref{eq:pcgtsp-lb}
по известному порядку обхода кластеров $\sigma$
находится решение исходной задачи путём поиска кратчайшего тура,
посещающего кластера в этом порядке,
и из всех таких решений выбирается оптимальное.

Такой алгоритм оказывается вполне работоспособен,
однако в ходе его тестирования были выявлены некоторые недостатки:
  в некоторых задачах использованные методы
  оценки нижней границы
  (см. табл.~\ref{t:LBs})
  дают очень низкие значения;
  сведение всех путей вдоль префикса $\sigma$
  к минимальному $c_{min}$
  представляется слишком грубым;
  все префиксы $\sigma$,
  оканчивающиеся на один и тот же
  кластер $V_r$,
  но разный порядок <<внутренних>> кластеров,
  оказываются тесно связаны;
  это делает сложным попытки распараллелить
  выполнение алгоритма.

Поэтому была разработана вторая версия того же алгоритма,
устроенная по классической схеме
динамического программирования (DP)
Хелда-Карпа
\autocite{HeldKarp1962},
модифицированной для задачи PCGTSP
и дополненной стратегией отсечения~\eqref{eq:pcgtsp:cut}.

Каждое состояние DP
(запись в таблице поиска)
соответствует частичному
$v$-$u$-пути
и индексируется кортежем
$(V_1, \mathcal C',V_r,v, u)$, где
$V_1$ и $V_r$ -- начальный и конечный кластеры маршрута,
$v$ и $u$ -- начальная и конечная его вершины ($v\in V_1$, $u\in V_r$),
$\mathcal C'$ -- {\it порядковый идеал} частично упорядоченного множества $\mathcal C$
и играет здесь роль, аналогичную префиксу $\sigma$
в первой версии алгоритма.
По определению идеала
\(
    \forall V\in\mathcal C', V'\in\mathcal C\
    \left((V',V)\in A\right)
    \Rightarrow (V'\in\mathcal C')
\),
поэтому $V_1$
принадлежит произвольному идеалу
$\mathcal C'\subset\mathcal C$.
Пусть $\mathfrak I_k$
-- подмножество идеалов одного размера
$k\in\{1,\ldots,m\}$.
Очевидно,
$\mathfrak I_1=\{\{V_1\}\}$,
а значит первый слой
$\mathcal L_1$
таблицы поиска строится тривиально.
Индуктивное построение остальных слоев
описано в Алгоритме~\ref{alg:A2}.

\begin{algorithm}[t]
  \caption{DP ::  индуктивное построение таблицы поиска}\label{alg:A2}
  \textbf{Вход:} орграф $G$, частичный порядок $\Pi$, слой таблицы поиска $\mathcal L_k$\\
  \textbf{Выход:} $(k+1)$-ый слой $\mathcal L_{k+1}$
  \begin{algorithmic}[1]
  \STATE инициализация $\mathcal L_{k+1}=\varnothing$
  \FORALL{$\mathcal C'\in\mathfrak I_k$}
    \FORALL{кластер $V_l\in\mathcal C\setminus\mathcal C'$, s.t. $\mathcal C'\cup \{V_l\}\in\mathfrak I_{k+1}$}
      \FORALL{$v\in V_1$ и $u\in V_l$}
        \IF{есть состояние $S=(\mathcal C',U,v,w)\in\mathcal L_k$, s.t. $(w,u)\in E$}
        \STATE создаем новое состояние $S'=(\mathcal C'\cup\{V_l\}, V_l, v, u)$
        \STATE $S'[cost] = \min\{S[cost] + c(w,u)\colon S=(\mathcal C',U,v,w)\in\mathcal L_k\}$
        \STATE $S'[pred] = \arg\min\{S[cost] + c(w,u)\colon S=(\mathcal C',U,v,w)\in\mathcal L_k\}$
        \STATE $S'[LB] = S'[cost] + \max\{L_1,L_2,L_3\}$
          \label{alg:A2:LB}
        \IF{$S'[LB] \leqslant UB$}    \label{alg:A2:cut}
          \STATE добавляем $S'$ к $\mathcal L_{k+1}$
        \ENDIF
      \ENDIF
      \ENDFOR
    \ENDFOR
  \ENDFOR
  \RETURN $\mathcal L_{k+1}$
  \end{algorithmic}
\end{algorithm}

Для оценки производительности предложенных алгоритмов
использовалась общедоступная библиотека
PCGTSPLIB
\autocite{SALMAN2020163}:
В качестве базы сравнения использовалось решение,
полученное решателем Gurobi.
Всем алгоритмам задавалось одно и то же
допустимое решение,
полученное эвристикой
PCGLNS.
В качестве критерия остановки
использовалось снижение до 5\%
верхней оценки точности,
определяемой как
$$
gap = \frac{UB - LB}{LB}
$$

Полученные результаты эксперимента
представлены в табл.~\ref{t:data},
которая организована следующим образом:
первая группа столбцов описывает
задачу,
включая её обозначение ID,
количество вершин $n$
и кластеров $m$,
а также стоимость стартового решения $UB_0$,
полученного эвристикой PCGLNS.
Затем следуют три группы столбцов
для решателя Gurobi
и двух предлагаемых алгоритмов.
Каждая группа содержит время
счета в секундах,
наилучшее значение нижней границы
$LB$
и оценку погрешности $gap$
в процентах.
Задачи,
в которых один из предлагаемых
алгоритмов превосходит Gurobi по производительности,
выделены жирным шрифтом.

\begin{table}[p]
  \centering
  \caption{Результаты экспериментов}
  \label{t:data}
  \scriptsize
  \def\arraystretch{1.5}
  \begin{tabular}{|r|c*{12}{|r}|}
  \hline
  \multicolumn{5}{|c|}{\textit{Задача}} &
    \multicolumn{3}{c|}{\textit{Gurobi}} &
    \multicolumn{3}{c|}{\textit{Ветвей и границ}} &
    \multicolumn{3}{c|}{\textit{DP}} \\ \hline
    \multicolumn{1}{|c|}{\textit{№}} &
    \multicolumn{1}{c|}{\textit{ID}} &
    \multicolumn{1}{c|}{\textit{n}} &
    \multicolumn{1}{c|}{\textit{m}} &
    \multicolumn{1}{c|}{\textit{UB$_0$}} &
    \multicolumn{1}{c|}{\textit{Время}} &
    \multicolumn{1}{c|}{\textit{LB}} &
    \multicolumn{1}{c|}{\textit{gap, \%}} &
    \multicolumn{1}{c|}{\textit{Время}} &
    \multicolumn{1}{c|}{\textit{LB}} &
    \multicolumn{1}{c|}{\textit{gap, \%}} &
    \multicolumn{1}{c|}{\textit{Время}} &
    \multicolumn{1}{c|}{\textit{LB}} &
    \multicolumn{1}{c|}{\textit{gap, \%}} \\ \hline
    {\bf 1}  & br17.12   & 92   & 17  & 43    & 82.00 & 43    & 0.00  & {\bf 11.2} & \textbf{43}    & {\bf 0.00}    & 27.3   & 43    & 0.00    \\ \hline
    2  & ESC07     & 39   & 8   & 1730  & 0.24   & 1730  & 0.00  & 1.3   & 1726  & 0.23    & 8.37   & 1730  & 0.00    \\ \hline
    3  & ESC12     & 65   & 13  & 1390  & 3.35   & 1390  & 0.00  & 4.3   & 1385  & 0.36    & 14.99  & 1390  & 0.00    \\ \hline
    4  & ESC25     & 133  & 26  & 1418  & 10.61  & 1383  & 0.00  & 32 & 1383    & 0.00    & 60.69  & 1383  & 0.00    \\ \hline
    5  & ESC47     & 244  & 48  & 1399  & 3773   & 1064  & 4.93 & 36000 & 980   & 42.76   & 36000  & 981   & 42.61   \\ \hline
    {\bf 6} & ESC63     & 349  & 64  & 62    & 25.35 & 62    & 0.00  & 1.3   & 62    & 0.00    & {\bf 0.52}   & \textbf{62}    & {\bf 0.00}    \\ \hline
    {\bf 7}  & ESC78     & 414  & 79  & 14872 & 1278.45 & 14630 & 1.66  & 1.3   & 14594 & 1.63    & {\bf 0.68}   & \textbf{14594} & {\bf 1.63}    \\ \hline
    8  & ft53.1    & 281  & 53  & 6194  & 36000  & 5479  & 13.04  & 36000 & 4839  & 28.27   & 36000  & 4839  & 28.27   \\ \hline
    9  & ft53.2    & 274  & 53  & 6653  & 36000  & 5511  & 20.7  & 36000 & 4934  & 34.84   & 36000  & 4940  & 34.68   \\ \hline
    10 & ft53.3    & 281  & 53  & 8446  & 36000  & 6354  & 32.92 & 36000 & 5465  & 54.55   & 36000  & 5465  & 54.55   \\ \hline
    {\bf 11} & ft53.4    & 275  & 53  & 11822 & 20635  & 11259 & 5.00  & 35865 & 11274 & 4.86    & \textbf{2225}   & \textbf{11290} & \textbf{4.71}    \\ \hline
    12 & ft70.1    & 346  & 70  & 32848 &  83.70 & 31521 & 4.21  & 36000 & 31153 & 5.44    & 36000  & 31177 & 5.36    \\ \hline
    13 & ft70.2    & 351  & 70  & 33486 & 36000  & 31787 & 5.35  & 36000 & 31268 & 7.09    & 36000  & 31273 & 7.08    \\ \hline
    14 & ft70.3    & 347  & 70  & 35309 & 36000  & 32775 & 7.73  & 36000 & 32180 & 9.72    & 36000  & 32180 & 9.72    \\ \hline
    {\bf 15} & ft70.4    & 353  & 70  & 44497 & 36000  & 41160 & 8.11  & 36000 & 38989 & 14.13   & \textbf{36000}  & \textbf{41640} & {\bf 6.86}    \\ \hline
    16 & kro124p.1 & 514  & 100 & 33320 & 36000  & 29541 & 12.79 & 36000 & 27869 & 19.56   & 36000  & 27943 & 19.24   \\ \hline
    17 & kro124p.2 & 524  & 100 & 35321 & 36000  & 29983 & 17.80 & 36000 & 28155 & 25.45   & 36000  & 28155 & 25.45   \\ \hline
    18 & kro124p.3 & 534  & 100 & 41340 & 36000 & 30669 & 34.79  & 36000 & 28406 & 45.53   & 36000  & 28406 & 45.53   \\ \hline
    19 & kro124p.4 & 526  & 100 & 62818 & 36000  & 46033 & 36.46 & 36000 & 38137 & 64.72   & 36000  & 38511 & 63.12   \\ \hline
    20 & p43.1     & 203  & 43  & 22545 & 4691   & 21677 & 4.00  & 36000 & 738   & 2954.88 & 36000  & 788   & 2761.04 \\ \hline
    21 & p43.2     & 198  & 43  & 22841 & 36000  & 21357 & 6.94  & 36000 & 749   & 2949.53 & 36000  & 877   & 2504.45 \\ \hline
    22 & p43.3     & 211  & 43  & 23122 & 36000  & 15884 & 45.57 & 36000 & 898   & 2474.83 & 36000  & 906   & 2452.10 \\ \hline
    {\bf 23} & p43.4     & 204  & 43  & 66857 & 36000  & 45198 & 47.92 & 4470  & 66846 & 0.00    & {\bf 333.02} & \textbf{66846} & {\bf 0.00}    \\ \hline
    24 & prob.100  & 510  & 99  & 1474  & 36000  & 805   & 83.10 & 36000 & 632   & 133.23  & 36000  & 632   & 133.23  \\ \hline
    25 & prob.42   & 208  & 41  & 232   & 13310 & 196   & 4.86 & 36000 & 149   & 55.70   & 36000  & 153   & 51.63   \\ \hline
    {\bf 26} & rbg048a   & 255  & 49  & 282   & 24.22  & 282   & 0.00  & 0.9   & 272   & 3.68    & {\bf 0.25}   & \textbf{272}   & \textbf{3.68}    \\ \hline
    {\bf 27} & rbg050c   & 259  & 51  & 378   &  13.83  & 378   & 0.00  & {\bf 0.2}   & \textbf{372}   & \textbf{1.61}    & 0.25   & 372   & 1.61    \\ \hline
    28 & rbg109a   & 573  & 110 & 848   & 6  & 848   & 0.00  & 2407  & 812   & 4.43    & 682    & 809   & 4.82    \\ \hline
    {\bf 29} & rbg150a   & 871  & 151 & 1415  & 15  & 1382  & 2.38  & {\bf 0.4}   & \textbf{1353}  & \textbf{4.58}    & 0.53   & 1353  & 4.58    \\ \hline
    {\bf 30} & rbg174a   & 962  & 175 & 1644  & 27 & 1605  & 2.43  & {\bf 0.4}   & \textbf{1568}  & \textbf{4.85}    & 0.67   & 1568  & 4.85    \\ \hline
    {\bf 31} & rbg253a   & 1389 & 254 & 2376  & 61  & 2307  & 2.99  & {\bf 0.8} & \textbf{2269} & \textbf{4.72} & 1.42   & 2269  & 4.72    \\ \hline
    {\bf 32} & rbg323a   & 1825 & 324 & 2547  & 416  & 2490  & 2.29 & {\bf 2.0}  & \textbf{2448}  & \textbf{4.04} & 3.59   & 2448  & 4.04    \\ \hline
    33 & rbg341a   & 1822 & 342 & 2101  & 18470  & 2033  & 4.97  & 36000 & 1840  & 14.18   & 36000  & 1840  & 14.18   \\ \hline
    34 & rbg358a   & 1967 & 359 & 2080  & 17807  & 1982  & 4.95  & 36000 & 1933  & 7.60    & 36000  & 1933  & 7.60    \\ \hline
    35 & rbg378a   & 1973 & 379 & 2307  & 32205  & 2199  & 4.91  & 36000 & 2032  & 13.53   & 36000  & 2031  & 13.59   \\ \hline
    36 & ry48p.1   & 256  & 48  & 13135 & 36000  & 11965 & 9.78  & 36000 & 10739 & 22.31   & 36000  & 10764 & 22.03   \\ \hline
    37 & ry48p.2   & 250  & 48  & 13802 & 36000  & 12065 & 14.39 & 36000 & 10912 & 26.48   & 36000  & 11000 & 25.47   \\ \hline
    38 & ry48p.3   & 254  & 48  & 16540 & 36000  & 13085 & 26.40 & 36000 & 11732 & 40.98   & 36000  & 11822 & 39.91   \\ \hline
    {\bf 39} & ry48p.4   & 249  & 48  & 25977 & 36000  & 22084 & 17.62 & 18677 & 25037 & 3.75    & {\bf 14001} & \textbf{25043} & {\bf  3.73}    \\ \hline
  \end{tabular}
\end{table}

Для 13 из 39 задач (33\%)
один из предложенных алгоритмов
показал лучшую производительность
(в 7 случаях в точности решения,
в 12 случаях во времени счета),
Для 10 постановок, включая одни из самых больших
\textit{rbg323a} и \textit{rbg358a}
(1825 и 1967 вершин соответственно)
было получено решение с точностью 5\%.
В целом, хотя Gurobi демонстрирует в среднем чуть лучшую производительность,
предложенные алгоритмы за редким исключением,
показывают вполне сопоставимые результаты.
