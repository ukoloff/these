%!TEX root = ../synopsis.tex

В {\bf третьей главе} 
рассматривается обобщённая задача коммивояжера с
ограничениями предшествования ({\it PCGTSP}) --
это хорошо известная задача комбинаторной оптимизации,
имеющая множество приложений помимо
оптимизации траектории режущего инструмента 
и привлекшая внимание многих исследователей.
К сожалению,
в отличие от задачи GTSP,
алгоритмические подходы к решению 
именно PCGTSP всё ещё малочисленны.
Алгоритм, предложенный в данной диссертационной работе,
на основе синтеза нескольких идей других исследователей,
представляет собой попытку
создать первый точный алгоритм решения
задачи PCGTSP в общем виде.

Задача определяется тройкой
$(G,\mathcal C,\Pi)$, 
где
$G=(V,E,c)$ -- взвешенный ориентированный граф на $n$
вершинах,
задающий веса $c(u,v)$ для всех своих ребер
$(u,v)\in E, u, v \in V$;
$\mathcal C=\{V_1,\ldots,V_m\}$ -- разбиение вершин
графа $G$ на $m$ кластеров;
$ \Pi = (\mathcal C, A) $ -- частичный порядок,
заданный на множестве кластеров.
Для каждой вершины 
$v\in V$, за $V(v)$ 
обозначим (единственный) кластер 
$V_p\in\mathcal C$, 
такой что
$v\in V_p$. 

Допустимым решением задачи
$(G,\mathcal C,\Pi)$
называется тур (замкнутый путь) $T$,
удовлетворяющий условиям:
\begin{itemize}
    \item 
    имеет длину $|T|=m$
    \item
    начинается и заканчивается в некоторой вершине $v_1\in V_1$
    \item 
    посещает каждый кластер $V_p\in\mathcal C$
    \item
	каждое ребро
	$(v_i, v_j)$ в $T$ 
	(кроме ребра $(v_m,v_1)$) 
	удовлетворяет ограничению предшествования,
	то есть
	 $(V(v_i),V(v_j))\in A$.
\end{itemize}

Каждому решению
$T=v_1, v_2, \ldots, v_m$
мы назначаем стоимость
\begin{equation}
    \label{eq:pctgsp-cost}
	cost(T) = c(v_m,v_1) + \sum_{i=1}^{m-1} c(v_i,v_{i+1})
\end{equation}

Требуется найти допустимый тур 
$ T $ 
с минимальной стоимостью
$$ 
cost (T) \to \min_T 
$$

