\usepackage{cmap}

\usepackage[utf8]{inputenc}
\usepackage[T2A]{fontenc}
\usepackage[english,russian]{babel}

\usepackage{indentfirst}
\setlength{\parindent}{2.5em}

% Убираем переносы в заголовках
\usepackage[raggedright]{titlesec}

% Короткое тире для ненумерованных списков
% ГОСТ 2.105-95, пункт 4.1.7, требует дефиса, но так лучше смотрится
\renewcommand{\labelitemi}{\normalfont\bfseries{--}}

%%% Выравнивание и переносы %%%
%% http://tex.stackexchange.com/questions/241343/what-is-the-meaning-of-fussy-sloppy-emergencystretch-tolerance-hbadness
%% http://www.latex-community.org/forum/viewtopic.php?p=70342#p70342
\tolerance 1414
\hbadness 1414
\emergencystretch 1.5em % В случае проблем регулировать в первую очередь
\hfuzz 0.3pt
\vfuzz \hfuzz
%\raggedbottom
%\sloppy                 % Избавляемся от переполнений

% Висячие строки
\clubpenalty=10000
\widowpenalty=10000
\brokenpenalty=4991
\raggedbottom

% Надписи рисунков и таблиц
\usepackage{caption}

\captionsetup{
  font=small,
  labelsep=period,
  justification=centering,
}

% https://tex.stackexchange.com/a/78777
% Подписи к таблицам
\DeclareCaptionFormat{hfillstart}{\hfill#1#2#3}
\captionsetup[table]
{
  format=hfillstart,
  labelsep=newline,
}

% Подписи к подрисункам
\captionsetup[subfigure]{
  labelformat=brace,
}

\usepackage[caption=false]{subfig}
\renewcommand\thesubfigure{\asbuk{subfigure}}
