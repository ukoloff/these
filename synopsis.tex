\documentclass[14pt]{extarticle}

\usepackage{cmap}

\usepackage[utf8]{inputenc}
\usepackage[T2A]{fontenc}
\usepackage[english,russian]{babel}

\usepackage{indentfirst}
\setlength{\parindent}{2.5em}

% Убираем переносы в заголовках
\usepackage[raggedright,tiny]{titlesec}

% Короткое тире для ненумерованных списков
% ГОСТ 2.105-95, пункт 4.1.7, требует дефиса, но так лучше смотрится
\renewcommand{\labelitemi}{\normalfont\bfseries{--}}

\input{include/hyphen.tex}
% Надписи рисунков и таблиц
\usepackage{caption}

\captionsetup{
  font=small,
  labelsep=period,
  justification=centering,
}

% https://tex.stackexchange.com/a/78777
% Подписи к таблицам
\DeclareCaptionFormat{hfillstart}{\hfill#1#2#3}
\captionsetup*[table]
{
  format=hfillstart,
  labelsep=newline,
}

% Подписи к подрисункам
\captionsetup*[subfigure]{
  labelformat=brace,
}

\usepackage[caption=false]{subfig}
\renewcommand\thesubfigure{\asbuk{subfigure}}


\usepackage{geometry}

\geometry{
    a4paper,
    top=2cm,
    bottom=2cm,
    left=2.5cm,
    right=1cm,
    nomarginpar,
    % showframe
}


\begin{document}

\section*{ОБЩАЯ ХАРАКТЕРИСТИКА РАБОТЫ}

\paragraph*{Актуальность темы исследования.}
В машиностроении и других отраслях промышленности в раскройно-заготовительном производстве 
большая часть продукции изготавливается из листового материала на технологическом оборудовании термической резки 
с числовым программным управлением 
(далее --- ЧПУ). 
К такому оборудованию относятся, в частности, 
машины лазерной, плазменной, кислородной резки. 
Машины лазерной резки с ЧПУ имеют широкое применение, 
что обусловлено возможностью обработки многих видов материалов различной толщины, 
высокой скоростью резки, возможностью обработки контуров различной сложности и хорошего качества реза, 
адаптации к постоянным изменениям номенклатуры выпускаемой продукции. 
Как известно, применение систем автоматизированного проектирования (САПР), 
предназначенных для разработки управляющих программ (УП) 
для машин листовой резки с ЧПУ, обеспечивает автоматизацию процесса проектирования УП. 
Проектирование УП для технологического оборудования листовой резки 
состоит из нескольких этапов. 
Первый этап предполагает предварительное геометрическое моделирование заготовок 
и разработку раскройной карты листового материала. 
На этапе проектирования раскроя возникает известная задача оптимизации фигурного раскроя листового материала, 
которая с точки зрения геометрической оптимизации
относится к классу трудно решаемых проблем раскроя-упаковки 
({\it Cutting \& Packing}). 
На следующем этапе проектирования УП осуществляется процесс назначения траектории 
перемещения режущего инструмента 
(маршрута резки) 
для полученного на первом этапе варианта раскроя. 
На этом этапе возникают актуальные научно-практические задачи 
оптимизации маршрута режущего инструмента. 
Целью этих задач обычно является минимизация стоимости и / или времени процесса резки, 
связанного с обработкой требуемых контуров деталей из листового материала, 
за счет определения оптимальной последовательности вырезки контуров 
и выбора необходимых точек для термической врезки в листовой материал 
с учетом технологических ограничений процесса резки. 
Следует отметить, что современные специализированные САПР предоставляют 
базовый инструментарий для решения задач рационального раскроя материалов 
и подготовки УП для технологического оборудования листовой резки с ЧПУ. 
Вместе с тем разработчики систем автоматизированного проектирования УП 
для оборудования листовой резки с ЧПУ не уделяют должного внимания 
проблеме оптимизации маршрута резки. 
Существующее программное обеспечение САПР не гарантирует 
получение оптимальных траекторий перемещения инструмента 
при одновременном соблюдении технологических требований резки, 
обусловленных необходимостью уменьшения термических деформаций материала, 
которые могут приводить к существенным искажениям геометрии вырезаемых деталей. 
Отметим также, что пользователи САПР в основном используют интерактивный режим проектирования УП. 
При этом при проектировании маршрута резки зачастую применяется 
стандартная техника резки «по замкнутому контуру» и путь инструмента строится 
только с точки зрения минимизации холостых переходов режущего инструмента.

\section*{ОСНОВНОЕ СОДЕРЖАНИЕ РАБОТЫ}

\section*{ОСНОВНЫЕ РЕЗУЛЬТАТЫ И ВЫВОДЫ}

\section*{СПИСОК ПУБЛИКАЦИЙ ПО ТЕМЕ ДИССЕРТАЦИИ}

\end{document}